%% The following is a directive for TeXShop to indicate the main file
%%!TEX root = diss.tex

\chapter{Introduction}
\label{ch:Introduction}

This thesis is a theoretical study on how to control quantum dynamics of quasiparticles in a periodic lattice potential. 
Although some work presented in the thesis is quite general and can be applied to any system with a periodic potential, 
the main subjects of the research are systems of ultracold atoms and molecules as their experimental manipulation is 
relatively easy. The control schemes proposed in the thesis also relate closely with the recent developments of laser 
cooling and trapping of atoms and molecules. Therefore, to give readers a broad context in which to understand the 
thesis, this chapter presents an introductory overview of the field of ultracold molecules. In \autoref{sec:ultracold}, we briefly introduce ultracold  molecules and explain why they are interesting. After
that, in \autoref{sec:control}, we focus on the controllability property of ultracold systems and review recent works 
based on controlling the internal and external degrees of freedoms of  molecules. 
Finally, we give an outline of the thesis in \autoref{sec:outline} and discuss the connection and contribution of  
this thesis to the field of ultracold atoms and molecules. 

\section{Opportunities offered by ultracold temperatures}
\label{sec:ultracold}

%what is ultracold atoms and molecules?
%Controlling materials has always been the main theme of scientific quest for generations of people. For thousands of
%years, our control over materials focused on macroscopic scale. With the advance of technology, nowadays we are 
%able to put microscopic objects into scientific investigations. Especially, the control of atoms and molecules, the basic 
%building blocks of materials, is attracting the attention of scientists worldwide. 
Cooling gasesous ensembles of atoms and molecules to extremely low temperatures is  revolutionizing the 
field of atomic, molecular and optical physics. 
%In the past three decades, there has been great success in creating ultracold atomic gases, which  revolutionized the 
%field of atomic and optical physics\cite{southwell2002, chu2002, anglin2002, rolston2002, burnett2002, udem2002,
%monroe2002}.
In the past three decades, there has been great success in creating ultracold atomic gases\cite{southwell2002, chu2002}, 
leading to many exciting applications,  
including Bose$-$Einstein condensation of atomic gases\cite{anglin2002}, nonlinear and quantum atom 
optics\cite{rolston2002}, collisions at ultracold temperatures\cite{burnett2002}, all-optical atomic
clock\cite{udem2002},
and quantum information processing with atoms and photons\cite{monroe2002}. 
Prompted by this success with 
ultracold atoms, researchers are now putting great effort to extend the method of ultracold physics to molecules,
creating the emerging field of ultracold molecules. 
As basic building blocks of materials, molecules,  just like atoms,  at ultracold temperatures are expeted to have  an great
impact on different areas of science and technology such as precision measurements of fundamental constants,  quantum simulations of condensed matter systems, quantum information processing, precise control of molecular
dynamics and ultracold chemistry\cite{our-njp-review, friedrich2009, schnell2009, Bell2009, krems2010cold, Ni2009,
Jin2011, Jin2012, quemener2012, Baranov2012}. 


The word ``ultracold'' merits some explanation. By convention, a distinction is made between two ranges of 
temperature $T$: cold means that 1 mK $<$ $T$ $<$ 1 K; ultracold means that $T < $ 1 mK.   Different from its literal 
meaning, the word  ``ultracold'' doesn't directly 
refer to temperatures  as most ultracold systems under experimental investigations are not in thermodynamic 
equilibrium and no temperature $T$ can be ascribed to them.  Nevertheless, the temperature $T$ is used as a way to 
quantify the kinetic energy of particles according to the relation $E_{\rm kin} = k_{\rm B} T$ where $k_{\rm B}$ is the 
Boltzmann constant.   So the actual meaning of ``ultracold'' is ``ultra-slow'', refering to the translational motion of 
particles.  
The small kinetic energies of  ultracold atoms or molecules have some important consequences, which is summaried below.

First, ultracold temperatures allow for novel macroscopic quantum states of matter. A particle in the gaseous phase behaves as a quantum-mechanical 
wavepacket with the extension on the order of its de Broglie wavelength, given by
\oneline{
\Lambda = \frac{h}{m v} \mbox{ or } \frac{h}{\sqrt{2\pi m k_{\rm B} T} } \ . 
}
At ultracold temperatures, the de Broglie wavelength is larger than its counterparts at room temperatures by several orders of magnitude, 
and is comparable to or larger than the mean distance between particles in gaseous phases. Under these conditions,
the particles in a condensate becomes indistinguishable and their wavefunctions overlap, leading to the formation of
a macroscopic coherent matter wave, that is, Bose-Einstein condenstate\cite{ketterle2002, anglin2002} for bosonic 
particles in which all particles occupy the lowest quantum state, and quantum degenerate Fermi gases\cite{lev2002, stefano2008} for fermionic 
particles in which every accessible energy level is occupied by exactly one particle due to the Pauli exclusion principle. 

Second, ultracold temperatures lead to new collision phenomena.
Because of the large de Broglie wavelengths of ultracold particles, their low-energy collison is 
qualitatively different from the collision at thermal temperature. At ultracold temperatures, particles
do not have well-defined trajectories in phase space, the spherical harmonics $Y_{lm}(\theta, \phi)$, also called 
partial waves with $l = 0, 1, 2, \cdots$, are used to describe their collisions. In the absence of external field, as $T 
\rightarrow 0$, the quantum threshold regime is reached and only the lowest allowed partial wave contributes to the 
collision\cite{wigner1948, krems2005}. For collisions between reactive particles at ultracold temperatures, the quantum phenomena is
even more dramatic. As the kinetic energy is so small, the tunneling under reaction barriers becomes the 
dominant mechanism of chemical reactions, giving temperature-independent reaction rate constants that 
can be very large at zero Kelvin\cite{balakrishnan2001, bodo2002, bodo2004, weck2004}. Moreover, quantum statistics 
governs the chemistry at ultracold temperatures. For example, depending on 
whether the colliding molecules are in the same internal quantum state or not, the collisions can be dominated by $s$-
wave collisions or $s$-wave and $p$-wave collisions, giving very different reaction rates\cite{ospelkaus2010}. 

Third, ultracold temperatures enable numerious control possibilities. It has been a long-sought goal to control atomic and 
molecular dynamics by utilizing their interactions with external fields. However, at normal temperatures, the thermal 
motion of atoms and molecules randomize their encounters and diminishes the effect of external fields. As the temperature
approaches a few $\mu$K, the kinetic energy of ultracold molecules becomes so small  that it is even less than 
the energy splittings of hyperfine structures. In this case, an external field of modest strength can produce an perturbation
in energy that is larger than the translation energy of molecules, creating dramatic changes in the collisions properties of
molecules. which allows for numerious studys on controlling ultracold collision by external field\cite{krems2005}. The 
slow motion of particles at ultracold temperature makes the trapping and confinement of them relatively easy. 

Fourth, 
 have a profound influence. 

why are they unqiue?

%\section{Ultracold systems on Optical lattices}
%\label{sec:opticalLattices}
%
%what's optical lattices?



\section{Controllability of ultracold systems}
\label{sec:control}


\section{Physics of ultracold molecules on optical lattices}
\label{sec:opticalLattice}

\section{Thesis outline}
\label{sec:outline}


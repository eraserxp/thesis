\chapter{Outlook}
\label{ch:conclusion}

Trapping ultracold atoms and molecules in optical lattices has opened an exciting frontier of condensed 
matter research\cite{Baranov2012}. In the limit of strong trapping fields that completely suppress particle tunneling
in optical lattices, an artifical crystal of atoms and molecules, called the Mott insulator phase, appears. These artificial crystals possess 
unique properties -- in particular, they offer  the ability to address single lattice sites\cite{atom-mott1, atom-mott2} and the possibility for controlling interparticle interactions with external electric or magnetic fields\cite{quemener2012, Baranov2012},  making them a very good platform to
study collective phenomena\cite{our-njp-review, quemener2012, Baranov2012}. The thesis, by utilizing  the controllability of the artificial crystals of ultracold particles, explores theoretically the mechanisms for  
controlling the quantum dynamics
of quasiparticles. 
%In the following, we discuss the limitations of the thesis and point out possible future research directions.  

%\section{Summary of the thesis}
%
%In \autoref{ch:biexciton}, we studied a particular kind of quasiparticles -- rotational Frenkel excitons in a 
%periodic lattice potential. These quasiparticles are the collective excitonic modes of polar molecules 
%trapped on an optical lattice in the Mott insulator state, and are induced by the dipole-dipole interaction which couples the 
%rotational states of different molecules. As superpositions of elementary excitations localized on different
% lattice sites, the rotational excitons  bear a strong resemblance to the electronic excitons in organic solids\cite{agranovich}.
%Because of the controllability of the ultracold system of polar molecules, the properties of the rotational excitons can also 
%be manipulated. In \autoref{ch:biexciton}, we were interested in the binding of the excitons. 
%We showed that the application of an moderate electric field,  through mixing rotational states of different parity, can give rise 
%to an non-linear dynamic interaction $D$ between the rotational excitons.  This dynamic interaction doesn't occur in molecular crystals 
%with inversion symmetry and has so far eluded the experimental observation with only one exception\cite{frenkelxx, 
%frenkelxx2}. For the aritifical crystal of polar molecules, we found that the dynamic interaction is always attractive and its 
%strength can be tunned by the external DC field. This leads to controllable formation of biexciton states and tunable binding
%energy as demonstrated numerically for a 1D array of LiCs molecules on an optical lattice. We also obtained the 
%two-excitation spectra of the rotational excitons and derived analytical expressions for the wavefunction of biexciton states 
%using the nearest-neighbor approximation. In an effort to extend the theoretical model of exciton binding, we calculated 
%three-excitation spectra of rotational excitons and observed the three-body bound states of the excitons.
%To make our theoretical study of biexciton states more relevant to experimental investigation, we proposed an nonoptical way to create the rotational biexciton states, avoiding the difficulty involved in 
%direct excitation of biexcitons. This method makes use of the resonance between the high-energy (N=2) excitonic states 
%and the biexciton states of low-energy (N=1) excitons and can produce biexcitons with high efficiency. 
%
%
%
%In \autoref{ch:energy-transfer}, we studied a more general type of quasiparticles -- collective excitations in an array of 
%coupled monomers and considered the generic problem of excitation energy transfer in the array of monomers. In our research, the excitations can be of any type and the array can be ordered or disordered.  We 
%showed that the energy transfer through an array of coupled quantum monomers can be controlled by applying a 
%transient external potential which modifies the phase of the quantum states of the individual monomers. The success of the method relies 
%on the two very different time scales in the quantum evolution of the many-body states: one time scale is related to the 
%local excitation of a single monomer, and the other depends on the interaction between two monomers that is responsible
%for the propagation of excitation energy. If the former time scale is much faster than the latter, it is possible to find a suitable
%local perturbation to a single monomer that is adiabatic with respect to the former time scale but is sudden
%with respect to the latter time scale.  In an ordered crystal, if such perturbations are applied to give different phases to 
%different monomers, the quasimoment of the collective excitation is modified and its propagation behavior is influenced
%as well. Our research showed that different phase transformations can accelerate or decelerate quantum energy transfer and 
%spatially focus delocalized excitations onto different parts of those ordered arrays. On the other hand, for a completely disordered array,  random scattering at numerous lattice sites disturbs the above phase transformations. In this case,  inspired by the ``transfer matrix'' methods for focusing of a 
%collimated light beam in opaque medium\cite{opaque-1, Gigan-TMeasure-PRL10, Mosk-NPhot10, Cizmar-NPhot10, 
%Silberberg-11, Chatel-Focusing-11, Lagendijk-Focusing-11, zhenia-11, cui-11, kim-11},
%we developed another scheme of phase transformation that 
%can achieve effective focusing of a delocalized excitation in the presence of strong disorders. To make connection with the current study of  
%ultracold molecules,  we also considered possible experimental implementations of the proposed technique in an array of 
%ultracold atoms or molecules trapped on an optical lattice and demonstrated the feasibility of the phase transformations. 
%  
%
%
%In \autoref{ch:greenfunc}, I developed a numerically efficient method to calculate the two-particle Green's functions in a 
%lattice with arbitrary disorders. The method can be viewed as an extension of Berciu's method \cite{Berciu2010, Berciu2011, 
%Berciu2012} to an aperiodic lattice. By grouping Green's functions into different set of vectors, this method rewrites the 
%equation of motion of Green's function to a recursion relation that links three consecutive vector. Then  based on physical 
%reasoning, certain boundary conditions can be assumed. This initialize the recursive calculations and all the Green's 
%functions can be calculated.  The recursive method is presented in the form of a generic algorithm which can be easily 
%adapted to long-range interactions and high-dimensional systems. As an application of the method, I proposed 
%to use the Green’s function to study the tunneling of biexciton states through
% impurities.
%
%
%
%\section{Future research directions}
%%

The research on exciton binding, as presented in \autoref{ch:biexciton}, proposed a conceptually simple scheme to control 
exciton--exciton interactions and offered a new tool for various fundamental directions of research. In the future, the research might be extended in the following directions.
\begin{enumerate}
%
\item In the current research, we 
assumed the optical lattices are deep and ignored the translational motion of molecules within the trap. Future work might 
consider the scenario of Frenkel excitons coupled with lattice phonons in shallow optical lattices to form so-called polarons. It would be very interesting to explore whether the attractive dynamic interaction 
between excitons leads to the formation of bipolarons. 
%
\item In \autoref{ch:biexciton}, we only considered  simple 
$^{1}\Sigma$ molecules. In the future, it might be interesting to consider more complicated molecules that can interact 
with more types of external fields. For example, an ensemble of $^{2}\Sigma$ molecules in the ro-vibrational
ground state trapped on an optical lattice has been shown to exhibit collective spin excitations. It would be interesting to 
see whether the spin exciton-exciton interaction in such a system can be controlled with both electric and magnetic fields and if there exist
bound states of the spin excitons. 
%
\item Our research has shown the existence of  the two-body and three-body bound states of excitons under certain 
conditions. However, this does not mean that biexciton and triexciton will appear at any exciton concentrations. 
When more excitons are present, the combination of biexciton and exciton might be unstable to triexciton formation
and triexciton $+$ exciton might be unstable to bigger bound complexes.  So it is necessary to extend our theoretical
model to handle $n$-body bound states and to investigate the fundamental limits of exciton clustering. 
\end{enumerate}

The research on exciton propagation, as presented in \autoref{ch:energy-transfer}, proposed a very general way to control
the collective excitation energy transfer in ordered and disordered arrays. As the excitations can be of any type, the control 
scheme may be applied to a wide range of systems. Future work may proceed in the following directions.
\begin{enumerate}
%
\item Our study has so far ignored the interactions between excitons. This is justified for the case of the rotational 
excitations in small external fields as the exciton--exciton interaction is small compared with the hopping interaction. 
However as \autoref{ch:biexciton} shows, the dynamic interaction between excitons can be comparable and even a few 
times larger than the hopping interaction at certain electric field strengths. In this situation, it is important to investigate
 if the control scheme still works. Particularly, it would be very interesting to see whether we can control the propagation
 of the biexciton states. 
%The scheme  to various types of quasiparticles. For 
%example, the biexciton state found in \autoref{ch:biexciton} is a very promising candidate. As the biexciton state is a 
%bound state due to the dynamic interaction, it would be very interesting to see how the exciton-exciton interaction will 
%influence the control of energy transfer. 
%
\item Up to this point, we have only considered the feasibility of the control scheme in the system of ensembles of ultracold
 atoms and molecules trapped on optical lattices. It might be also worthwhile to consider other types of systems. For 
 example, an ensemble of molecules can be trapped in rare gas matrices or in solid hydrogen matrices and form a disordered array with the lattice separation of only a few nm. It would be interesting to see whether our control scheme
 can work in the arrays with such a small lattice constant. 
\end{enumerate}

The method to calculate Green's functions, as presented in \autoref{ch:greenfunc}, is very generic. Future research might
use it to study all sorts of many-body physics in the presence of strong disorders. For example, it is possible to
use the method to calculate the local density of states in a  high-dimensional disordered lattice, which might yield useful information about the 
existence of bound states of quasiparticles in the presence of strong disorders. 


 
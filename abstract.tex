%% The following is a directive for TeXShop to indicate the main file
%%!TEX root = diss.tex

\chapter{Abstract}

This thesis describes research on controlling the dynamics of quasiparticles in 
 periodic and disordered lattice potentials. Working with model systems of arrays of atoms and molecules trapped in  optical lattices, 
I focus on, but not limited to, the rotational excitons of polar molecules and propose to use external fields to 
control the binding and propagation of quasiparticles. 

First, we study the binding of  rotational excitons in a periodic potential. We show that non-linear interactions of such excitons can
 be controlled by an electric field. The exciton-exciton interactions can be tuned
 to induce exciton pairing, leading to the formation of biexcitons and three-body bound states of excitons. In addition, we propose a non-optical way to create biexcitons by splitting a high-energy exciton into two 
low-energy excitons. 

Second, we present schemes to control the propagation a collective
 excited state in ordered and disordered aggregates of coupled particles. We demonstrate that the dynamics of these excitations can be controlled by
 applying a transient external potential which modifies the phase of the quantum
 states of the individual particles. The method is based on an interplay of adiabatic
 and sudden time scales in the quantum evolution of the many-body states. We show
 that specific phase transformations can be used to accelerate or decelerate quantum
 energy transfer and spatially focus delocalized excitations onto different parts of
arrays of quantum particles. For the model systems of atoms and molecules trapped in an optical lattice, we consider possible experimental implementations
 of the proposed technique and study the effect of disorder, due to the presence of
 impurities, on its fidelity. We further show that the proposed technique can allow
 control of energy transfer in completely disordered systems.
 

Finally, in an effort to refine the theoretical tools to study dynamics of quasiparticles, we extend calculations of lattice 
Green's functions to disordered systems. We develop a generic algorithm that can 
be easily adapted to systems with long-range interactions and high dimensionalities. As an application of the method, we propose 
to use the Green's function to study the tunneling of biexciton states through
 impurities.


% Consider placing version information if you circulate multiple drafts
%\vfill
%\begin{center}
%\begin{sf}
%\fbox{Revision: \today}
%\end{sf}
%\end{center}

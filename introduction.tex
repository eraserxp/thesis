%% The following is a directive for TeXShop to indicate the main file
%%!TEX root = diss.tex

\chapter{Introduction}
\label{ch:Introduction}

This thesis is a theoretical study on controlling the quantum dynamics of quasiparticles in periodic and disordered lattice potentials. 
Although some work presented in the thesis is quite general and can be applied to any system with a periodic or disordered lattice potential, 
the main systems considered here are ultracold atoms and molecules, as their experimental manipulation is 
relatively easy. The control schemes proposed in the thesis are also related closely to recent developments in the laser 
cooling and trapping of atoms and molecules. Therefore, to give the reader a broad context in which to understand the 
thesis, this chapter presents an introductory overview of the field of ultracold atoms and molecules. In \autoref{sec:ultracold}, I briefly introduce ultracold  atoms and molecules and discussed the unique features of ultracold ensembles. After
that, in \autoref{sec:control}, I discuss the controllability of ultracold molecules. In
 \autoref{sec:opticalLattice}, I introduce a particular ultracold system with polar molecules trapped on optical lattices.  
Finally, I outline the research presented in the thesis in \autoref{sec:outline}. 

\section{Implications of ultracold temperatures}
\label{sec:ultracold}

 
Cooling gaseous ensembles of atoms and molecules to extremely low temperatures has revolutionized the 
field of atomic, molecular and optical physics. 
The development of experimental techniques for cooling atoms to ultracold temperatures has led to many ground-beaking applications\cite{southwell2002, chu2002},  
including the fundamental studies of Bose$-$Einstein condensation of weakly interacting particles\cite{anglin2002}, nonlinear and quantum atom 
optics\cite{rolston2002}, collisions at ultracold temperatures\cite{burnett2002}, the development of all-optical atomic
clock\cite{udem2002},
and quantum information processing with atoms and photons\cite{monroe2002}. 
Inspired by this success with 
ultracold atoms, researchers are now aiming to extend the experimental methods to cooling molecules.
The experimental work with ultracold molecules is likewise expected to have  a great
impact on different areas of science and technology, which is covered by numerous review articles\cite{our-njp-review, friedrich2009, schnell2009, Bell2009, krems2010cold, Ni2009,
Jin2011, Jin2012, quemener2012, Baranov2012}. 
%such as precision measurements of fundamental constants,  quantum simulations of condensed matter systems, quantum information processing, precise control of molecular
%dynamics and ultracold chemistry\cite{our-njp-review, friedrich2009, schnell2009, Bell2009, krems2010cold, Ni2009,
%Jin2011, Jin2012, quemener2012, Baranov2012}. 
%The word ``ultracold'' merits some explanation. 
By convention, a distinction is made between two ranges of 
temperature $T$: cold 1 mK $<$ $T$ $<$ 1 K, and ultracold $T < $ 1 mK.   
%Different from its literal 
%meaning, the word  ``ultracold'' doesn't directly 
%refer to temperatures  as most ultracold systems under experimental investigations are not in thermodynamic 
%equilibrium and no temperature $T$ can be ascribed to them.  Nevertheless, the temperature $T$ is used as a way to 
%quantify the kinetic energy of particles according to the relation $E_{\rm kin} = k_{\rm B} T$ where $k_{\rm B}$ is the 
%Boltzmann constant.   So the actual meaning of ``ultracold'' is ``ultra-slow'', referring to the translational motion of 
%particles. 
At ultracold temperatures, the translational motion of particles has a velocity that can be as low as a few cm/s, in 
sharp contrast with the speed of a few hundred m/s at room temperatures. 
The small kinetic energies associated with the ultra-slow translational motion have some important consequences, summarized below.
%These ultracold temperatures have some important consequences, which is summarized below.

First, ultracold temperatures allow for novel macroscopic quantum states of matter. A particle in the gaseous phase behaves as a quantum-mechanical 
wavepacket with the extension on the order of its de Broglie wavelength, given by
\oneline{
\Lambda = \frac{h}{m v} \ . %\mbox{ or } \frac{h}{\sqrt{2\pi m k_{\rm B} T} } \ . 
}
For the ultra-slow velocity $v$ at ultracold temperatures, the corresponding de Broglie wavelength is larger than at room temperatures by several orders of magnitude, 
and is comparable to or larger than the mean distance between particles in the gas phase. Under these conditions,
the particles in a gaseous ensemble become indistinguishable and their wavefunctions overlap heavily, leading to the formation of
a macroscopic coherent matter wave, that is, a Bose-Einstein condensate\cite{ketterle2002, anglin2002} for bosonic 
particles in which all particles occupy the lowest quantum state, and a quantum degenerate Fermi gas\cite{lev2002, stefano2008} for fermionic 
particles of same kind in which every accessible energy level is occupied by exactly one particle due to the Pauli exclusion principle. 

Second, ultracold temperatures lead to new collision phenomena.
Because of the large de Broglie wavelengths of ultracold particles, their low-energy collision is 
qualitatively different from the collision at thermal temperatures. At ultracold temperatures, particles
do not have well-defined trajectories in phase space, the spherical harmonics $Y_{lm}(\theta, \phi)$, also called 
partial waves with $l = 0, 1, 2, \cdots$ that represent the angular momenta of the rotational motion of the collision complex,  are commonly used to describe the collisions. In the absence of external fields, as $T 
\rightarrow 0$, the quantum threshold regime is reached and only the lowest allowed partial wave contributes to the 
collision\cite{wigner1948, krems2005}. For collisions between reactive particles at ultracold temperatures, the quantum phenomena are
even more dramatic. As the kinetic energy is so small, the tunneling under reaction barriers becomes the 
dominant mechanism of chemical reactions, giving temperature-independent reaction rate constants that 
can be very large at zero Kelvin\cite{balakrishnan2001, bodo2002, bodo2004, weck2004}. Moreover, quantum statistics 
governs reactive collisions at ultracold temperatures. For example, depending on 
whether the colliding molecules are in the same internal quantum state, the collisions can be dominated by 
$s$-wave collisions or $s$-wave and $p$-wave collisions, giving very different reaction rates\cite{ospelkaus2010}. 

Third, ultracold temperatures enable numerous control possibilities. It has been a long-sought goal to control atomic and 
molecular dynamics by utilizing their interactions with external fields. However, at normal temperatures, the thermal 
motion of atoms and molecules randomizes their encounters and diminishes the effect of external fields. As the temperature
approaches a few $\mu$K, the kinetic energy of ultracold molecules becomes so small  that it is even less than 
the energy splittings of hyperfine structures of molecules. In this case, an external field of moderate strength can produce a perturbation
in energy that is larger than the translational energy of molecules, creating dramatic changes in the collisions properties of
molecules\cite{krems2005}. Normally, ultracold particles are obtained by first cooling them to cold temperatures and then loading them into a trap for further evaporative cooling\cite{metcalf1999}. 
This inspired the development of a variety of 
traps, such as electrostatic traps\cite{bethlem2000}, AC electric traps\cite{vanVeldhoven2005}, optical and microwave
traps\cite{optical-lattice-review, deMille2004}, and magnetic and magneto-optical traps\cite{hogan2008, 
vanhaecke2002, wang2004}. Within the traps, the thermal motion of particles is restricted, so their
interaction with external fields can be more precisely controlled. This leads to unprecedented precision of 
spectroscopic measurement because of the much longer  interrogation time\cite{vandeMeerakker2005, 
gilijamse2007, campbell2008}. Because the shapes of the traps can be changed, the trapped ultracold atoms and 
molecules can be confined in restricted geometries. This may be used for quantum simulations of 
fundamental many-body systems\cite{Baranov2012} and various setups for quantum computing\cite{brennen1999,
jaksch1999, pachos2003, kay2004, micheli2006, buchler2007a, bloch2008}.  


\section{External field control of ultracold molecules}
\label{sec:control}
Perhaps the most important feature of ultracold atoms and molecules is their controllability via interactions with  
external fields. Since the corresponding research field is so vast, as reviewed by Ref.\cite{lemoshko2013}, 
%here we limit ourself to giving only a brief outline on the external field control of ultracold molecules. 
%For ultracold atoms, much of their controllability comes from the magnetic field tunable Feshbach resonances\cite{chin2010}, 
%which can be used to precisely control the
%scattering length of collisions. The Feshbach resonance occurs when the kinetic energy  of the colliding particles 
%matches the energy of a bound state of the collision complex, giving a huge change of the scattering length. When 
%the bound state of the collision complex has a different magnetic moment from that of two free atoms, a magnetic 
%field can be used to change the energy of the bound state, thereby shifting the feshbach resonance points. As the scattering
%length is proportional to the effective interparticle interaction in ultracold quantum gases, tuning the scattering length is
%actually changing the interaction between atoms. Thus Feshbach resonances provide a mechanism to tune the effective
%interparticle interaction in ultracold gases\cite{chin2010, kohler2006} and leads to numerous applications, including
%magneto-association to produce ultracold molecules\cite{ni2008, danzl2010}, Bose-Einstein condensate of fermionic dimers\cite{zwierlein2003, regal2004}, and the crossover from Bose-Einstein condensate to Bardeen-Cooper-Schreifer superconductor in dilute gases\cite{chin2004}. 
we  limit the discussion to ultracold molecules as they are more relevant for our research.  
Compared with atoms, molecules have additional vibrational and rotational degrees of freedom, which leads to rich 
control possibilities. In the following, I briefly outline the possibilities from the aspect of single molecule and from
the aspect of intermolecular interaction separately. 

First of all, the internal states of ultracold molecules can be selected by their interaction with  
external fields. This is very important because it enables the production of a near quantum degenerate gas of 
molecules in their rovibronic ground state\cite{ni2008, danzl2010}, which provides a basis for other research that 
requires a supply of  ultracold molecules. 
Recently, it has been demonstrated that a rovibronic ground-state molecular quantum gas  can be prepared in a single 
hyperfine state or in an arbitrary coherent superposition of hyperfine states\cite{ospelkaus2010} by using a two-photon
microwave transition in the presence of a magnetic field. This control over hyperfine structures, together with the 
manipulation of electronic, vibrational, and rotational states, provides the full control over all internal degrees of
freedom of molecules. As quantum information can be encoded 
into the internal states of molecules,  the ability to manipulate the internal states constitutes the basis of 
quantum computation. For instance, as presented in Ref.\cite{pellegrini2011},  universal quantum gates can be 
implemented using the hyperfine levels of ultracold 
heteronuclear polar molecules in their electronic, vibrational, and rotational ground state. These high-fidelity 
logic gates, driven by microwave pulses between hyperfine states, offer versatile encoding possibilities as a 
consequence of the richness of the energy structures and the state mixing of molecules in external fields. The precise control of internal states also 
facilitates quantum simulation of many-body systems using ultracold molecules. For example, it has been shown that hyperfine structures of polar molecules in an one-dimensional 
lattice can be controllably accessed and manipulated as a resource for generating complex quantum dynamics
\cite{Carr2}.  

Moreover, the permanent electric dipole moment of polar molecules enables the tuning of long-range intermolecular 
interactions. The electric dipole moment describes the separation of positive and negative electrical charges
in a molecule and is given by
\oneline{
\boldsymbol{\mu} = \sum_{n} q_n \mathbf{r}_{n} \ ,
}
where $\mathbf{r}_{n}$ gives the positions of charges $q_n$ and the sum is over all charges in the molecule including 
the electrons and nuclei. For molecules in a particular electronic and vibrational state, it is convenient to define the
dipole operator as $\mathbf{d} = \matelement{\psi}{\boldsymbol{\mu}} {\psi}$ by integrating over the electronic and
vibrational wavefunction $\psi$. The dipole moment operator only couples states with different
parity. Take a simple $^{1}\Sigma$ molecule for example,  its rotational eigenstate $\ket{J, M}$  has a well-defined parity 
determined by the angular momentum quantum number $J$ in the following way:
\oneline{
P \ket{J, M} = (-1)^{J} \ket{J, M} \ ,
}
where $P$ is the parity operator. It follows that molecules in a rotational eigenstate have no net dipole moment as $\matelement{J, M}{\mathbf{d}} {J, M} = 0$ in the laboratory frame. 
%At ultracold temperatures, in consideration of stability against inelastic collisions, 
%molecules are usually prepared in their electronic, vibrational and rotational ground state, therefore they possess no net
If molecules are prepared in their electronic, vibrational and rotational ground state, they possess no net
dipole moment and interact at large intermolecular distances via a van der Waals attraction which falls off as $\frac{1 }{r^6}$ with respect to the
interparticle separation $r$. With the application of electric fields that mix rotational states of different parity, static or 
oscillating dipoles are induced in molecules, leading to nonzero interaction -- dipole-dipole interaction, which falls off
as $\frac{1}{r^3}$ and dominates over the van der Waals interaction at long range. The dipole-dipole interaction can be
modified by external fields to change the strength and shape of the interaction potential between 
polar molecules\cite{micheli2007, buchler2007a}. This tunability, combined with the anisotropy characteristic of the
dipole-dipole interaction, has inspired a large body of research on the condensed matter theory of dipolar quantum 
gases\cite{baranov2008, lahaye2009, trefzger2011, Baranov2012}. 




\section{Ultracold polar molecules on optical lattices}
\label{sec:opticalLattice}

In the past few years, our group has been interested in ensembles of ultracold polar molecules trapped on optical lattices.  Optical lattices are
 periodic potentials formed by the standing wave patterns of  laser 
 beams\cite{Jessen1996, phillips1998}. 
% These periodic potentials use ac-stark shift to trap and organize ultracold atoms and molecules in 
% crystal-like structures, creating artificial crystals. 
The simplest optical lattice is an one-dimensional lattice created by  superimposing two counter-propagating laser 
beams 
with the same frequency, intensity and polarization.  If the two laser beams propagate in the $+z$ and $-z$ directions respectively, the 
resulting total electric field is given by
\multiline{
\mathbf{E}(z, t) &=& E_0 \cos( k z - \omega t) \mathbf{\hat{e}} + E_0 \cos( - k z - \omega t) \mathbf{\hat{e}} \nonumber \\
&=& 2 E_0 \cos(\omega t) \cos(k z)\mathbf{\hat{e}} \ ,
}
where $\mathbf{\hat{e}}$ is the polarization vector. This oscillating electric field induces an oscillating dipole moment in a
particle and at the same time interacts with the dipole moment, creating a time-averaged periodic potential $V(z) \propto E_0^2  \cos^2 (k z)$ whose troughs can be used as microtraps in the $z$ direction.  The depth of those microtraps is 
proportional to the light intensity and their effective sizes can be as small as several tens of nm. If the two laser 
beams are Gaussian beams, their Gaussian intensity profiles in the $xy$ plane will provide an additional weak radial 
confinement potential. This can be combined with the periodic standing wave pattern to form an one-dimensional array of  disk-like 
trapping potentials. Optical lattices with higher dimensions can be created by using additional pairs of laser beams. For example,
adding a pair of counter-propagating laser beams along the $x$ direction to the current one-dimensional lattice can 
produce a two-dimensional array of traps, each of which looks like a tube in the $y$ direction. These tube-like potentials 
can be broken into small segments along the $y$ direction by adding a third pair of  laser beams in that direction.
This leads to a three-dimensional periodic potential. 
In addition to dimensionality, the geometry of optical lattices can also be changed by adjusting the angles between different pair of laser beams. For instance, if three pairs of laser beams are arranged in the $xy$ 
plane with an angle $2\pi/3$ with respect to each other and they are all polarized along the $z$ direction, the resulting
lattice potential $V(x, y) \propto 3 + 4\cos(3 k x/2)\cos(\sqrt{3}ky/2) + 2\cos(\sqrt{3}ky)$ is a triangular lattice in two
dimensions. Apart from dimensionality and geometry, the trap depths and the lattice separations of optical lattices can also be controlled
by changing the laser intensities and wavelengths. This kind of flexibility doesn't exist in natural solid state crystals. 
Moreover, compared with natural solid state crystals with the lattice constants on the order of a few angstroms, optical lattices have much larger lattice site separations on the order of an optical wavelength (micron), which open the possibility to address individual sites in optical lattices\cite{atom-mott1, atom-mott2}. 
%Because of the ability to change the geometry of the set of laser beams and their intensity, it is possible to realize optical lattices of various geometry and dimensionalities with tunable trap depth\cite{Jessen1996,
%  phillips1998}. With polar molecules trapped inside, optical lattices provide a very versatile platform for studying
Due to all the above features, optical lattices with polar molecules trapped inside, provide a very versatile platform for studying controllable many-body 
  phenomena\cite{Baranov2012} and for implementing new schemes of quantum computing\cite{demille, yelin2006, ortner2011}. 
  
The low-energy physics of particles trapped in an optical lattice can be described by the Hubbard Hamiltonian
\oneline{
H = - \sum_{i, j, \sigma} J_{ij}^{\sigma} b^{\dagger}_{i, \sigma} b_{j, \sigma} + \sum_{i, j, \sigma, \sigma'} 
\frac{U_{i j}^{\sigma \sigma'}}{2} b^{\dagger}_{i, \sigma} b_{i, \sigma} b^{\dagger}_{j, \sigma'} b_{j, \sigma'} \ ,
\label{eqn:hubbardHam}
}
where $b^{\dagger}_{i, \sigma} (b_{i, \sigma})$ are the creation (annihilation) operators for a particle at site $i$ in the
internal state $\sigma$, $J_{ij}^{\sigma}$ describes the hopping of a particle from site $j$ to site $i$, and $U_{i j}^{\sigma \sigma'}$ is the onsite ($i=j$) or offsite ($i\neq j$) interaction between the particle at site $i$ in state $\sigma$ and the particle at site $j$ in state $\sigma'$. By tuning the trap depth, the ratio between $J$ and $U$ can
be controlled. When  the particle number is commensurate with the 
number of lattice sites and the trap depth is deep enough such that $J \ll U$, a phase transition from a superfluid phase
or a  Fermi liquid to a Mott insulator can be 
observed\cite{Greiner2002, spielman2007, haller2010, joerdens2008, Schneider2008}. In the Mott insulator phase, the tunneling of particles between sites
is suppressed and the number of particles at a site is an integer value. For the system that our group is interested -- ensembles of 
polar molecules trapped on optical lattices, a lot of effort has been devoted to exploring exotic quantum phases (see Ref.\cite{Baranov2012} and the references therein) because of the long-range and anisotropic character of the 
dipole-dipole interaction between trapped polar molecules.  
In those studies, many novel phase diagrams, associated with different rates of particle tunneling and various types of 
interparticle interactions, have been discovered.

We instead focus on the Mott insulator phase and explore the tunability of the system of ultracold molecules in the 
context of collective excitations. The form of the relevant Hamiltonian is still the same as \autoref{eqn:hubbardHam}, but the creation
and annihilation operators are defined with respect to the quasiparticles -- the molecular excitations rather than the real 
particles.  So $J$ now describes the propagation of the excitation energy in the crystal and $U$ represents the interaction
between two or more molecular excitations. This is a new line of research which may be exploited to study collective 
excitations in new interaction regimes. For example, previous work by our group members has demonstrated
that the artificial crystals of polar molecules in the Mott insulator phase can be used to investigate controllable exciton-impurity interactions\cite{felipe},
to control collective spin excitations\cite{perez-rios2010}, to engineer open quantum systems with tunable coupling to the bath\cite{felipe-polarons}, and to study new polaron models\cite{felipe-arxive-polaron}. 


\section{Thesis outline}
\label{sec:outline}

The current thesis extends the work on collective excitations in several new directions. 
%In the following, I outline the research conducted in the thesis chapter by chapter. 
%The outline of the whole thesis is the following: \autoref{ch:background} introduces the 
%background material that can help the reader to understand the research of the thesis; \autoref{ch:conclusion} summarizes
%the thesis and gives comments on future research directions; the remaining chapters describe the research work of 
%the thesis. 

%Unlike the previous 
%studies of our group which considered  only one excitation, the research presented in \autoref{ch:biexciton} increased the number of 
%excitations to two and three and investigated the binding mechanism of these excitations. 
\autoref{ch:biexciton} investigates the interactions between multiple collective excitations, and
in particular the binding mechanism of these excitations. 
We study a particular kind of quasiparticles -- rotational Frenkel excitons in a 
periodic lattice potential. These quasiparticles are the collective excitonic modes of polar molecules 
trapped on an optical lattice in the Mott insulator phase, and are induced by the dipole-dipole interaction which couples the 
rotational states of different molecules. 
%As superpositions of elementary excitations localized on different
% lattice sites, the rotational excitons  bear a strong resemblance to the electronic excitons in organic %solids\cite{agranovich}.
We show that the application of a moderate electric field,  through mixing rotational states of different parity, can give 
rise to a non-linear dynamic interaction $D$ between the rotational excitons. The dynamic interaction is always attractive and its 
strength can be tuned by the external DC field. This leads to controllable formation of biexciton states with tunable binding
energy, as demonstrated numerically for a 1D array of LiCs molecules on an optical lattice. We also obtain the 
two-excitation spectra of the rotational excitons and derive analytical expressions for the wavefunction of biexciton states 
using the nearest-neighbor approximation. In an effort to extend the theoretical model of exciton binding, we calculate 
three-excitation spectra of rotational excitons and observe the three-body bound states of the excitons.
To make our theoretical study of biexciton states more relevant to experiments, we propose an nonoptical way to create the rotational biexciton states, avoiding the difficulty involved in 
direct excitation. This method makes use of the resonance between the high-energy ($N=2$) excitonic states 
and the biexciton states of low-energy ($N=1$) excitons and can produce biexcitons with high efficiency. 

\autoref{ch:energy-transfer} focuses on the propagation properties of excitations. This chapter proposes a new scheme to control the 
energy transfer in ordered and disordered crystals. 
%In \autoref{ch:energy-transfer}, we studied a more general type of quasiparticles -- collective excitations in an array of 
%coupled monomers and considered the generic problem of excitation energy transfer in the array of monomers. In our research, the excitations can be of any type and the array can be ordered or disordered.  
We 
show that the energy transfer through an array of coupled quantum monomers can be controlled by applying a 
transient external potential which modifies the phase of the quantum states of the individual monomers. The success of the 
method relies 
on two very different time scales in the quantum evolution of the many-body states, namely the fast time scale 
corresponding to the excitation of a single monomer and the slow one associated with the excitation hopping between 
monomers.  Because of these two different time scales,  it is possible to find a suitable
local perturbation to a single monomer that is adiabatic with respect to the monomer excitation but is sudden
with respect to the excitation hopping. 
%one time scale is related to the 
%local excitation of a single monomer, and the other depends on the interaction between two monomers that is responsible
%for the propagation of excitation energy. If the former time scale is much faster than the latter, it is possible to find a suitable
%local perturbation to a single monomer that is adiabatic with respect to the former time scale but is sudden
%with respect to the latter time scale. 
 In an ordered crystal, if such perturbations are applied to give different phases to 
different monomers, the momentum of the collective excitation is modified and its propagation behavior is influenced
as well. Our work shows that different phase transformations can accelerate or decelerate quantum energy transfer and 
spatially focus delocalized excitations onto different parts of those ordered arrays. On the other hand, for a completely disordered array,  random scattering at numerous lattice sites disturbs the above phase transformations. In this case,  inspired by the ``transfer matrix'' methods for focusing  a 
collimated light beam in an opaque medium\cite{opaque-1, Gigan-TMeasure-PRL10, Mosk-NPhot10, Cizmar-NPhot10, 
Silberberg-11, Chatel-Focusing-11, Lagendijk-Focusing-11, zhenia-11, cui-11, kim-11},
we develop another scheme of phase transformation that 
can achieve effective focusing of a delocalized excitation in the presence of strong disorder. To make connection with the current study of  
ultracold molecules,  we also consider possible experimental implementations of the proposed technique in an array of 
ultracold atoms or molecules trapped on an optical lattice, and demonstrate the feasibility of the phase transformations. 

\autoref{ch:greenfunc} extends the method\cite{Berciu2010, Berciu2011, Berciu2012} for computing the Green's 
functions for particles in periodic potentials to calculate 
two-particle 
Green's functions in disordered crystals. This provides a powerful tool to study the quantum dynamics of quasiparticles in a disordered potential. 
% and describe a method to study the tunneling of biexciton states through impurities. 
This chapter shows that by grouping Green's functions into different set of vectors, the 
equation of motion of Green's function can be rewritten as a recursion relation that links three consecutive vectors. 
Provided that certain boundary conditions are assumed, the recursion relation enables the recursive calculations of Green's 
functions. I present the recursive method in the form of a generic algorithm which can be easily 
adapted to systems with long-range interactions and high dimensionalities, and describe its advantage over the conventional methods to calculate Green's functions. As an application of the method, I propose
to use the Green's function to study the tunneling of biexciton states through
 impurities.



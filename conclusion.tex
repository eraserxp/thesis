\chapter{Conclusion}
\label{ch:conclusion}

Trapping ultracold atoms and molecules in optical lattices has opened an exciting frontier of condensed 
matter research\cite{Baranov2012}. In the limit of strong trapping fields that completely suppress particle tunneling
in optical lattices, an artificial crystal of atoms and molecules, called the Mott insulator phase, appears. These artificial crystals possess 
unique properties -- in particular, they offer  the ability to address single lattice sites\cite{atom-mott1, atom-mott2} and the possibility for controlling interparticle interactions with external electric or magnetic fields\cite{quemener2012, Baranov2012},  making them a very good platform to
study collective phenomena\cite{our-njp-review, quemener2012, Baranov2012}. The thesis, by utilizing  the controllability of the artificial crystals of ultracold particles, explores theoretically the mechanisms for  
controlling the quantum dynamics
of quasiparticles.  

\section{Summary of the thesis}

The control schemes presented in this thesis can be categorized into two major types: 1) mixing internal states by external fields; 2) changing the phase of internal states by temporal perturbations. 

The first type
of control schemes is commonly used to control the interparticle interactions in the field of ultracold atoms and molecules. 
For example, it has been shown that both the shape and strength of the dipole-dipole interaction between molecules
can be controlled by external fields\cite{micheli2007}, which inspires numerous study on using molecules trapped on optical
lattices for quantum simulation\cite{Barnett2006, micheli2006, Brennen2007, Buchler2007, Carr, Carr2, Trefzger2010, Kestner2011, gorshkov, gorshkov2}. Instead, the current thesis studied the possibility of using the external fields to control
inter-quasiparticle interactions rather than interparticle interactions.  In the case of the rotational excitation of polar molecules 
trapped in an optical lattice, the dipole-dipole operator leads to 
different kinds of inter-quasiparticle interactions, namely the hopping interaction that is responsible for excitation 
propagation, the attractive dynamic interaction that can induce exciton binding, and the conversion interaction that can split
a high-energy exciton into two low-energy excitons. These interactions are associated with different dressed rotational states of molecules, which correspond to different mixing of bare rotational states in external fields.  So the different inter-quasiparticle
interactions respond differently to the change of external fields. 
This thesis showed that the ratio between the hopping and the 
dynamic interaction can be tuned by an electric field to induce the formation of biexciton and triexciton states with tunable 
binding energies, and found that the conversion interaction leads to an non-optical way to create biexciton states from the high-energy ($N=2$)
excitonic states under the resonance condition when a high-energy
excitonic state have the same energy as two low-energy ($N=1$) excitonic states. These results demonstrated that the 
physics of dipole-dipole interactions in the context of quasiparticles is also rich and opened up the possibility to control the 
binding and creation of collective excitation modes of polar molecules trapped on optical lattices. 

The second type of control schemes is reminiscent of the techniques used for strong-field alignment and orientation of 
molecules in the gas phase\cite{alignment-review}. The essential idea is to use an external field to perturb the internal states
of molecules for a short time, adding phases to the wavefunctions of the internal states. This perturbation doesn't change
the interparticle interactions as the internal states remain the same except that their phases are changed. Though the phase
changes are seemingly unimportant, they play an important role in determining the dynamics of quasiparticles. Since a collective excitation in an array of coupled monomers is a superposition of internal states of monomers, any phase change
of the internal states will translate to a change of quasimomentum of  the collective excitation, and influence the excitation
energy propagation in the array. The thesis showed that specific phase transformations can be used to accelerate or 
decelerate excitation energy transfer and spatially focus delocalized excitations in arrays of quantum particles.  The 
proposed control scheme is a very general as the excitations can be of any type and the array can be ordered and 
disordered. This thesis also demonstrated the feasibility of the control scheme in the system of ultracold atoms and 
molecules trapped on optical lattices. All the above results showed that the phases of individual monomer state in an array of coupled monomers can be used to manipulate the quantum energy transfer in the array, which 
pointed out a new way to control the dynamics of quasiparticles.  


In addition, the thesis extended calculations of lattice Green's functions to disordered systems. The proposed method
is  numerically more efficient than conventional methods and may be used to study the dynamics of quasiparticles in 
strongly disordered potentials. 

\section{Limitations and future research directions}
%%
%In this section, I discuss the limitations of the thesis and explain how going beyond those limitations can  lead to even more interesting results. 

Conventionally, the 1D and 2D crystals of ultracold atoms and molecules are created in 3D optical lattices. In a 3D 
optical lattice, since the intensity of every laser pair can be tuned independently, the trap depths along the $x$,
$y$ and $z$ directions can be adjusted separately. By making the trap depths along the $x$ and $y$ directions so deep
that the particle tunneling along these two directions are completely suppressed, an effective 1D crystal can be constructed
in which the trapped particles are only allowed to move in the periodic potentials along the $z$ direction. Similarly, an effective 2D crystal
can be created by suppressing the motion of trapped particles along just one direction. 
%In these crystals, it is the trapped atoms or molecules that are moving.
 In  this thesis, I study the excitations, rather than real particles,  trapped in 1D and 2D dimensions. 
To confine excitations in reduced dimensionalities, it is not enough to use the effective 1D and 2D systems described 
previously since excitations can still propagate along a direction even if the tunneling of real particles along that direction is 
completely suppressed. 
Theoretically speaking, we need a real 1D or 2D crystal in a 3D optical lattices, where only the sites along a single line 
or on a single plane are occupied by atoms or molecules. The problem is that such an 1D or 2D crystal is  difficult to obtain in experiments as it requires high precision to remove particles from the lattice sites that don't belong to the
required arrays. In practices, one may still use a 3D optical lattice with one polar molecule per site in the Mott insulator 
phase. To approximate the 1D and 2D crystals required in the thesis, the wavelengths of laser beams in some directions 
can be increased, producing larger lattice separations along those directions and reducing the strength of coupling between 
molecules in the corresponding dimensions. This leads to suppressing of the excitation propagation along those directions,
creating the quasi-1D and quasi-2D arrays. 
% with the required 1D and 2D arrays along the dimensions with the smaller lattice separations.  
However, limited by the available wavelengths of the laser beams used to form optical lattices, the ratio between the large 
and small lattice separations along different directions is typically about 2$\sim$3. This means that the interaction 
between nearest neighbors along the direction with the large lattice constant will be of the same magnitude as the 
interaction between next nearest neighbors or next next nearest neighbors along the direction with small lattice constant.
As shown in \autoref{ch:energy-transfer}, the long-range interaction between molecules can play a role in determining
the dynamics of quasiparticles, therefore it is necessary to include the weak interactions along the suppressed dimension(s). 
In addition, because the dipole-dipole interaction is anisotropic and the relative orientation of the dipole moments with respect to the intermolecular axis are different, the weak interactions along the suppressed dimension(s) generally have different signs from the strong interaction within the 1D and 2D arrays. So it would be interesting to investigate whether the results obtained for purely 1D and 2D arrays are valid for the quasi-1D and quasi-2D crystals and especially the effect of 
the anisotropic weak interactions. 


 In the thesis, 
I have assumed that the optical lattices are deep and ignored the translational motion of molecules within the trap. In fact,
going beyond this assumption may also lead to interesting results. 
As Ref. \cite{felipe-polarons} shows, the rotational excitons can interact with the translational motion of molecules to
form polarons. In the usual case, the lattice deformation associated with polaron may lead to an effective attraction 
between polarons, producing bipolarons which are bound states of two polarons.  In the current artificial crystals of polar 
molecules, the exciton-exciton dynamic interaction induces an extra interaction between polaron, it would be interesting to
explore whether this leads to the formation of bipolarons. If this new type of bipolarons exist, their propagation behavior 
might be different from that of the conventional bipolarons induced by lattice deformation.



%In the thesis, 
%I have assumed that the optical lattices are deep and ignored the translational motion of molecules within the trap. In fact,
%going beyond this assumption may lead to interesting results related to high-temperature superconductivity. 
%As Ref. \cite{felipe-polarons} shows, the rotational excitons can interact with the translational motion of molecules to
%form polarons. 
%%Understanding the mechanism for high-temperature superconductivity is a great theoretical challenge in condensed matter physics\cite{alexandrov2003}.
%Since bipolarons, bound states of two polarons, has been suggested as a possible mechanism for high temperature superconductivity\cite{alexandrov2003}, 
% it would be very interesting to explore whether the attractive dynamic interaction between excitons leads to the formation of bipolarons in the current artificial crystals of polar molecules. Though the bipolarons will be different from those in high-temperature superconductors, studying the properties of the bipolarons in our system, especially its propagation behavior, may shed some light on the theory of high-temperature superconductivity.  

% Future work might 
%consider the scenario of Frenkel excitons coupled with lattice phonons in shallow optical lattices to form so-called polarons. It would be very interesting to explore whether the attractive dynamic interaction 
%between excitons leads to the formation of bipolarons. 


% In \autoref{ch:biexciton}, we only considered  simple 
%$^{1}\Sigma$ molecules. In the future, it might be interesting to consider more complicated molecules that can interact 
%with more types of external fields. For example, an ensemble of $^{2}\Sigma$ molecules in the ro-vibrational
%ground state trapped on an optical lattice has been shown to exhibit collective spin excitations. It would be interesting to 
%see whether the spin exciton-exciton interaction in such a system can be controlled with both electric and magnetic fields and if there exist
%bound states of the spin excitons. 


\autoref{ch:biexciton} has shown the existence of  the two-body and three-body bound states of excitons under certain 
conditions. However, this does not mean that biexciton and triexciton will appear at any exciton concentrations. 
When more excitons are present, the combination of biexciton and exciton might be unstable to triexciton formation
 and triexciton $+$ exciton might be unstable to bigger bound complexes.  So it is necessary to extend our theoretical
model to handle $n$-body bound states and to investigate the fundamental limits of exciton clustering. 


The work on energy transfer presented in the thesis has so far ignored the interactions between excitons. This is justified for the case of the rotational 
excitations in small external fields as the exciton--exciton interaction is small compared with the hopping interaction. 
However as \autoref{ch:biexciton} shows, the dynamic interaction between excitons can be comparable and even a few 
times larger than the hopping interaction at certain electric field strengths. In this situation, it is important to investigate
 if the control scheme still works. Particularly, it would be very interesting to see whether we can control the propagation
 of the biexciton states. 
 
 
%The scheme  to various types of quasiparticles. For 
%example, the biexciton state found in \autoref{ch:biexciton} is a very promising candidate. As the biexciton state is a 
%bound state due to the dynamic interaction, it would be very interesting to see how the exciton-exciton interaction will 
%influence the control of energy transfer. 

% Up to this point, we have only considered the feasibility of the control scheme in the system of ensembles of ultracold
% atoms and molecules trapped on optical lattices. It might be also worthwhile to consider other types of systems. For 
% example, an ensemble of molecules can be trapped in rare gas matrices or in solid hydrogen matrices and form a disordered array with the lattice separation of only a few nm. It would be interesting to see whether our control scheme
% can work in the arrays with such a small lattice constant. 

%The method to calculate Green's functions, as presented in \autoref{ch:greenfunc}, is very generic. Future research might
%use it to study all sorts of many-body physics in the presence of strong disorders. For example, it is possible to
%use the method to calculate the local density of states in a  high-dimensional disordered lattice, which might yield useful information about the 
%existence of bound states of quasiparticles in the presence of strong disorders. 


 
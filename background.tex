
\chapter{Background material}
\label{ch:background}
This chapter  briefly outlines the background knowledge that is directly related to the research in later chapters.


\section{Review of angular momentum theory}
\label{sec:angularMomentum}

In this section, we briefly review of the theory of angular momentum. Instead of giving a comprehensive 
picture as various books\cite{edmonds-book, rose-book, brink-book, zare-book, silver-book, kleinman-book, Varshalovich-book, RotSpect} have already done, we take a pragmatic approach that is more concerned about intuitive 
understanding, discussing only the concepts and ideas that are relevant to research in this thesis. 

\subsection{Group and class in group theory}  
\label{sec:groupTheory}

Symmetry is very important in that the laws of physics are easier to understand if the underlying symmetry are 
appreciated. The theory of angular momentum is essentially the study of certain symmetry in quantum mechanics. 
As group theory is an important tool that is used to determine symmetry, we introduce the concepts of group
and class in this subsection.

A set of operations $\{ A, B, C, \cdots \}$ is called a group if it satisfies the following properties:
\begin{itemize}
\item there is an identity operation $I$ in the set such that $A I = A$
\item  each operation $A$ has an inverse operation $A^{-1}$ in the same set such that $A A^{-1} = I$
\item  the multiplication of two operations is also a operation in the same set
\item  operations are associative such that $(AB)C = A(BC)$
\end{itemize}
To simplify the expression of all the symmetry operations in a group, the concept of class is introduced. A class of a 
group is defined as all the elements in the group that are conjugated to each other. By ``conjugated'', we means that 
 any two elements $A$ and $B$ in a class satisfy
\oneline{
X^{-1} A X = B
}
for any element $X$ in the group. A group can be divided into classes and then we can work with classes rather than 
all the symmetry operations in a group. 

\subsection{Rotation operators and spherical harmonics}

As the theory of angular
 momentum concerns about the symmetry under rotations, we start by defining the rotation operator in quantum 
 mechanics. Given a rotation $R$, we associate a rotation operator 
 $D(R)$ that transform a state from the original system to the rotated system:
 \oneline{
 \ket{\psi}_{R} = D(R) \ket{\psi} \ .
 }
To construct the rotation operator, we denote the rotation operator $D(\mathbf{\hat{n}}, d\phi)$ for infinitesimal 
rotation through angle $d\phi$ about an axis $\mathbf{\hat{n}}$. Because $D(\mathbf{\hat{n}}, d\phi)$ must be linear
in $d\phi$ for small rotations and becomes identity operator when $d\phi = 0$, it is useful to define the angular
 momentum operator $J_{\mathbf{\hat{n}}}$ about the axis $\mathbf{\hat{n}}$ as 
 \oneline{
 J_{\mathbf{\hat{n}}} \equiv i \hbar\lim_{\phi \rightarrow 0} \frac{D(\mathbf{\hat{n}}, \phi) - 1}{\phi} \ . \label{eqn:Jdef} \ ,
 }
such that
\oneline{
D(\mathbf{\hat{n}}, d\phi) = 1 - i\left(\frac{ J_{\mathbf{\hat{n}}}  }{\hbar} \right) d\phi \ . \label{eqn:rotationOperator}
}
From the above equation, we can regard $J_{\mathbf{\hat{n}}}$ as the generator of rotation in the sense that $ d\phi  \cdot J_{\mathbf{\hat{n}}} f$ gives the increment of a general function $f$ after the infinitesimal 
rotation $R$ through $d\phi$ about the axis $\mathbf{\hat{n}}$.
By compounding successively infinitesimal rotations about the same axis, we can obtain the finite rotation operator
\multiline{
D( \mathbf{\hat{n}}, \alpha) &=& \lim_{m\rightarrow \infty} \left[ 1- i\left(\frac{ \alpha}{m}\right) \left( \frac{ J_{\mathbf{\hat{n}}} }{\hbar}\right) \right]^m \nonumber \\
&=& \exp\left( \frac{ - i \alpha  J_{\mathbf{\hat{n}} } }{\hbar} \right) \nonumber \\
&=& 1 - \frac{i\alpha J_{\mathbf{\hat{n}} } }{\hbar} - \frac{1}{2!} \frac{ \alpha^2 J_{\mathbf{\hat{n}} }^2}{\hbar^2} + \cdots 
} 
It is obvious that the set of all rotations satisfy the four conditions of a group (see 
\autoref{sec:groupTheory}). Correspondingly, 
 the infinite set of rotation operators also satisfy the same conditions and thus form a group which is called the full 
rotation group. 


Rotations have the following important property:  rotations about the same axis commute whereas rotations about 
different axes do not.  This property leads to the commutation relationship of the angular momentum operator $J$. 
To illustrate this point, let's consider rotations $R_x$, $R_y$, and $R_z$ about the $x$, $y$ and $z$ axis respectively. In 
coordinate space, these rotations can be represented by $3\times 3$ matrices. For instance, the rotation of angle $\phi$ about the $z$ axis can be expressed as
\oneline{
R_{z} (\phi) = \mat{\cos\phi & -\sin\phi &  0 \\ \sin\phi& \cos\phi& 0 \\ 0& 0& 1}
} 
In the case of an infinitesimal angle $\varepsilon$, the above equation can be written as
\oneline{
R_{z} (\varepsilon) = \mat{1-\frac{\varepsilon^2}{2} & -\varepsilon &  0 \\ \varepsilon& 1-\frac{\varepsilon^2}{2}& 0 \\ 0& 0& 1}
} 
By Comparing the effect of a $y$-axis rotation followed by an $x$-axis rotation with that of an $x$-axis rotation 
followed by a $y$-axis rotation, we can show that 
\oneline{
R_x (\varepsilon) R_y (\varepsilon) - R_y (\varepsilon) R_x (\varepsilon) = R_z (\varepsilon^2) - I \ . \label{eqn:rotationCommutation}
}
Assuming the corresponding rotation operators satisfy a similar equation as \autoref{eqn:rotationCommutation} and
making use of \autoref{eqn:rotationOperator}, we obtain
\oneline{
\left[ J_x, J_y\right] = i\hbar J_{z} \ .
}
Repeating the above analysis for other axes, we obtain the commutation relations of angular momentum:
\oneline{
\left[ J_i, J_j\right] = i\hbar \varepsilon_{ijk} J_{k} \ , \label{eqn:angularMomentumCommutation}
}
where $i$, $j$, $k$ can be any one of $x$, $y$ and $z$. 

Many important properties follow from the angular-momentum commutation relation represented by \autoref{eqn:angularMomentumCommutation}. For example, by making use of \autoref{eqn:angularMomentumCommutation}, we can show that the operator $\mathbf{J}^2$ defined by
\oneline{
\mathbf{J}^2 \equiv J_x^2 + J_y^2 + J_z^2
}
commutes with any one of $J_x$, $J_y$ and $J_z$, namely,
\oneline{
\left[ \mathbf{J}^2, J_k \right] = 0 , \;\;\; (k = x, y, z) \ .
}
Since $\mathbf{J}^2$ and $J_z$ commutes, the eigenstates of $\mathbf{J}^2$ can be chosen to be the eigenstates of
$J_z$ simultaneously. We denote these state by $\ket{j, m}$ such that
\multiline{
\mathbf{J}^2\ket{j, m} &=& j (j+1)\ket{j, m} \nonumber \\
J_z \ket{j, m} &=& m \ket{j, m} \ .
}
To determine the allowed values for $j$ and $m$, it is convenient to work with the non-Hermitian operators
\oneline{
J_{\pm} = J_x \pm i J_y \ ,
}
where $J_{+}$ is called the raising operator and $J_{-}$ is called the lowering operator. $J_{\pm}$ satisfy the following
commutation relations:
\multiline{
\left[ J_{+}, J_{-}\right] &=& 2 \hbar J_z \nonumber \\
\left[ J_{z}, J_{\pm}\right] &=& \pm \hbar J_{\pm} \nonumber \\
\left[ \mathbf{J}^2, J_{\pm}\right] &=& 0 \ ,
}
all of which can easily be obtained from \autoref{eqn:angularMomentumCommutation}. 

As a special case of the angular momentum operator, the orbital angular momentum operator 
$\mathbf{L} = \mathbf{r} \times \mathbf{p}$ satisfies the same commutation relations
\oneline{
\left[ L_i, L_j\right] = i\hbar \varepsilon_{ijk} L_{k} \ .
}
Using the fact that momentum is the generator of translation, we have
\multiline{
&&\left[ 1 - i \left(\frac{d\phi}{\hbar}\right) L_z\right] \ket{x, y, z} = \left[ 1 - i \left(\frac{d\phi}{\hbar}\right) \left( x p_y - y p_x \right)\right] \ket{x, y, z} \nonumber \\
&& = \left[ 1 - i \left(\frac{p_y}{\hbar}\right) x d\phi + i \left(\frac{p_x}{\hbar}\right) y d\phi \right] \ket{x, y, z} \nonumber \\
&& = \ket{x - y d\phi, y + x d\phi, z} \ . \label{eqn:rotationZ}
}
The above equation shows exactly the effect of an infinitesimal rotation about $z$ axis as one would expect. For an 
arbitrary state $\ket{\psi}$, an infinitesimal rotation about $z$ axis change the wavefunction $\left\langle x, y, z \right.\ket{\psi}$ to 
\oneline{
\bra{x, y, z} \left[ 1 - i \left(\frac{d\phi}{\hbar}\right) L_z\right] \ket{\psi} = \left\langle x + y d\phi, y - x d\phi, z\right. \ket{\psi} \ .
}
By changing to the spherical coordinates $(r, \theta, \phi)$, the above equation becomes
\multiline{
&&\bra{r, \theta, \phi} \left[ 1 - i \left(\frac{d\phi}{\hbar}\right) L_z\right] \ket{\psi} = \bra{r, \theta, \phi-d\phi} \psi \rangle \nonumber \\
&& =\bra{r, \theta, \phi} \psi \rangle -d\phi \frac{\partial}{\partial \phi} \bra{r, \theta, \phi} \psi \rangle \ .
}
Therefore, we obtain
\oneline{
L_{z} = -i\hbar \frac{\partial}{\partial \phi} \ .
}
The expressions for other orbital angular momentum operators 
\multiline{
L_x &=&  i\hbar \left( \sin\phi \frac{\partial}{\partial \theta} + \cot\theta \cos\phi \frac{\partial}{\partial \phi}\right) \nonumber \\
L_y &=& - i\hbar \left( \cos\phi \frac{\partial}{\partial \theta} - \cot\theta \sin\phi \frac{\partial}{\partial \phi}\right) \ ,
}
can also  be derived in the same way. 

The eigenfunctions of $\mathbf{L}^2$ and $L_z$ are known as the spherical harmonics, which are given by
\oneline{
Y_{l m}(\theta, \phi) = \Theta_{lm}(\theta) \Phi_{m}(\phi)
}
where
\oneline{
\Phi_{m}(\phi) = \sqrt{\frac{1}{2\pi}}\exp(i m \phi) \ ,
}
and 
\oneline{
\Theta_{lm}(\theta) = (-1)^{m} \left[ \frac{2 l + 1}{2} \frac{(l-m)!}{(l+m)!} \right]^{1/2} P_l^{m}(\cos\theta)
}
for $m\ge 0$ and $\Theta_{lm}(\theta) =(-1)^{m} \Theta_{l-m}(\theta) $ for $m<0$. Sometimes, it is more convenient
to work with modified spherical harmonics $C_{lm}$ defined by
\oneline{
C_{lm} = \sqrt{\frac{4\pi}{2 l + 1}} Y_{lm}(\theta, \phi) \ .
}

\subsection{Rotation matirces}

\subsection{Coupling of angular momenta}

\subsection{Irreducible spherical tensor operators}



\section{Diatomic molecule in external field}
\label{sec:moleculeInField}

\subsection{DC field}
\label{sec:dcField}

\subsection{AC field}
\label{sec:acField}



\section{Dipole-dipole interaction between two molecules in optical lattices}
\label{sec::ddInteraction}

The energy difference between the excited sate and the ground state, $\Delta_{eg}$ can be easily calculated by diagonalizing  the total Hamiltonian including the external potential. We know how to calculate the potential due to dipole-dipole interaction 
\begin{equation}
\hat{V}_{\mbox{\tiny dd}} = \left(\frac{1}{R^3}\right)\left[ \mathbf{d}_{A}\cdot\mathbf{d}_{B}-3(\mathbf{d}_{A}\cdot\hat{R})\cdot(\mathbf{d}_{B}\cdot\hat{R})\right] \ , \nonumber
\end{equation}
($\vec{R}$ is the vector connecting the centers of mass of two molecules, and $\hat{R}$ is an unit vector in the direction of $\vec{R}$), 


In this appendix, we will calculate the dipole-dipole interaction between two molecules by using the theory of angular momentum. The treatment of the theory of angular momentum is simplified so that only the necessary part that are directly used in the derivation of dipole-dipole interaction is covered. The structure of this appendix is  unconventional. We starts with a problem and then develop the techniques to solve it. In a sense, the appendix is a series of problem-solving blocks. However, in order not to interrupt the logic of problem solving, we will not separate them into different subsections. 

We have two molecules A and B, which are in states $|N_{A}M_{A}\rangle$ and $|N_{B}M_{B}\rangle$ respectively. In free space, the molecules can rotate along the axis connecting their center of mass, then their state are expressed by
\begin{equation}
|N_{A}M_{A}\rangle |N_{B}M_{B}\rangle |lm\rangle  \ , \nonumber
\end{equation}
where $l$ is the total angular momentum of A and B together. 

However, in solid state or optical lattices, the position of molecules are fixed to a good approximation, they cannot rotate along that axis. In this case, their states are given by $|N_{A}M_{A}\rangle |N_{B}M_{B}\rangle$ if there is no interaction between them. If we consider the interaction between the two molecules, we can always use $|N_{A}M_{A}\rangle |N_{B}M_{B}\rangle$ as basis set. By expanding the Hamiltonian in this basis set and diagonalizing it, we can get the new eigenstates.

As the first step, we have to know what the interaction between the two molecules is. If A has a permanent dipole of $\mathbf{d}_{A}$ and B has a permanent dipole of $\mathbf{d}_{B}$ and the vector connecting their centers of mass is $\mathbf{R}$, then the dipole-dipole interaction between A and B is given by
\begin{equation}
\hat{V}_{\mbox{\tiny dd}} = \left(\frac{1}{R^3}\right)\left[ \mathbf{d}_{A}\cdot\mathbf{d}_{B}-3(\mathbf{d}_{A}\cdot\hat{R})(\mathbf{d}_{B}\cdot\hat{R})\right]  \ . \label{dd-operator}
\end{equation}
Here we only consider the dipole moment, there also exists quadrupole, octopole, ..., etc., but their magnitudes are so small that we can ignore them at large distances, say, 400 nm. Our final goal is to evaluate the matrix element $\langle N_{A}M_{A}|\langle N_{B}M_{B} | \hat{V}_{\mbox{\tiny dd}}|N_{A}^{'}M_{A}^{'}\rangle |N_{B}^{'}M_{B}^{'}      \rangle$. The general scheme to calculate the matrix elements is based on the fact that the dipole-dipole operator (Eq. (\ref{dd-operator}) ) can be expressed in terms of harmonics operators. Then its matrix elements can be written in terms of 3-j symbols, which can be calculated numerically. 

We first consider the simpler term in Eq. (\ref{dd-operator}), $\mathbf{d}_{A}\cdot\mathbf{d}_{B}$, which can be expressed in terms of a tensor product:
\begin{equation}
\mathbf{d}_{A}\cdot\mathbf{d}_{B} = - \sqrt{3} \left[ \mathbf{d}_{A}^{(1)} \otimes \mathbf{d}_{B}^{(1)} \right]_{0}^{(0)} \ . \label{dot-product}
\end{equation}
But why should we do that? The answer will become clear later when you see the tensor product is related to spherical harmonics. Now, the task is to prove Eq. (\ref{dot-product}). The proof is enclosed between two straight lines. 

-----------------------------------------------------------------------------------------------

Based on the property of tensor product, we have
\begin{equation}
\left[ A^{(1)} \otimes B^{(1)} \right]_{q}^{(k)} =\sum_{m} \langle 1m, 1\;q-m | kq \rangle A(1,m)B(1, q-m) \ , \label{tensor-product}
\end{equation}
where the superscript $(k)$ is the rank of the tensor. A tensor $T$ of rank one is a vector operator. It consists of three operators 
\begin{equation}
T(1,1) = - \frac{1}{\sqrt{2}} (x + i y) \ , \label{t11}
\end{equation}
\begin{equation}
T(1,0) = z \ , \label{t10}
\end{equation}
and 
\begin{equation}
T(1,-1) =  \frac{1}{\sqrt{2}} (x - i y) \ .  \label{t1-1}
\end{equation}

The intuitive understanding of Eq. (\ref{tensor-product}) is that two angular momentums $j_{1} =1$ and $j_{2}=1$ couple to give rise to another angular momentum $j_{3}=k$ with the projection along some direction to be $q$. 

In order to know the relation between $\mathbf{d}_{A}\cdot\mathbf{d}_{B}$ and $\left[ \mathbf{d}_{A}^{(1)} \otimes \mathbf{d}_{B}^{(1)} \right]_{0}^{(0)}$, we need to calculate $\langle 1m, 1\;q-m | kq \rangle$ in Eq. (\ref{tensor-product}). They are the Clebsch-Gordan coefficients, which appear when you express the couple representation $|j_{1} j_{2} j m\rangle$ in terms of the uncouple representation $|j_{1}m_{1}\rangle |j_{2}m_{2}\rangle$. Due to symmetry consideration, it is often convenient to expressed the coefficients by the 3-j symbols. The relation between Clebsch-Gordan coefficient and 3-j symbol is given by the following two equations:
\begin{equation}
\left( 
\begin{array}{ ccc }
j_{1} & j_{2} & j_{3} \\
m_{1} & m_{2} & m_{3}
\end{array}
\right) \equiv (-1)^{j_{1} - j_{2} - m_{3}} (2j_{3} +1)^{-\frac{1}{2}} \langle j_{1}m_{1}, \; j_{2} m_{2} | j_{3}\; -m_{3}\rangle \ ,
\end{equation}
\begin{equation}
 \langle j_{1}m_{1}, \; j_{2} m_{2} | j_{3}\; -m_{3}\rangle \equiv (-1)^{j_{1} - j_{2} + m_{3}} (2j_{3} +1)^{\frac{1}{2}}
\left( 
\begin{array}{ ccc }
j_{1} & j_{2} & j_{3} \\
m_{1} & m_{2} & -m_{3}
\end{array}
\right)  \ . \label{CG-3j}
\end{equation}
The 3-j symbols are more symmetric than Clebsch-Gordan coefficients. They have the following important property: 
\begin{equation}
\left\{
\begin{array}{ccc}
j_{1}&j_{2}&j_{3} \\
m_{1}&m_{2}&m_{3}
\end{array}
\right\}
=0 \;\;\;\; \mbox{unless } m_{1} + m_{2} + m_{3} = 0
\end{equation}
which physically means that only certain angular momentum eigenstates $\ket{j,m}$ are coupled.
An even permutation of the columns of 3-j symbol does not change its value and an odd permutation multiplier the initial value by $(-1)^{j_{1} + j_{2} + j_{3}}$. Moreover,
\begin{equation}
\left\{
\begin{array}{ccc}
j_{1}&j_{2}&j_{3} \\
m_{1}&m_{2}&m_{3}
\end{array}
\right\}
=
(-1)^{j_{1} + j_{2} + j_{3}}
\left\{
\begin{array}{ccc}
j_{1}&j_{2}&j_{3} \\
-m_{1}&-m_{2}&-m_{3}
\end{array}
\right\}
\end{equation}
This implies
\begin{equation}
\left\{
\begin{array}{ccc}
j_{1}&j_{2}&j_{3} \\
0&0&0
\end{array}
\right\}
=0 \;\; \;\;\mbox{unless } j_{1} + j_{2} + j_{3} = \mbox{even.}
\end{equation}

First, we consider the case where $k=1$ in Eq. (\ref{tensor-product}), that is $\left[ \mathbf{d}_{A}^{(1)} \otimes \mathbf{d}_{B}^{(1)} \right]_{0}^{(0)}$. Based on Eq. (\ref{tensor-product}), we obtain
\begin{eqnarray}
\left[ \mathbf{d}_{A}^{(1)} \otimes \mathbf{d}_{B}^{(1)} \right]_{0}^{(0)} & = & \sum_{m=-1,0,1} \langle 1 m, 1 -m | 00\rangle \mathbf{d}_{A}(1, m) \mathbf{d}_{B}(1, -m) \nonumber \\
&=& \langle 1 1, 1 -1 | 00\rangle \mathbf{d}_{A}(1, 1) \mathbf{d}_{B}(1, -1) + \langle 1 0, 1 0 | 00\rangle \mathbf{d}_{A}(1, 0) \mathbf{d}_{B}(1, 0) \nonumber \\
&  & + \langle 1 -1, 1 1 | 00\rangle \mathbf{d}_{A}(1, -1) \mathbf{d}_{B}(1, 1) \nonumber \\
&=&\frac{1}{\sqrt{3}} \mathbf{d}_{A}(1, 1) \mathbf{d}_{B}(1, -1) - \frac{1}{\sqrt{3}}\mathbf{d}_{A}(1, 0) \mathbf{d}_{B}(1, 0) +  \frac{1}{\sqrt{3}} \mathbf{d}_{A}(1, -1) \mathbf{d}_{B}(1, 1) \nonumber \\
\end{eqnarray}
(Mathematica can be used to evaluate the Clebsch-Gordan coefficients.)
Substituting Eq. (\ref{t11}), (\ref{t10}) and (\ref{t1-1}) into the above equation, we obtain
\begin{equation}
\left[ \mathbf{d}_{A}^{(1)} \otimes \mathbf{d}_{B}^{(1)} \right]_{0}^{(0)}  = -\frac{1}{\sqrt{3}} (\mathbf{d}_{A,x}\mathbf{d}_{B,x} + \mathbf{d}_{A,y}\mathbf{d}_{B,y} + \mathbf{d}_{A,z}\mathbf{d}_{B,z}) = -\frac{1}{\sqrt{3}} \mathbf{d}_{A}\cdot\mathbf{d}_{B} \ .
\end{equation}
Therefore, the tensor product of rank $k=0$ is related to the scalar product. Similarly, it can be shown that the tensor product of rank $k=1$ is related to the cross product. 
\begin{equation}
\left[ A^{(1)} \otimes B^{(1)} \right]_{q}^{(k)} = (\mathbf{A}\times \mathbf{B})\cdot \mathbf{\hat{e}}_{q} \ ,
\end{equation}
where
\begin{eqnarray}
\mathbf{\hat{e}}_{+1} &=& -\frac{1}{\sqrt{2}}(\mathbf{\hat{X}} + i \mathbf{\hat{Y}}) \nonumber \\
\mathbf{\hat{e}}_{0} &=& \mathbf{\hat{Z}} \nonumber \\
\mathbf{\hat{e}}_{-1} &=& \frac{1}{\sqrt{2}}(\mathbf{\hat{X}} - i \mathbf{\hat{Y}})
\end{eqnarray}


-----------------------------------------------------------------------------------------------

As for the second term in the square bracket in Eq. (\ref{dd-operator}), we can do the same thing. Instead of considering the product of two scalars $(\mathbf{d}_{A}\cdot\hat{R})(\mathbf{d}_{B}\cdot\hat{R})$, we consider it as a special case of the scalar product of two vectors (thinking of a vector with two components being zero ). Then according to Eq. (\ref{dot-product}), we have
\begin{equation}
(\mathbf{d}_{A}\cdot\hat{R})(\mathbf{d}_{B}\cdot\hat{R})= 3\left[ \left[ \mathbf{d}_{A}^{(1)} \otimes \mathbf{\hat{R}}^{(1)} \right]^{(0)}\otimes \left[ \mathbf{d}_{B}^{(1)} \otimes \mathbf{\hat{R}}^{(1)} \right]^{(0)}  \right]_{0}^{(0)}  \ .
\end{equation}
There is a simple understanding of the above equation: an angular momentum $\mathbf{d}_{A}$ with $j_{1}=1$ couples  another angular momentum $\hat{R}$ with $j_{2}=1$ to give rise to a new angular momentum, and an angular momentum $\mathbf{d}_{B}$ with $j_{3}=1$ couples  another angular momentum $\hat{R}$ with $j_{4}=1$ to give rise to another new angular momentum, and then these two new angular momentum couples with each other. It is necessary to regroup these operators such that dipole operators are separated from the position operators. The regrouping of operators corresponds to different coupling schemes (orders) of the four angular momentums. In the following, we will talk about the coupling of four angular momentums.

-----------------------------------------------------------------------------------------------

The coupling of four angular momenta has more than one possible coupling scheme, and all these coupling schemes are related by unitary transformations. We might couple $j_{1}$, $j_{2}$, $j_{3}$ and $j_{4}$ in such a way that $j_{1} + j_{4}=j_{14}$, $j_{2} + j_{3}= j_{23}$, and $j_{14} + j_{23} = j$. The eigenfunctions in this coupling scheme is give by $|(j_{1}j_{4})j_{14}(j_{2}j_{3})j_{23}jm \rangle$. The relation between the states corresponding to different coupling schemes is
\begin{eqnarray}
|(j_{1}j_{4})j_{14}(j_{2}j_{3})j_{23}jm \rangle &=& \sum_{j_{12}}\sum_{j_{34}} \langle (j_{1}j_{2})j_{12}(j_{3}j_{4})j_{34}j  | (j_{1}j_{4})j_{14}(j_{2}j_{3})j_{23}j \rangle \nonumber \\
& & \times |(j_{1}j_{4})j_{14}(j_{2}j_{3})j_{23}jm \rangle \ , \label{9-j1}
\end{eqnarray}
and the $9-j$ symbol is defined by
\begin{eqnarray}
&&\langle (j_{1}j_{2})j_{12}(j_{3}j_{4})j_{34}j | |(j_{1}j_{4})j_{14}(j_{2}j_{3})j_{23}j \rangle \nonumber \\
& & = \sqrt{(2j_{12} + 1) (2j_{34} +1)(2j_{14}+1)(2j_{23} + 1)}
\left\{
\begin{array}{ccc}
j_{1}& j_{2}&j_{12} \\
j_{3}&j_{4}&j_{34} \\
j_{14}&j_{23}&j 
\end{array}
\right\} \ . \label{9-j2}
\end{eqnarray}
The first two rows of the $9-j$ symbol is related to the coupling scheme served as the basis set and the last row is associated with the coupling scheme of $|(j_{1}j_{4})j_{14}(j_{2}j_{3})j_{23}jm \rangle$, the eigenfunction that we want to expand. 

-----------------------------------------------------------------------------------------------

Let's expand $\left[ \left[ \mathbf{d}_{A}^{(1)} \otimes \mathbf{\hat{R}}^{(1)} \right]^{(0)}\otimes \left[ \mathbf{d}_{B}^{(1)} \otimes \mathbf{\hat{R}}^{(1)} \right]^{(0)}  \right]_{0}^{(0)}$ in terms of another coupling scheme, namely $\mathbf{d}_{A}$ couples $\mathbf{d}_{B}$ and $\mathbf{\hat{R}}$ couples $\mathbf{\hat{R}}$. Based on Eq. (\ref{9-j1}) and (\ref{9-j2}), we have
\begin{equation}
(\mathbf{d}_{A}\cdot\hat{R})(\mathbf{d}_{B}\cdot\hat{R})= 3\sum_{k}(2k + 1) 
\left\{
\begin{array}{ccc}
1& 1&k \\
1&1&k \\
0&0&0 
\end{array}
\right\} 
\left[ \left[ \mathbf{d}_{A}^{(1)} \otimes \mathbf{d}_{B}^{(1)} \right]^{(k)}\otimes \left[ \mathbf{\hat{R}}^{(1)} \otimes \mathbf{\hat{R}}^{(1)} \right]^{(k)}  \right]_{0}^{(0)} \ . \label{regrouping}
\end{equation}
Obviously, in the above $9-j$ symbol $1+1=k$ and $1+1=k$ is the new coupling scheme, and $0+0=0$ is the original coupling scheme. The $9-j$ symbols can be expressed in terms of $6-j$ symbols (related to the coupling of 3 angular momenta). In the special case where the final angular momentum $j_{9}=0$, we have
\begin{eqnarray}
\left\{
\begin{array}{ccc}
j_{1}& j_{2}&j_{3} \\
j_{4}&j_{5}&j_{6} \\
j_{7}&j_{8}&j_{9}
\end{array}
\right\} 
&=&(-1)^{j_{2} + j_{3} + j_{4} + j_{7}}[(2j_{3} +1)(2j_{7} + 1)]^{-\frac{1}{2}} \nonumber \\
&  & \times 
\left\{
\begin{array}{ccc}
j_{1}& j_{2}&j_{3} \\
j_{5}&j_{4}&j_{7} 
\end{array}
\right\} 
\delta_{j_{3}j_{6}}\delta_{j_{7}j_{8}} \ . 
\end{eqnarray}
Then
\begin{eqnarray}
\left\{
\begin{array}{ccc}
1& 1&k \\
1&1&k \\
0&0&0 
\end{array}
\right\} 
&=& (-1)^{k+2} (2k+1)^{-\frac{1}{2}}
\left\{
\begin{array}{ccc}
1& 1&k \\
1&1&0 
\end{array}
\right\} \nonumber \\
&=& (-1)^{k+2} (2k+1)^{-\frac{1}{2}} 
\left\{ 
\begin{array}{cc}
\frac{1}{3} (-1)^{-k} & 0\leq k \leq 2 \\
0 & \mbox{otherwise}
\end{array}
\right.
\end{eqnarray}
Because $k$ results from the coupling of $j_{1}=1$ and $j_{2}=1$, it is in the range of $|j_{1}-j_{2}|, \cdots, |j_{1}+j_{2}|$, and the above equation can be simplified as
\begin{equation}
\left\{
\begin{array}{ccc}
1& 1&k \\
1&1&k \\
0&0&0 
\end{array}
\right\} = \frac{(2k + 1)^{-\frac{1}{2}}}{3} \ . \label{9-j-value}
\end{equation}  
Substituting Eq. (\ref{9-j-value}) into Eq. (\ref{regrouping}) and using Eq. (\ref{tensor-product}), we obtain
\begin{eqnarray}
(\mathbf{d}_{A}\cdot\hat{R})(\mathbf{d}_{B}\cdot\hat{R})&=&\sum_{k=0,1,2}\sum_{q}(2k+1)^{\frac{1}{2}}\langle k q, k -q | 00 \rangle\left[ \mathbf{d}_{A}^{(1)} \otimes \mathbf{d}_{B}^{(1)} \right]^{(k)}_{q} \left[ \mathbf{\hat{R}}^{(1)} \otimes \mathbf{\hat{R}}^{(1)} \right]^{(k)}_{-q}    \nonumber \\
&=& \sum_{k=0,1,2}\sum_{q} (-1)^{k-q} \left[ \mathbf{d}_{A}^{(1)} \otimes \mathbf{d}_{B}^{(1)} \right]^{(k)}_{q} \left[ \mathbf{\hat{R}}^{(1)} \otimes \mathbf{\hat{R}}^{(1)} \right]^{(k)}_{-q} \ . \label{scalarXscalar}
\end{eqnarray}
Since the tensor product of rank 1 is related to the cross product, $\left[ \mathbf{\hat{R}}^{(1)} \otimes \mathbf{\hat{R}}^{(1)} \right]^{(k=1)}_{-q}$ is associated with $\mathbf{\hat{R}}\times\mathbf{\hat{R}}$ and it is zero. Then Eq. (\ref{scalarXscalar}) can be simplified further as
\begin{eqnarray}
(\mathbf{d}_{A}\cdot\hat{R})(\mathbf{d}_{B}\cdot\hat{R})&=&\left[ \mathbf{d}_{A}^{(1)} \otimes \mathbf{d}_{B}^{(1)} \right]^{(0)}_{0} \left[ \mathbf{\hat{R}}^{(1)} \otimes \mathbf{\hat{R}}^{(1)} \right]^{(0)}_{0} \nonumber \\
& & + \sum_{|q|=0,1,2} (-1)^{q} \left[ \mathbf{d}_{A}^{(1)} \otimes \mathbf{d}_{B}^{(1)} \right]^{(2)}_{q} \left[ \mathbf{\hat{R}}^{(1)} \otimes \mathbf{\hat{R}}^{(1)} \right]^{(2)}_{-q} \nonumber \\
&=& (-\frac{1}{\sqrt{3}} \mathbf{d}_{A}\cdot \mathbf{d}_{B})(-\frac{1}{\sqrt{3}}\mathbf{\hat{R}}\cdot\mathbf{\hat{R}} ) \nonumber \\
& & +  \sum_{q} (-1)^{q} \left[ \mathbf{d}_{A}^{(1)} \otimes \mathbf{d}_{B}^{(1)} \right]^{(2)}_{q} \left[ \mathbf{\hat{R}}^{(1)} \otimes \mathbf{\hat{R}}^{(1)} \right]^{(2)}_{-q} \nonumber \\
&=& \frac{\mathbf{d}_{A}\cdot \mathbf{d}_{B}}{3}+  \sum_{q} (-1)^{q} \left[ \mathbf{d}_{A}^{(1)} \otimes \mathbf{d}_{B}^{(1)} \right]^{(2)}_{q} \left[ \mathbf{\hat{R}}^{(1)} \otimes \mathbf{\hat{R}}^{(1)} \right]^{(2)}_{-q} \nonumber \\ 
\label{term2}
\end{eqnarray}
Substitution of Eq. (\ref{term2}) into the expression for the dipole-dipole interaction (Eq. (\ref{dd-operator})) yields
\begin{equation}
\hat{V}_{\mbox{\tiny dd}}(\mathbf{R}) = \frac{-3}{R^{3}} \sum_{q} (-1)^{q} \left[ \mathbf{d}_{A}^{(1)} \otimes \mathbf{d}_{B}^{(1)} \right]^{(2)}_{q} \left[ \mathbf{\hat{R}}^{(1)} \otimes \mathbf{\hat{R}}^{(1)} \right]^{(2)}_{-q} \ .
\end{equation}

-----------------------------------------------------------------------------------------------

Before we go on, it is helpful to get familiar with rotational matrices and spherical harmonics.


-----------------------------------------------------------------------------------------------



\section{Introduction to exciton}
\label{sec: exciton}
We describe the state of a crystal system by specifying the occupation state of each molecules that constitute the crystal. For example, we denote the crystal wavefunction as
\begin{equation}
\left. |N_{1f_{1}} N_{2f_{2}} ...N_{nf} ... \right\rangle \nonumber \ ,
\end{equation}
where $N_{nf}$ represents the occupation number of state $f$ in molecule $n$ and it is either 0 or 1. Accordingly, the occupation number operator is defined as
\begin{equation}
\hat{N}_{nf} |...N_{nf}...\rangle = N_{nf} |...N_{nf}...\rangle \ .
\end{equation}
For one molecule $n$, it must be in some state $g$, therefore
\begin{equation}
\sum_{g}\hat{N}_{ng} = 1 \ . \label{must-be-in-a-state}
\end{equation}
It is convenient to introduce two operators $b_{nf}^{\dagger}$ and $b_{nf}$ associated with molecule $n$ and state $f$,
\begin{eqnarray}
b_{nf}^{\dagger} |...N_{nf}...\rangle &=& (1-N_{nf})|...(N_{nf}+1)...\rangle \ , \nonumber \\
b_{nf} |...N_{nf}...\rangle &=& N_{nf}|...(N_{nf}-1)...\rangle \ , \label{single-molecule-operator}
\end{eqnarray}
and express the occupation number operator as the product of these two operators,
\begin{equation}
\hat{N}_{nf} =b_{nf}^{\dagger} b_{nf} \ .
\end{equation}
The physical meaning of the two operators is clear: $b_{nf}$ creates a state $f$ in molecule $n$ and $b_{nf}$ destroy a state $f$ in molecule $n$.  It follows from Eq. (\ref{single-molecule-operator}) that
\begin{eqnarray}
b_{nf}b_{nf}^{\dagger} &+& b_{nf}^{\dagger}b_{nf} = 1 \ , \nonumber \\
b_{nf}b_{nf}&=&b_{nf}^{\dagger}b_{nf}^{\dagger} =0 \ . \label{commutation-relation-single-molecule}
\end{eqnarray}
Because an operator for a specific molecule $n$ and state $f$ does not operate on other molecules and other states, any two operators that correspond to different molecules $n$ and $m$ or different states $f$ and $f'$ commute all the time. 

Based on the physical meaning of $b_{nf}^{\dagger}$ and $b_{nf}$, it can be easily seen that the exciton creation and annihilation operators can be written as
\begin{equation}
P_{nf}^{\dagger} = b_{nf}^{\dagger} b_{n0} \ , \;\;P_{nf}=b_{n0}^{\dagger} b_{nf} \ . \label{creation&annihilation}
\end{equation}
Substituting Eq. (\ref{commutation-relation-single-molecule}), we have
\begin{eqnarray}
P_{nf}^{\dagger} P_{nf} &=& b_{nf}^{\dagger} b_{n0}b_{n0}^{\dagger} b_{nf} \nonumber \\
                                      &=& b_{nf}^{\dagger}(1-b_{n0}^{\dagger}b_{n0})b_{nf} \nonumber \\
                                      &=& \hat{N}_{nf} \ , \label{creation-annihilation}
\end{eqnarray}
\begin{eqnarray}
P_{nf} P_{nf}^{\dagger} &=& b_{n0}^{\dagger} b_{nf} b_{nf}^{\dagger} b_{n0} \nonumber \\
                                      &=& b_{n0}^{\dagger}(1-b_{nf}^{\dagger}b_{nf})b_{n0} \nonumber \\
                                      &=& \hat{N}_{n0} \ . \label{annihilation-creation}
\end{eqnarray}
Subtracting Eq. (\ref{creation-annihilation}) from Eq. (\ref{annihilation-creation}) and making use of Eq. (\ref{must-be-in-a-state}), we obtain
\begin{equation}
P_{nf} P_{nf}^{\dagger} - P_{nf}^{\dagger} P_{nf} = 1-\hat{N}_{nf} - \sum_{g\neq 0} \hat{N}_{ng}  \ . \label{exact-relation}
\end{equation}
Because $P_{nf}$ and $P_{nf}^{\dagger}$ don't operate on a different
molecule $m$, any exciton operators corresponding to different molecules
commute with each other. Then the exact statistics is given by 
\begin{equation}
P_{nf}P_{n'f}^{\dagger}-P_{nf}^{\dagger}P_{n'f}  =  \delta_{nn'}\left(1-\hat{N}_{nf}-\sum_{g\neq0}\hat{N}_{ng}\right)  \label{exact-relation2}
\end{equation}
and 
\begin{equation}
P_{nf}P_{nf'}^{\dagger}-P_{nf}^{\dagger}P_{nf'}=\delta_{ff'}\ . 
\end{equation}
Unfortunately, it is cumbersome to take into account of the exact
statistics of excitons in most cases. So it is often desirable to approximate excitons as Boses when only a small number of molecules in the crystal are excited. In other words, if the following inequality
\begin{equation}
\left\langle N_{nf} \right\rangle \ll 1 \label{condition-for-Bose-approx}
\end{equation}
holds, the exciton creation and annihilation operators satisfy the commutation rules for Bose operators
\begin{equation}
P_{n'f'}P_{nf}^{\dagger} - P_{nf}^{\dagger}P_{n'f'}=\delta_{nn'}\delta_{ff'} \ . \label{approx-relation}
\end{equation}
In the following derivation, we are assuming the exciton operators are Bose operators and satisfy Eq. (\ref{approx-relation}). 

The way to use operator $P_{nf}$ and $P_{nf}^{\dagger}$ to describe excitons is called site representation. Due to the periodicity of crystal, there is another way to represent excitons by making use of the Fourier transforms of $P_{nf}$ and $P_{nf}^{\dagger}$: 
\begin{eqnarray}
P_{nf} &=& \frac{1}{\sqrt{N}} \sum_{k} P_{f}(k) e^{ikn} \nonumber \\
P_{nf}^{\dagger} &=& \frac{1}{\sqrt{N}} \sum_{k} P_{f}^{\dagger}(k) e^{-ikn} \ , \label{unitary-transform}
\end{eqnarray}
and this is called wavevector representation. Since the unitary transformation does not change the commutation rule of operators, we expect that the following equation
\begin{equation}
P_{f',k'}P_{f,k}^{\dagger} - P_{f,k}^{\dagger}P_{f',k'}=\delta_{kk'}\delta_{ff'} \ . \label{approx-relation-in-wavevector}
\end{equation}
also hold. This can be verified by substituting Eq. (\ref{unitary-transform}) into Eq. (\ref{approx-relation}). 






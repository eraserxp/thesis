%% The following is a directive for TeXShop to indicate the main file
%%!TEX root = diss.tex

\chapter{Introduction}
\label{ch:Introduction}

This thesis is a theoretical study on how to control quantum dynamics of quasiparticles in a periodic lattice potential. 
Although some work presented in the thesis is quite general and can be applied to any system with a periodic potential, 
the main subjects of the research are systems of ultracold atoms and molecules as their experimental manipulation is 
relatively easy. The control schemes proposed in the thesis also relate closely with the recent developments of laser 
cooling and trapping of atoms and molecules. Therefore, to give readers a broad context in which to understand the 
thesis, this chapter presents an introductory overview of the field of ultracold atoms and molecules. In \autoref{sec:ultracold}, we briefly introduce ultracold atoms and molecules and explain why they are interesting. After
that, in \autoref{sec:control}, we focus on the controllability property of ultracold systems and review recent works 
based on controlling the internal and external degrees of freedoms of  molecules. 
Finally, we give an outline of the thesis in \autoref{sec:outline} and discuss the connection and contribution of  
this thesis to the field of ultracold atoms and molecules. 

\section{Ultracold atoms and molecules}
\label{sec:ultracold}

%what is ultracold atoms and molecules?
%Controlling materials has always been the main theme of scientific quest for generations of people. For thousands of
%years, our control over materials focused on macroscopic scale. With the advance of technology, nowadays we are 
%able to put microscopic objects into scientific investigations. Especially, the control of atoms and molecules, the basic 
%building blocks of materials, is attracting the attention of scientists worldwide. 
In the past three decades, there has been great success in creating ultracold atomic gases, which  revolutionized the 
field of atomic and optical physics \cite{southwell2002, chu2002} and allowed for many exciting applications,  
including Bose$-$Einstein condensation of atomic gases\cite{anglin2002}, nonlinear and quantum atom 
optics\cite{rolston2002}, collisions at ultracold temperatures\cite{burnett2002}, all-optical atomic
clock\cite{udem2002},
and quantum information processing with atoms and photons\cite{monroe2002}. Prompted by this success with 
ultracold atoms, researchers are now putting great effort to extend the method of ultracold physics to molecules,
creating the emerging field of ultracold molecules. 
As basic building blocks of materials, molecules,  just like atoms,  in ultracold temperatures are expeted to have  an great
impact on different areas of science and technology such as precision measurements of fundamental constants,  quantum simulations of condensed matter systems, quantum information processing, precise control of molecular
dynamics and ultracold chemistry\cite{our-njp-review, friedrich2009, schnell2009, Bell2009, krems2010cold, Ni2009,
Jin2011, Jin2012, quemener2012, Baranov2012}. 


An important feature of ultracold atoms and molecules is their small kinetic energies. The word  ``ultracold'' doesn't
relate to temperatures directly, its real meaning is ``ultra-slow'', refering to the translational motion of particles. 

why are they unqiue?

%\section{Ultracold systems on Optical lattices}
%\label{sec:opticalLattices}
%
%what's optical lattices?



\section{Quantum control of ultracold systems}
\label{sec:control}

\section{Thesis outline}
\label{sec:outline}


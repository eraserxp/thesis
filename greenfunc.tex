%chapter on Green's function
\chapter{Lattice Green's function of two-particle state}
\label{ch:greenfunc}

\section{Introduction}
\label{sec:introGreenFunc}

Green's function is an important theoretical tool in the study of condensed matter systems. Its applications range from

The numerical evaluation of the Green's function is a cumbersome task.

In this chapter, we extend the numerical method \cite{Berciu2010, Berciu2011, Berciu2012} developed by Berciu to a system with arbitary disorders. Then we employ the Green's function to study the tunneling of biexciton states through
a disorder. 

\section{Equation of motion for Green's function}
\label{sec:equationOfMotion}

To derive the equation of motion for Green's function, we start from the Hamiltonian for the system. We are 
considering a periodic lattice with the Hamiltonian:
\multiline{
\hat{H} = \sum_{l} e_{l} \pd{l}\p{l} + \sum_{l}\sum_{r\neq 0} t_{l, l+r} \pd{l}\p{l+r} + \sum_{l}\sum_{r\neq 0} d_{l, l+r} \pd{l}\pd{l+r}\p{l}\p{l+r} \ ,
}
where $e_l$ is the energy of the particle at site $l$, $\pd{l}$ create a particle at site $l$, $t_{l, l+r}$ represents the hopping interaction between
site $l$ and site $l+r$, and $d_{l, l+r}$ represents the dynamic interaction between
site $l$ and site $l+r$. Note that the Hamiltonian $\hat{H}$ is very general as there is no restriction on the values 
of $e_l$, $t_{l, l+r}$, and $d_{l, l+r}$, and thus it can describe both an ordered system and an 
disordered system. The Green's operator is defined in terms of the Hamiltonian,
\oneline{
\hat{G}(z) \equiv (z - \hat{H})^{-1} \ ,
}
where $z = E + i\eta$ is the complex energy. From the definition of the Green's operator, it is clear that  
\oneline{
(z - \hat{H}) \hat{G}(z) = 1 \ . \label{eqn:equationForG}
}
The above equation contains all information about the system. 

Depending on the system in consideration, the creation and anhilation opreators will satisfy different commutation
rules. For instance, if the particles are femions, $\pd{n}$ and $\p{m}$ satisfy the anticommutation relations:
\multiline{
&&\{\pd{n}, \p{m}\} = \pd{n}\p{m} + \p{m}\pd{n} = \delta_{n, m} \ , \nonumber \\
&&\{\pd{n}, \pd{m}\} = \{\p{n}, \p{m}\} = 0 \ ,
}
and if particles are bosons, $\pd{n}$ and $\p{m}$ satisfy the commutation relations:
\multiline{
&&[\pd{n}, \p{m}] = \pd{n}\p{m} - \p{m}\pd{n} = \delta_{n, m} \ , \nonumber \\
&&[\pd{n}, \pd{m}] = [\p{n}, \p{m}] = 0 \ .
}
In the current study, we are considering collective excitations of atom or molecules in the lattice. They have the 
characteristics of both fermions and bosons. When two excitations reside at different sites, they behave like
bosons, that is,
\oneline{
\pd{n}\p{m} - \p{m}\pd{n} = 0 \label{eqn:boson-like}
}
if $n \neq m$; when excitations are at the same site, they behave like fermions, that is,
\oneline{
\pd{n}\p{m} + \p{m}\pd{n} = 0 \label{eqn:fermion-like}
}
if $n = m$. For convenience, we can combine \autoref{eqn:boson-like} and \autoref{eqn:fermion-like} into one equation:
\oneline{
\p{m}\pd{n} = \delta_{m, n} + (1- 2 \delta_{m, n}) \pd{n}\p{m} \ .  \label{eqn:excitonCommutation}
}

Using the two-particle basis set $\ket{n}\ket{m}$ (and $n \neq m$), \autoref{eqn:equationForG} can be rewritten as
\oneline{
\bra{n} \bra{m} (z - \hat{H}) \hat{G}(z) \ket{n^{\prime} } \ket{m^{\prime} } = \bra{n} \bra{m} \ket{n^{\prime} } \ket{m^{\prime} } \ . \label{eqn:startEqnForG}
}
To simplify the above equation, we make use of \autoref{eqn:excitonCommutation} and evaluate the effect of the 
Hamiltonian operating on the basis set:
\multiline{
\hat{H} \ket{n} \ket{m} &=& (e_{n} + e_{m} ) \ket{n} \ket{m} + \sum_{r \neq 0}(1-\delta_{n-r, m}) t_{n-r, n} \ket{n-r} \ket{m}  \nonumber \\
&& + \sum_{r \neq 0} (1-\delta_{n, m-r}) t_{m-r, m} \ket{n} \ket{m - r} +  \sum_{r \neq 0} d_{n, m} \delta_{n-m, \pm r} \ket{n} \ket{m}  \ ,  \nonumber \\
}
where the factors like $(1-\delta_{n-r, m})$ appear because only the two-particle states $\ket{n}\ket{m}$ with 
$n \neq m$ are included in the basis set, 
then the left hand side of \autoref{eqn:startEqnForG} becomes
\multiline{
&&\bra{n} \bra{m} (z - \hat{H}) \hat{G}(z) \ket{n^{\prime} } \ket{m^{\prime} } \nonumber \\
&&= (z - e_{n} - e_{m} ) G(n, m, n^{\prime}, m^{\prime}; z) 
- \sum_{r \neq 0} (1-\delta_{n-r, m}) t_{n-r, n} G(n-r, m, n^{\prime}, m^{\prime}; z) \nonumber \\
&& - \sum_{r \neq 0}(1-\delta_{n, m-r}) t_{m-r, m} G(n, m-r, n^{\prime}, m^{\prime}; z) 
 -  \sum_{r \neq 0} d_{n, m} \delta_{n-m, \pm r}  G(n, m, n^{\prime}, m^{\prime}; z) \ , \nonumber \\
}
where $G(n, m, n^{\prime}, m^{\prime}; z)$ represents the matrix element of the Green's operator in the two-particle 
basis set and it is given by
\oneline{
G(n, m, n^{\prime}, m^{\prime}; z) =  \bra{n} \bra{m} \hat{G}(z) \ket{n^{\prime} } \ket{m^{\prime} } \ .
}
Similarly, substituting \autoref{eqn:excitonCommutation} into the right hand side of \autoref{eqn:startEqnForG}  gives
rise to 
\oneline{
 \bra{n} \bra{m} \ket{n^{\prime} } \ket{m^{\prime} }= \delta_{m, n^{\prime}}  \delta_{n, m^{\prime}} +  \delta_{n, n^{\prime}}  \delta_{m, m^{\prime}} \ .
}
Therefore, the equation of the motion for the Green's function is given by
\multiline{
&&\left(z - e_{n} - e_{m} -  \sum_{r \neq 0} d_{n, m} \delta_{n-m, \pm r} \right) G(n, m, n^{\prime}, m^{\prime}; z) \nonumber \\
&& - \sum_{r \neq 0} (1-\delta_{n-r, m}) t_{n-r, n} G(n-r, m, n^{\prime}, m^{\prime}; z) \nonumber \\
&& - \sum_{r \neq 0} (1-\delta_{n, m-r})t_{m-r, m} G(n, m-r, n^{\prime}, m^{\prime}; z) \nonumber \\
&& =  \delta_{m, n^{\prime}}  \delta_{n, m^{\prime}} +  \delta_{n, n^{\prime}}  \delta_{m, m^{\prime}} \ . \label{eqn:eqnOfMotion}
}
Because two excitations cannot reside at the same lattice site, we know the Green's function 
$G(n, m, n^{\prime}, m^{\prime}; z)$ is meaningless whenever $n=m$ or $n^{\prime} = m^{\prime}$. It is clear that
\autoref{eqn:eqnOfMotion} can't guarantee all the Green's functions that it contains are physical even if we restrict
the basis set to be just 

Since all the Green's function in \autoref{eqn:eqnOfMotion} have the same parameters $n^{\prime}$ and
 $m^{\prime}$, we define 
\oneline{
\Tilde{G} (n, m; z) = G(n, m, n^{\prime}, m^{\prime}; z) \ 
}
to simplify the notation. Then \autoref{eqn:eqnOfMotion} can be written as
\multiline{
&&\left(z - e_{n} - e_{m} -  \sum_{r \neq 0} d_{n, m} \delta_{n-m, \pm r} \right) \Tilde{G}(n, m; z) \nonumber \\
&& - \sum_{r \neq 0} (1-\delta_{n-r, m}) t_{n-r, n} \Tilde{G}(n-r, m; z) - \sum_{r \neq 0} (1-\delta_{n, m-r}) t_{m-r, m} \Tilde{G}(n, m-r; z) \nonumber \\
&& =  \delta_{m, n^{\prime}}  \delta_{n, m^{\prime}} +  \delta_{n, n^{\prime}}  \delta_{m, m^{\prime}} \ . \label{eqn:eqnOfMotionSimplified}
}

\section{Numerical calculation of Green's function}
\label{sec:numericGreen}

The equation of motion of Green's function, \autoref{eqn:eqnOfMotionSimplified}, is our starting point to calculate
the Green's function. In the nearest neighbor approximation, the distance between two interacting sites $r$ can only 
be 1 or $-1$, then the quation of motion of Green's function becomes
\multiline{
&&\left(z - e_{n} - e_{m} -   d_{n, m} \delta_{n-m, \pm 1} \right) \Tilde{G}(n, m; z) \nonumber \\
&& - (1- \delta_{n-1, m}) t_{n-1, n} \Tilde{G}(n-1, m; z) - (1- \delta_{n+1, m}) t_{n+1, n} \Tilde{G}(n+1, m; z) \nonumber \\
&& - (1- \delta_{n, m-1}) t_{m-1, m}\Tilde{G}(n, m-1; z) - (1- \delta_{n, m+1}) t_{m+1, m} \Tilde{G}(n, m+1; z) \nonumber \\
&& =  \delta_{m, n^{\prime}}  \delta_{n, m^{\prime}} +  \delta_{n, n^{\prime}}  \delta_{m, m^{\prime}} \ . \label{eqn:eqnOfMotionNNA}
}
As metioned before, the factors like $(1- \delta_{n-1, m})$ in \autoref{eqn:eqnOfMotionNNA} are used to eliminate 
the unphysical Green's functions like $\Tilde{G}(i, i; z)$. For instance, when $n = m - 1$, $\Tilde{G}(n+1, m; z)$ and 
$\Tilde{G}(n, m-1; z)$ will disappear in \autoref{eqn:eqnOfMotionNNA} because of the factors in front of them. 
\Autoref{eqn:eqnOfMotionNNA} shows that $\Tilde{G}(n, m; z)$ only relates directly with at most four other Green's
functions in the nearest neighbor approximation. This relationship can be shown schematically as follows:
\multiline{
& \vdots & \nonumber \\
\Tilde{G}(n-1, m+1; z) &\rightarrow& \bigg\{  \Tilde{G}(n-2, m+1; z),  \Tilde{G}(n-1, m; z), \Tilde{G}(n, m+1; z),\Tilde{G}(n-1, m+2; z) \bigg\} \nonumber \\
\Tilde{G}(n, m; z) &\rightarrow& \bigg\{  \Tilde{G}(n-1, m; z),  \Tilde{G}(n, m-1; z), \Tilde{G}(n+1, m; z),\Tilde{G}(n, m+1; z) \bigg\} \nonumber \\
\Tilde{G}(n+1, m-1; z) &\rightarrow& \bigg\{  \Tilde{G}(n, m-1; z),  \Tilde{G}(n+1, m-2; z), \Tilde{G}(n+2, m-1; z),\Tilde{G}(n+1, m; z) \bigg\} \nonumber \\
& \vdots &  \nonumber \\
}
For the Green's function $\Tilde{G}(i, j; z)$ on the left, the summation of parameters $i + j = n + m$ is fixed. For the 
Green's functions $\Tilde{G}(i^{\prime}, j^{\prime}; z)$ on the right, the summation of parameters
 $i^{\prime} + j^{\prime}$ can only be either $n+m-1$ or $n+m+1$. 
This inspire us to group some Green's functions into a vector $\mathbf{V}_{K}$ according to their parameters $n$ and $m$, that is,
\oneline{
\mathbf{V}_{K} = \colvec{5}{\cdots}{\Tilde{G}(i-1, j+1; z)}{\Tilde{G}(i, j; z)}{\Tilde{G}(i+1, j-1; z)}{\vdots} \ ,
} 
then \autoref{eqn:eqnOfMotionNNA} can be written as
\multiline{
\mathbf{V}_{K} = \boldsymbol\alpha_{K}(z) \mathbf{V}_{K-1} +\boldsymbol\beta_{K}(z)\mathbf{V}_{K+1} 
}
if $n^{\prime} + m^{\prime} \neq K$, and as
\multiline{
\mathbf{V}_{K} =  \boldsymbol\alpha_{K}(z) \mathbf{V}_{K-1} +\boldsymbol\beta_{K}(z)\mathbf{V}_{K+1}  + \mathbf{C}
}
if $n^{\prime} + m^{\prime} = K$. In the above two equations, $\boldsymbol\alpha_{K}(z)$ and
 $\boldsymbol\beta_{K}(z)$
are matrices and $\mathbf{C}$ is a constant vector, and their values will be determined later.  


\section{The tunneling of biexciton state through a disorder}
\label{sec:biexcitonTunneling}

\section{Conclusion}
\label{sec:conclusionGreenFunc}

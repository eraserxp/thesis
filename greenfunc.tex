%chapter on Green's function
\chapter{Lattice Green's function of two-particle state}
\label{ch:greenfunc}

\section{Introduction}
\label{sec:introGreenFunc}

Green's function is an important theoretical tool in the study of condensed matter systems. Its applications range from

The numerical evaluation of the Green's function is a cumbersome task.

In this chapter, we extend the numerical method \cite{Berciu2010, Berciu2011, Berciu2012} developed by Berciu to a system with arbitary disorders. Then we employ the Green's function to study the tunneling of biexciton states through
a disorder. 

\section{Equation of motion for Green's function}
\label{sec:equationOfMotion}

To derive the equation of motion for Green's function, we start from the Hamiltonian for the system. We are 
considering a periodic lattice with the Hamiltonian:
\multiline{
\hat{H} = \sum_{l} e_{l} \pd{l}\p{l} + \sum_{l}\sum_{\delta\neq 0} t_{l, l+\delta} \pd{l}\p{l+\delta} + \sum_{l}\sum_{\delta\neq 0} d_{l, l+\delta} \pd{l}\pd{l+\delta}\p{l}\p{l+\delta} \ ,
}
where $e_l$ is the energy of the particle at site $l$, $\pd{l}$ create a particle at site $l$, $t_{l, l+\delta}$ represents the hopping interaction between
site $l$ and site $l+\delta$, and $d_{l, l+\delta}$ represents the dynamic interaction between
site $l$ and site $l+\delta$. The Green's operator is defined in terms of the Hamiltonian,
\oneline{
\hat{G}(z) \equiv (z - \hat{H})^{-1} \ ,
}
where $z = E + i\eta$ is the complex energy. From the definition of the Green's operator, it is clear that  
\oneline{
(z - \hat{H}) \hat{G}(z) = 1 \ . \label{eqn:equationForG}
}
The above equation contains all information about the system. 

Depending on the system in consideration, the creation and anhilation opreators will satisfy different commutation
rules. For instance, if the particles are femions, $\pd{n}$ and $\p{m}$ satisfy the anticommutation relations:
\multiline{
&&\{\pd{n}, \p{m}\} = \pd{n}\p{m} + \p{m}\pd{n} = \delta_{n, m} \ , \nonumber \\
&&\{\pd{n}, \pd{m}\} = \{\p{n}, \p{m}\} = 0 \ ,
}
and if particles are bosons, $\pd{n}$ and $\p{m}$ satisfy the commutation relations:
\multiline{
&&[\pd{n}, \p{m}] = \pd{n}\p{m} - \p{m}\pd{n} = \delta_{n, m} \ , \nonumber \\
&&[\pd{n}, \pd{m}] = [\p{n}, \p{m}] = 0 \ .
}
In the current study, we are considering collective excitations of atom or molecules in the lattice. They have the 
characteristics of both fermions and bosons. When two excitations reside at different sites, they behave like
bosons, that is,
\oneline{
\pd{n}\p{m} - \p{m}\pd{n} = 0 \label{eqn:boson-like}
}
if $n \neq m$; when excitations are at the same site, they behave like fermions, that is,
\oneline{
\pd{n}\p{m} + \p{m}\pd{n} = 0 \label{eqn:fermion-like}
}
if $n = m$. For convenience, we can combine \autoref{eqn:boson-like} and \autoref{eqn:fermion-like} into one equation:
\oneline{
\p{m}\pd{n} = \delta_{m, n} + (1- 2 \delta_{m, n}) \pd{n}\p{m} \ .  \label{eqn:excitonCommutation}
}

Using the two-particle basis set $\ket{n}\ket{m}$, \autoref{eqn:equationForG} can be rewritten as
\oneline{
\bra{n} \bra{m} (z - \hat{H}) \hat{G}(z) \ket{n^{\prime} } \ket{m^{\prime} } = \bra{n} \bra{m} \ket{n^{\prime} } \ket{m^{\prime} } \ . \label{eqn:startEqnForG}
}
To simplify the above equation, we make use of \autoref{eqn:excitonCommutation} and evaluate the effect of the 
Hamiltonian operating on the basis set:
\multiline{
\hat{H} \ket{n} \ket{m} &=& (e_{n} + e_{m} ) \ket{n} \ket{m} + \sum_{\delta \neq 0} t_{n-\delta, n} \ket{n-\delta} \ket{m} + \sum_{\delta \neq 0} t_{m-\delta, m} \ket{n} \ket{m - \delta} \nonumber \\
&& +  \sum_{\delta \neq 0} d_{n, m} \delta_{n-m, \pm\delta} \ket{n} \ket{m} \ ,
}
then the left hand side of \autoref{eqn:startEqnForG} becomes
\multiline{
&&\bra{n} \bra{m} (z - \hat{H}) \hat{G}(z) \ket{n^{\prime} } \ket{m^{\prime} }= (z - e_{n} - e_{m} ) G(n, m, n^{\prime}, m^{\prime}; z) \nonumber \\
&&- \sum_{\delta \neq 0} t_{n-\delta, n} G(n-\delta, m, n^{\prime}, m^{\prime}; z) - \sum_{\delta \neq 0} t_{m-\delta, m} G(n, m-\delta, n^{\prime}, m^{\prime}; z) \nonumber \\
&& -  \sum_{\delta \neq 0} d_{n, m} \delta_{n-m, \pm\delta}  G(n, m, n^{\prime}, m^{\prime}; z) \ ,
}
where $G(n, m, n^{\prime}, m^{\prime}; z)$ represents the matrix element of the Green's operator in the two-particle 
basis set and it is given by
\oneline{
G(n, m, n^{\prime}, m^{\prime}; z) =  \bra{n} \bra{m} \hat{G}(z) \ket{n^{\prime} } \ket{m^{\prime} } \ .
}
Similarly, substituting \autoref{eqn:excitonCommutation} into the right hand side of \autoref{eqn:startEqnForG}  gives
rise to 
\oneline{
 \bra{n} \bra{m} \ket{n^{\prime} } \ket{m^{\prime} }= \delta_{m, n^{\prime}}  \delta_{n, m^{\prime}} +  \delta_{n, n^{\prime}}  \delta_{m, m^{\prime}} \ .
}
Therefore, the equation of the motion for the Green's function is given by
\multiline{
&&\left(z - e_{n} - e_{m} -  \sum_{\delta \neq 0} d_{n, m} \delta_{n-m, \pm\delta} \right) G(n, m, n^{\prime}, m^{\prime}; z) \nonumber \\
&& - \sum_{\delta \neq 0} t_{n-\delta, n} G(n-\delta, m, n^{\prime}, m^{\prime}; z) - \sum_{\delta \neq 0} t_{m-\delta, m} G(n, m-\delta, n^{\prime}, m^{\prime}; z) \nonumber \\
&& =  \delta_{m, n^{\prime}}  \delta_{n, m^{\prime}} +  \delta_{n, n^{\prime}}  \delta_{m, m^{\prime}} \ . \label{eqn:eqnOfMotion}
}
Since all the Green's function in \autoref{eqn:eqnOfMotion} have the same parameters $n^{\prime}$ and
 $m^{\prime}$, we define 
\oneline{
\Tilde{G} (n, m; z) = G(n, m, n^{\prime}, m^{\prime}; z) \ 
}
to simplify the notation. Then \autoref{eqn:eqnOfMotion} can be written as
\multiline{
&&\left(z - e_{n} - e_{m} -  \sum_{\delta \neq 0} d_{n, m} \delta_{n-m, \pm\delta} \right) G(n, m; z) \nonumber \\
&& - \sum_{\delta \neq 0} t_{n-\delta, n} G(n-\delta, m; z) - \sum_{\delta \neq 0} t_{m-\delta, m} G(n, m-\delta; z) \nonumber \\
&& =  \delta_{m, n^{\prime}}  \delta_{n, m^{\prime}} +  \delta_{n, n^{\prime}}  \delta_{m, m^{\prime}} \ . \label{eqn:eqnOfMotionSimplified}
}

\section{Numerical calculation of Green's function}
\label{sec:numericGreen}

\section{The tunneling of biexciton state through a disorder}
\label{sec:biexcitonTunneling}

\section{Conclusion}
\label{sec:conclusionGreenFunc}

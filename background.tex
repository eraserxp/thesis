
\chapter{Background material}
\label{ch:background}
This chapter  briefly outlines the background material that is directly related to the research in later chapters.


\section{Review of angular momentum theory}
\label{sec:reviewAngularMomentum}

In this section, we briefly review the theory of angular momentum. Instead of giving a comprehensive 
picture as various books\cite{edmonds-book, rose-book, brink-book, zare-book, silver-book, kleinman-book, Varshalovich-book, sakurai-book, RotSpect} have already done, we take a pragmatic approach that is more concerned with intuitive 
understanding, discussing only the concepts and ideas that are relevant to research in this thesis. 

\subsection{Symmetry groups}  
\label{sec:groupTheory}

%Symmetry is very important in that the laws of physics are easier to understand if the underlying symmetry are 
%appreciated. 
The theory of angular momentum is essentially the study of symmetry under rotations in quantum mechanics. 
As group theory is an important tool that is used to determine symmetry, we introduce the concepts of group theory
 in this subsection.

A set of operations $\{ A, B, C, \cdots \}$ is called a group if it satisfies the following properties:
\begin{itemize}
\item there is an identity operation $I$ in the set such that $A I = A$
\item  each operation $A$ has an inverse operation $A^{-1}$ in the same set such that $A A^{-1}= A^{-1} A = I$
\item  the multiplication of two operations is also an operation in the same set
\item  operations are associative such that $(AB)C = A(BC)$
\end{itemize}
%To simplify the expression of all the symmetry operations in a group, the concept of class is introduced. A class of a 
%group is defined as all the elements in the group that are conjugated to each other. By ``conjugated'', we means that 
% any two elements $A$ and $B$ in a class satisfy
%\oneline{
%X^{-1} A X = B
%}
%for any element $X$ in the group. A group can be divided into classes and then we can work with classes rather than 
%all the symmetry operations in a group. 

\subsection{Rotation operators and spherical harmonics}

As the theory of angular
 momentum concerns  symmetry under rotations, we start by defining the rotation operator in quantum 
 mechanics. Given a rotation $R$, we associate a rotation operator 
 $D(R)$ that transform a state from the original system to the rotated system
 \oneline{
 \ket{\psi}_{R} = D(R) \ket{\psi} \ .
 }
To construct the rotation operator, we denote the rotation operator $D(\mathbf{\hat{n}}, d\phi)$ for infinitesimal 
rotation by angle $d\phi$ about an axis $\mathbf{\hat{n}}$. Because $D(\mathbf{\hat{n}}, d\phi)$ must be linear
in $d\phi$ for small rotation angle and becomes the identity operator when $d\phi = 0$, it is useful to define the angular
 momentum operator $J_{\mathbf{\hat{n}}}$ for rotations about the axis $\mathbf{\hat{n}}$ as 
 \oneline{
 J_{\mathbf{\hat{n}}} \equiv i \hbar\lim_{\phi \rightarrow 0} \frac{D(\mathbf{\hat{n}}, \phi) - 1}{\phi} \ . \label{eqn:Jdef} \ ,
 }
such that
\oneline{
D(\mathbf{\hat{n}}, d\phi) = 1 - i\left(\frac{ J_{\mathbf{\hat{n}}}  }{\hbar} \right) d\phi \ . \label{eqn:rotationOperator}
}
From the above equation, we can regard $J_{\mathbf{\hat{n}}}$ as the generator of rotation in the sense that $ J_{\mathbf{\hat{n}}}$ produces the increment of a general function $f$ due to a rotation $R$ 
by $d\phi$ about the axis $\mathbf{\hat{n}}$, that is,
\oneline{
-i d\phi \left(\frac{ J_{\mathbf{\hat{n}}}  }{\hbar} \right) f = D(\mathbf{\hat{n}}, d\phi) f - f \ .
}
By compounding successively infinitesimal rotations about the same axis, we can obtain the finite rotation operator
\multiline{
D( \mathbf{\hat{n}}, \alpha) &=& \lim_{m\rightarrow \infty} \left[ 1- i\left(\frac{ \alpha}{m}\right) \left( \frac{ J_{\mathbf{\hat{n}}} }{\hbar}\right) \right]^m \nonumber \\
&=& \exp\left( \frac{ - i \alpha  J_{\mathbf{\hat{n}} } }{\hbar} \right) \nonumber \\
&=& 1 - \frac{i\alpha J_{\mathbf{\hat{n}} } }{\hbar} - \frac{1}{2!} \frac{ \alpha^2 J_{\mathbf{\hat{n}} }^2}{\hbar^2} + \cdots  \ . \label{eqn:rotationInTermsOfJ}
} 
where in the second step the following equation,
\oneline{
\lim_{x\rightarrow \infty}\left[ 1 + \left( \frac{1}{x} \right) \right]^{x} = e \ ,
}
was used. 
It is obvious that the set of all rotation operations satisfy the four conditions of a group (see 
\autoref{sec:groupTheory}). Correspondingly, 
 the infinite set of rotation operators also satisfy the same conditions and thus form a group which is called the full 
rotation group. 


Rotations have the following important property:  rotations about the same axis commute whereas rotations about 
different axes do not.  This property leads to the commutation relation of the angular momentum operator $J$ and
its Cartesian components. 
To illustrate this point, let's consider rotations $R_x$, $R_y$, and $R_z$ about the $x$, $y$ and $z$ axes respectively. In 
coordinate space, these rotations can be represented by $3\times 3$ matrices. For instance, the rotation by angle $\phi$ about the $z$ axis can be expressed as
\oneline{
R_{z} (\phi) = \mat{\cos\phi & -\sin\phi &  0 \\ \sin\phi& \cos\phi& 0 \\ 0& 0& 1} \ .
} 
In the case of an infinitesimal angle $\phi\rightarrow\varepsilon$, the above equation can be written as
\oneline{
R_{z} (\varepsilon) = \mat{1-\frac{\varepsilon^2}{2} & -\varepsilon &  0 \\ \varepsilon& 1-\frac{\varepsilon^2}{2}& 0 \\ 0& 0& 1} \ .
} 
By comparing the effect of a $y$-axis rotation followed by an $x$-axis rotation with that of an $x$-axis rotation 
followed by a $y$-axis rotation, we can show that 
\oneline{
R_x (\varepsilon) R_y (\varepsilon) - R_y (\varepsilon) R_x (\varepsilon) = R_z (\varepsilon^2) - I \ . \label{eqn:rotationCommutation}
}
Assuming the corresponding rotation operators satisfy a similar equation as \autoref{eqn:rotationCommutation} and
making use of \autoref{eqn:rotationOperator}, we obtain
\oneline{
\left[ J_x, J_y\right] = i\hbar J_{z} \ .
}
Repeating the above analysis for other axes, we obtain the commutation relations for the components of angular momentum
\oneline{
\left[ J_i, J_j\right] = i\hbar \varepsilon_{ijk} J_{k} \ , \label{eqn:angularMomentumCommutation}
}
where $i$, $j$, $k$ can be any one of $x$, $y$ and $z$, and $ \varepsilon_{ijk}$ is defined as 
\oneline{
\varepsilon_{ijk} = \left\{ 
  \begin{array}{l l}
    1 & \quad \text{if $(i, j, k)$ is an even permutation of $(x, y, z)$}\\
    -1 & \quad \text{if $(i, j, k)$ is an odd permutation of $(x, y, z)$} \\
    0  &   \quad \text{if two or more indices are equal} \\
  \end{array} \right. \ .
}

Many important properties follow from the angular-momentum commutation relation represented by \autoref{eqn:angularMomentumCommutation}. For example, by making use of \autoref{eqn:angularMomentumCommutation}, we can show that the operator $\mathbf{J}^2$ defined by
\oneline{
\mathbf{J}^2 \equiv J_x^2 + J_y^2 + J_z^2 \ ,
}
commutes with any one of $J_x$, $J_y$ and $J_z$, namely,
\oneline{
\left[ \mathbf{J}^2, J_k \right] = 0 , \;\;\; (k = x, y, z) \ .
}
Since $\mathbf{J}^2$ and $J_z$ commute, the eigenstates of $\mathbf{J}^2$ can be chosen to be also the eigenstates of
$J_z$. We denote these states by $\ket{j, m}$ such that
\multiline{
\mathbf{J}^2\ket{j, m} &=& j (j+1)\hbar^2 \ket{j, m} \ , \nonumber \\
J_z \ket{j, m} &=& m\hbar \ket{j, m} \ .
}
To determine the allowed values for $j$ and $m$, it is convenient to work with the non-Hermitian operators
\oneline{
J_{\pm} = J_x \pm i J_y \ ,
}
where $J_{+}$ is called the raising operator and $J_{-}$ is called the lowering operator. $J_{\pm}$ satisfy the following
commutation relations
\multiline{
\left[ J_{+}, J_{-}\right] &=& 2 \hbar J_z \ , \nonumber \\
\left[ J_{z}, J_{\pm}\right] &=& \pm \hbar J_{\pm} \ , \nonumber \\
\left[ \mathbf{J}^2, J_{\pm}\right] &=& 0 \ ,
}
all of which can easily be derived from \autoref{eqn:angularMomentumCommutation}. 

As a special case of the angular momentum operator, the orbital angular momentum operator 
$\mathbf{L} = \mathbf{r} \times \mathbf{p}$ has components that satisfy the same commutation relations
\oneline{
\left[ L_i, L_j\right] = i\hbar \varepsilon_{ijk} L_{k} \ .
}
Using the fact that momentum is the generator of translation
\multiline{
T(d\mathbf{x}) \ket{\mathbf{r}} = \left[ 1 - i \left(\frac{\mathbf{p}}{\hbar}\right) \cdot d\mathbf{x}\right]\ket{\mathbf{r}} = \ket{\mathbf{r} + dx \,\mathbf{\hat{x}}} \ ,
}
 we have
\multiline{
&&\left[ 1 - i \left(\frac{d\phi}{\hbar}\right) L_z\right] \ket{x, y, z} = \left[ 1 - i \left(\frac{d\phi}{\hbar}\right) \left( x p_y - y p_x \right)\right] \ket{x, y, z} \nonumber \\
&& = \left[ 1 - i \left(\frac{p_y}{\hbar}\right) x d\phi + i \left(\frac{p_x}{\hbar}\right) y d\phi \right] \ket{x, y, z} \nonumber \\
&& = \ket{x - y d\phi, y + x d\phi, z} \ . \label{eqn:rotationZ}
}
The above equation shows exactly the effect of an infinitesimal rotation about the $z$ axis, as one would expect. For an 
arbitrary state $\ket{\psi}$, an infinitesimal rotation about $z$ axis changes the wavefunction $\left\langle x, y, z \right.\ket{\psi}$ to 
\oneline{
\bra{x, y, z} \left[ 1 - i \left(\frac{d\phi}{\hbar}\right) L_z\right] \ket{\psi} = \left\langle x + y d\phi, y - x d\phi, z\right. \ket{\psi} \ .
}
By changing to spherical coordinates $(r, \theta, \phi)$, the above equation can be converted to
\multiline{
&&\bra{r, \theta, \phi} \left[ 1 - i \left(\frac{d\phi}{\hbar}\right) L_z\right] \ket{\psi} = \bra{r, \theta, \phi-d\phi} \psi \rangle \nonumber \\
&& =\bra{r, \theta, \phi} \psi \rangle -d\phi \frac{\partial}{\partial \phi} \bra{r, \theta, \phi} \psi \rangle \ .
}
Therefore, we obtain
\oneline{
L_{z} = -i\hbar \frac{\partial}{\partial \phi} \ .
}
In the way, we can also derive the expressions for the other orbital angular momentum operators, giving
\multiline{
L_x &=& - i\hbar \left( -\sin\phi \frac{\partial}{\partial \theta} - \cot\theta \cos\phi \frac{\partial}{\partial \phi}\right) \ , \nonumber \\
L_y &=& - i\hbar \left( \cos\phi \frac{\partial}{\partial \theta} - \cot\theta \sin\phi \frac{\partial}{\partial \phi}\right) \ .
}


The eigenfunctions of $\mathbf{L}^2$ and $L_z$ are known as the spherical harmonics, which are given by
\oneline{
\braket{\mathbf{\hat{n}}}{l, m} = Y_{l m}(\theta, \phi) \ ,
}
where $\theta$ and $\phi$ specify the orientation of $\mathbf{\hat{n}}$. The dependence of spherical harmonics on 
angles can be separated so that
\oneline{
Y_{l m}(\theta, \phi) = \Theta_{lm}(\theta) \Phi_{m}(\phi) \ , \label{eqn:sphericalHarmonicsExpression}
}
where
\oneline{
\Phi_{m}(\phi) = \sqrt{\frac{1}{2\pi}}\exp(i m \phi) \ ,
}
and 
\oneline{
\Theta_{lm}(\theta) = (-1)^{m} \left[ \frac{2 l + 1}{2} \frac{(l-m)!}{(l+m)!} \right]^{1/2} P_l^{m}(\cos\theta) \ ,
}
for $m\ge 0$ and $\Theta_{lm}(\theta) =(-1)^{m} \Theta_{l-m}(\theta) $ for $m<0$, and $P_l^{m}(\cos\theta)$ are the 
associated Legendre polynomials. Sometimes, it is more convenient
to work with modified spherical harmonics $C_{lm}$ defined by
\oneline{
C_{lm}(\theta, \phi) = \sqrt{\frac{4\pi}{2 l + 1}} Y_{lm}(\theta, \phi) \ .
}

\subsection{Rotation matrices}

Since an arbitrary rotation can be decomposed into rotations about three coordinate axes ($x$, $y$, and $z$ axis), 
it is usually convenient to express the orientations of a body in terms of rotations about some fixed axes. For this to 
work, we have to choose an initial orientation as a reference on which the rotations are operating. 
For a vector described by the two spherical polar angles $\theta$ and $\phi$, we use a reference orientation parallel 
to the space-fixed $Z$ axis. It is obvious to see that a rotation of this reference orientation through an angle $\theta$ about 
the space-fixed $Y$ axis and a rotation by an angle $\phi$ about the space-fixed $Z$ axis can reproduce the 
orientation of the vector. For a general body, three Euler angles $\phi$, $\theta$ and $\chi$ are needed to
 describe an arbitrary rotation.
We attach a second coordinate system 
$(x, y, z)$ to the body and refer to it as the body-fixed axis system. This axis system is called 
body-fixed as by construction it moves and rotates along with the body as a whole. 
In contrast, the original coordinate system $(X, Y, Z)$ is fixed in space, so it is called the space-fixed axis system. 
To describe the orientation of a body, we imagine that the body is initially at a position with its body-fixed axis system
 coincident with the space-fixed axis system, and then we carry out the following three rotations:
\begin{enumerate}
  \item rotate by $\chi$ about the space-fixed $Z$ axis, 
  \item  rotate by $\theta$ about the space-fixed $Y$ axis,
  \item  rotate by $\phi$ about the space-fixed $Z$ axis. 
\end{enumerate}
These three rotations can produce arbitrary orientation of the body. For an intuitive understanding of the Euler 
angles, we link the orientation of the body directly with $\phi$, $\theta$ and $\chi$. It can be shown that $\theta$ and
$\phi$ can be used to define the orientation of the body-fixed $z$ axis in the space-fixed axis system,  in the same way as
$\theta$ and $\phi$ are used to define the orientation of a vector. The angle $\chi$ measures a rotation about the 
body-fixed $z$ axis. It is an azimuthal angle about the $z$ axis just as $\phi$ is an azimuthal angle about the $Z$ axis. 

Based on \autoref{eqn:rotationInTermsOfJ}, the rotation operator that corresponds to the three Euler angles can be 
written as
\oneline{
D(\phi, \theta, \chi) = \exp\left(\frac{-i \phi J_{Z}}{\hbar}\right) \exp\left(\frac{-i \theta J_{Y}}{\hbar}\right) \exp\left(\frac{-i \chi J_{Z}}{\hbar}\right) \ . \label{eqn:rotationEulerAngles}
}
Because of the closure property of the full rotation group, the multiplication of three rotation operators in 
\autoref{eqn:rotationEulerAngles} is equivalent to another rotation operator
\oneline{
D(R) =  \exp\left(\frac{-i \alpha \mathbf{J_{\hat{n}}}}{\hbar}\right) = D(\phi, \theta, \chi) \ ,
}
which corresponds to a rotation $R$  through an angle $\alpha$ about an axis $\mathbf{\hat{n}}$. 

Now we study the 
matrix elements of the rotation operator $D(R)$. As a consequence of that fact that $\mathbf{J}^2$ commutes with 
any component of $\mathbf{J}$,  the rotation operator $D(R)$, a function of $J_{\mathbf{\hat{n}}}$, also commutes
with $\mathbf{J}^2$. Thus the eigenfunctions $\ket{j, m}$ of $\mathbf{J}^2$ are also eigenfunctions of the rotation
operator and rotations don't change the total angular momentum. 
As a result, to see the effect of a rotation on a state with a definite angular momentum, we only need to calculate the
 matrix elements of the rotation operator between two states with the same $j$ value, that is
\oneline{
 D^{j}_{m^{\prime} m} = \bra{j, m^{\prime}}  \exp\left(\frac{-i \alpha \mathbf{J_{\hat{n}}}}{\hbar}\right) \ket{j, m} \ .
}
The $(2 j + 1) \times (2 j + 1)$ matrix formed by $D^{j}_{m^{\prime} m}$ in the above equation is called the rotation
matrix.  Making use of \autoref{eqn:rotationEulerAngles}, we can easily show that
\multiline{
D^j_{m^{\prime}, m} (\phi, \theta, \chi) &=& \exp(-i\phi m^{\prime} - i\chi m) \matelement{j, m^{\prime} }{\exp(-i\theta J_Y /\hbar)}{j, m} \nonumber \\
&\equiv& \exp(-i\phi m^{\prime} - i\chi m) d^j_{m^{\prime}, m}(\theta) \ , \label{eqn:rotationMatrixEulerAngle}
}
where $d^j_{m^{\prime}, m}(\theta)$ is the element of a reduced rotation matrix. As the above equation shows, the
reduced rotation matrix can be calculated from the matrix representation of $J_Y$ and  its 
expression is\cite{zare-book}:
\multiline{
d^j_{m', m}(\theta) =&& \sum_{t} (-1)^t \frac{\left[ (j+m')! (j-m')! (j+m)! (j-m)! \right]^{1/2} }{(j + m' -t)! (j-m-t)! t! (t+m-m')!} \nonumber \\
&& \times \left( \cos\frac{\theta}{2} \right)^{2 j + m' - m - 2 t} \left( \sin\frac{\theta}{2} \right)^{2 t - m' + m } \ , \label{eqn:reducedRotationMatrix}
}
where the sum over $t$ is for all integers for which the factorial arguments are nonnegative. It can be shown that  $d^j_{m', m}(\theta)$
satifies the following symmetry relations\cite{zare-book}:
\oneline{
d^j_{m', m}(\theta) = (-1)^{m' - m} d^j_{m, m'}(\theta) = (-1)^{m' - m} d^j_{-m', m}(\theta) \ . \label{eqn:dSymmetryRelation}
}

To understand the physical meaning of the rotation matrix, we start from a state $\ket{j, m}$ and rotate it. Even though
the rotation $R$ doesn't change the $j$ value, we will generally expect the $m$ value to be different after the rotation. 
Consequently, one would be interested in knowing the probability of the system being found in a state $\ket{j, m^{\prime}}$. By inserting an identity relation, the final state can be written as
\multiline{
D(R) \ket{j, m} &=& \sum_{j^{\prime}} \sum_{m^{\prime}} \ket{j^{\prime},m^{\prime}} \bra{j^{\prime},m^{\prime}} D(R) \ket{j, m} \nonumber \\
&=& \sum_{m^{\prime}} \ket{j, m^{\prime}} \bra{j, m^{\prime}} D(R) \ket{j, m} \nonumber \\
&=&  \sum_{m^{\prime}}  \ket{j, m^{\prime}} D^{j}_{m^{\prime}, m}(R) \ . \label{eqn:JMTransformRotation}
}
From the above equation, one can see that the matrix element $D^{j}_{m^{\prime}, m}(R)$ is simply the 
probability amplitude for the rotated state to
be found in $\ket{j, m^{\prime}}$ when the original state is $\ket{j, m}$. 

\subsection{Properties of rotation matrices}

Let's consider some important properties of the rotation matrices. From the last subsection, we know  the basis ket 
$\ket{j, m}$ are orthonormal and remain so on rotation, this means that the matrices that represent rotations are
unitary, that is, $D^{-1} = D^{\dagger}$. More specifically, we have 
\oneline{
D^{j}_{m', m}(R^{-1}) = \left( D^{\dagger} \right)^{j}_{m', m}(R) = \left[ D_{m', m}^{j} (R)\right]^{*} \ , \label{rotationInverse}
}
where $R^{-1}$ denotes the inverse of the rotation $R$. 

From \autoref{eqn:reducedRotationMatrix}, we can see the matrix elements of reduced rotation matrices are always
 real. Taking this fact into account, and using  \autoref{eqn:rotationMatrixEulerAngle} and \autoref{eqn:dSymmetryRelation}, we arrive at
\multiline{
D^{j *}_{m', m}(R) &=&  \exp\left[-i\phi (-m') - i\chi (-m)\right]  d^{j}_{m', m}(\theta) \nonumber \\
&=&  \exp\left[-i\phi (-m') - i\chi (-m)\right] (-1)^{m-m'} d^{j}_{-m', -m}(\theta) \nonumber \\
&=& (-1)^{m-m'} D^{j}_{-m', -m}(R) \ . \label{eqn:complexToNormal}
}
As will be seen later, this equation is very useful because it helps us to deal with the complex conjugates of matrix 
elements of rotation matrices. 

Another important property  is that the 
rotation matrix elements  $D^{l}_{m', m}$ reduces to spherical harmonics when $l$ is an integer and $m'$ or $m$ is zero\cite{zare-book}, that is
\multiline{
D^{l}_{m, 0}(\phi, \theta, \chi) &=& C_{l m}^{*}(\theta, \phi) = \sqrt{\frac{2 j + 1}{4 \pi } } Y_{l m}^{*}(\theta, \phi) \ , \nonumber \\
D^{l}_{0, m}(\phi, \theta, \chi) &=& C_{l -m} (\theta, \phi) = \sqrt{\frac{2 j + 1}{4 \pi } } Y_{l -m} (\theta, \phi) \  .
   \label{eqn:relationDandY}
}
Given a general state $\ket{ \mathbf{\hat{n} } }$ with the unit vector 
$\mathbf{\hat{n}}$ with the orientation specified by $(\theta, \phi)$, we know from previous discussion that  it can be constructed 
from the state $\ket{\mathbf{\hat{z}}}$ by a rotation about $y$ axis by angle $\theta$ and a rotation about $z$ axis by
angle $\phi$. Thus we obtain
\multiline{
\ket{ \mathbf{\hat{n} } } &=& D(\phi, \theta, \chi) \ket{\mathbf{\hat{z}}} \nonumber \\
&=& \sum_{l^{\prime}}\sum_{m^{\prime}} D(\phi, \theta, \chi) \ket{l^{\prime}, m^{\prime} } \braket{l^{\prime}, m^{\prime} } {\mathbf{\hat{z}}}
}
where $\chi$ is undetermined. Multiplying both sides of the above equation by $\bra{l, m}$, we get 
\oneline{
\braket{l, m } {\mathbf{\hat{n}}} = \sum_{m^{\prime}} D^{l}_{m, m^{\prime} } (\phi, \theta, \chi)  \braket{l, m^{\prime} } {\mathbf{\hat{z}}} \ . \label{eqn:DtransformSphericalHarmonics}
}
Based on the definition of the spherical harmonics, $\braket{l, m^{\prime} } {\mathbf{\hat{z}}}$ is just 
$Y_{lm^{\prime}}^{*}(\theta, \phi)$ with $\theta = 0$ and $\phi$ undetermined. 
%At $\theta = 0$, the associated Legendre
% Polynomials $P_{lm}(cos\theta)$ vanish except when $m=0$. Then from the expression for the spherical harmonics,
% \autoref{eqn:sphericalHarmonicsExpression} , we know that $Y_{lm}$ vanish for $m\neq 0$. Then 
For $\theta=0$ the associated Legendre polynomials satisfy $P_{l}^{m}(1) =\delta_{m, 0}$, so that the spherical harmonics of \autoref{eqn:sphericalHarmonicsExpression} gives rise to 
\multiline{
\braket{l, m^{\prime} } {\mathbf{\hat{z}}}  &=& Y_{lm}^{*}(\theta=0, \phi) \delta_{m, 0} \nonumber \\
&=& \sqrt{ \frac{2 l + 1}{4 \pi } }  \delta_{m, 0} \ .
} 
Substituting this equation into \autoref{eqn:DtransformSphericalHarmonics} yields \autoref{eqn:relationDandY}. 

Finally, we consider the integral of rotation matrices. From \autoref{eqn:rotationMatrixEulerAngle} and 
\autoref{eqn:reducedRotationMatrix}, it can shown that the integral of a rotation matrix over $d\omega = \sin\theta d\theta d\phi  d\chi$ is a
product of delta functions\cite{RotSpect}, that is
\oneline{
\int D^j_{m^{\prime}, m}(\phi, \theta, \chi) d\omega = 8\pi^2 \delta_{j, 0}  \delta_{m^{\prime}, 0}  \delta_{m, 0} \ .
}
Similarly, the integral of two rotation matrices is given by\cite{zare-book}
\oneline{
\int D^{j_1 *}_{m_1^{\prime}, m_1}(\phi, \theta, \chi) \; D^{j_2}_{m_2^{\prime}, m_2}(\phi, \theta, \chi)d\omega = \frac{8\pi^2}{2 j_1 + 1}\delta_{j_1, j_2}  \delta_{m_1^{\prime}, m_2^{\prime}}  \delta_{m_1, m_2} \ . \label{eqn:normalizationRotationMatrix} \ ,
}
which describes the normalization condition for rotation matrices. 

\subsection{Coupling of angular momenta}

Suppose a system can be divided into two parts with different angular momenta $\mathbf{J}_1$ and $\mathbf{J}_2$ 
respectively. When the two parts of the system interact through some physical mechanism, $\mathbf{J}_1$ and 
$\mathbf{J}_2$ become coupled and we define the addition of the two angular momenta as
\oneline{
\mathbf{J} = \mathbf{J}_1 +  \mathbf{J}_2 \ . 
 }  
It is easy to verify that $\mathbf{J}$ also satisfies the commutation rules of angular momentum
 (\autoref{eqn:angularMomentumCommutation}) and thus the sum of two angular momenta  is also an angular momentum.

In quantum mechanics, the state of a system is described by the simultaneous eigenfunctions of a complete set of 
commuting operators. For the current system of angular momenta, there are two complete sets of angular momentum
operators.  One set is $\mathbf{J}_1^2$, $\mathbf{J}_2^2$, $J_{1Z}$, and $J_{2Z}$, and therefore we can use their
simultaneous eigenstates $\ket{j_1, m_1; j_2, m_2}\equiv \ket{j_1, m_1}\ket{j_2, m_2}$ to describe the system.
This representation is called the uncoupled representation. The other complete set of commuting angular 
 momentum operators is $\mathbf{J}_1^2$, $\mathbf{J}_2^2$, $\mathbf{J}^2$, and $J_Z$. Their simultaneous 
 eigenstates $\ket{(j_1 j_2)j, m}$ which we sometimes write as $\ket{j, m}$ are used to describe the system. This
  representation is called the coupled representation as the quantum numbers $j$ and $m$ of the coupled angular 
  momentum are used. Note that the bracket ``()'' in $\ket{(j_1 j_2)j, m}$ indicates the coupling of the two 
  angular momenta $\mathbf{J}_1$ and $\mathbf{J}_2$ and the result of the coupling is represented by the number $j$ immediately after the bracket.

The two representations describe the same set of states of the system so they can be related by an unitary
transformation connecting two bases, that is
\oneline{
\ket{(j_1 j_2)j, m} = \sum_{m_1, m_2} \ket{j_1, m_1; j_2, m_2} \braket{j_1, m_1; j_2, m_2} {(j_1 j_2)j, m} \ , \label{eqn:defCGCoeffs}
}
where $\braket{j_1, m_1; j_2, m_2} {(j_1 j_2)j, m}$ are called the Clebsch-Gordan coefficients. Due to symmetry 
considerations, it is often convenient to express the coefficients by the 3-$j$ symbols. The relation between the Clebsch-Gordan 
coefficients and 3-$j$ symbols is\cite{zare-book}
\begin{equation}
\left( 
\begin{array}{ ccc }
j_{1} & j_{2} & j_{3} \\
m_{1} & m_{2} & m_{3}
\end{array}
\right) \equiv (-1)^{j_{1} - j_{2} - m_{3}} (2j_{3} +1)^{-\frac{1}{2}} \langle j_{1}m_{1}, \; j_{2} m_{2} | j_{3}\; -m_{3}\rangle \ ,
\end{equation}
\begin{equation}
 \langle j_{1}m_{1}, \; j_{2} m_{2} | j_{3}\; -m_{3}\rangle \equiv (-1)^{j_{1} - j_{2} + m_{3}} (2j_{3} +1)^{\frac{1}{2}}
\left( 
\begin{array}{ ccc }
j_{1} & j_{2} & j_{3} \\
m_{1} & m_{2} & -m_{3}
\end{array}
\right)  \ . \label{CG-3j}
\end{equation}
The 3-$j$ symbols are more symmetric than the Clebsch-Gordan coefficients. We do not discuss the symmetry properties here as they are not important for understanding the thesis. The interested readers should refer to Zare's book\cite{zare-book}.  The 3-$j$ symbols are subject to the selection rules of angular momentum, that is
%%\begin{itemize}
%%\item 
\begin{equation}
\left(
\begin{array}{ccc}
j_{1}&j_{2}&j_{3} \\
m_{1}&m_{2}&m_{3}
\end{array}
\right)
=0 \;\;\; \mbox{unless $m_{1} + m_{2} + m_{3} = 0$ and $|j_1 - j_2| \leq j_3 \leq j_1 + j_2$, } \nonumber 
\end{equation}
which physically means that only certain angular momenta are coupled. Many calculations in molecular spectroscopy boil
down to the evaluation of 3-$j$ symbols. For details on the evaluation, readers should refer to Zare's book\cite{zare-book}, which mentions some efficient algorithms for calculating 3-$j$ 
symbols and gives the algebraic expressions for the commonly encountered 3-$j$ symbols in Table 2.5. 
 

%%
%%\item An even permutation of the columns of 3-$j$ symbol does not change its value and an odd permutation multiplier the initial value by $(-1)^{j_{1} + j_{2} + j_{3}}$. 
%%
%%\item Moreover,
%\begin{equation}
%\left\{
%\begin{array}{ccc}
%j_{1}&j_{2}&j_{3} \\
%m_{1}&m_{2}&m_{3}
%\end{array}
%\right\}
%=
%(-1)^{j_{1} + j_{2} + j_{3}}
%\left\{
%\begin{array}{ccc}
%j_{1}&j_{2}&j_{3} \\
%-m_{1}&-m_{2}&-m_{3}
%\end{array}
%\right\}
%\end{equation}
%This implies
%\begin{equation}
%\left\{
%\begin{array}{ccc}
%j_{1}&j_{2}&j_{3} \\
%0&0&0
%\end{array}
%\right\}
%=0 \;\; \;\;\mbox{unless } j_{1} + j_{2} + j_{3} = \mbox{even.}
%\end{equation}
%%
%\end{itemize}

Different from the coupling of two angular momenta, the coupling of three angular momenta has more than one possible
 coupling scheme, and all these coupling schemes are related by unitary transformations. We might couple $\mathbf{J}_{1}$, $\mathbf{J}_{2}$ and $\mathbf{J}_{3}$ in such a way that
  $\mathbf{J}_{1} + \mathbf{J}_{2}=\mathbf{J}_{12}$, $\mathbf{J}_{12} + \mathbf{J}_{3}= \mathbf{J}$. The eigenfunctions in this coupling scheme are given by $\ket{\left[(j_1 j_2) j_{12} j_3\right] j, m}$, where
  the brackets ``[]'' and ``()'',  indicate the coupling of two angular momenta $j_{12}$ and $j_3$ 
  to produce  $j$. We can also couple $\mathbf{j}_{2}$ and $\mathbf{j}_{3}$ to produce a new angular momentum 
  $\mathbf{j}_{23}$, and then couple $\mathbf{j}_{23}$ with  $\mathbf{j}_{1}$ to produce the total angular
   momentum  $\mathbf{j}$. The corresponding eigenstates are given by $\ket{\left[j_1(j_2 j_3) j_{23}\right] j, m}$. 
The transformation between the two different coupling schemes is
\oneline{
 \ket{ (j_1 j_{23}) j, m } = \sum_{ j_{12} } \braket{ (j_{12} j_3) j, m }{ (j_1 j_{23}) j^{\prime}, m^{\prime} } \ket{(j_{12} j_3) j^{\prime}, m^{\prime}} \delta_{j, j^{\prime}} \delta_{m, m^{\prime}} \ ,
}
where $\ket{ (j_1 j_{23}) j, m }$ is a short-hand way to write  $\ket{\left[j_1(j_2 j_3) j_{23}\right] j, m}$.  In the above 
equation, since $\braket{ (j_{12} j_3) j, m \allowbreak}{ (j_1 j_{23}) j^{\prime}, m^{\prime} }$ is a scalar product, it 
doesn't depend on the orientation of the coordinate system and is independent of the projection quantum numbers. Thus we can drop $m$ and $m^{\prime}$ and write it as 
$\braket{ (j_{12} j_3) j }{ (j_1 j_{23}) j^{\prime} }$. These recoupling coefficients can be replaced with the so-called
6-$j$ symbols defined as\cite{zare-book}
\multiline{
\left\{
\begin{array}{ccc}
j_{1}&j_{2}&j_{12} \\
j_{3}&j&j_{23}
\end{array}
\right\}
= (-1)^{j_1 + j_2 + j_3 + j} \left[ (2 j_{12} + 1 ) ( 2 j_{23} + 1 )\right]^{-\frac{1}{2} } \braket{ (j_{12} j_3) j }{ (j_1 j_{23}) j^{\prime} } \ . \nonumber \\
}


  
  
Similarly, the coupling of four angular momenta also has more than one possible coupling scheme. For example, 
  $|\left[(j_{1}j_{2})j_{12}(j_{3}j_{4})j_{34}\right] j, m \rangle$ is associated with the coupling scheme
 \oneline{
 \mathbf{J}_{1} + \mathbf{J}_{2}=\mathbf{J}_{12}\ , \;\;
 \mathbf{J}_{3} + \mathbf{J}_{4}=\mathbf{J}_{34} \ , \;\;
 \mathbf{J}_{12} + \mathbf{J}_{34}=\mathbf{J} \ , \nonumber
  } 
and  $|\left[(j_{1}j_{4})j_{14}(j_{2}j_{3})j_{23}\right] j, m \rangle$ is associated with another coupling scheme:
 \oneline{
 \mathbf{J}_{1} + \mathbf{J}_{4}=\mathbf{J}_{14}\ , \;\;
 \mathbf{J}_{2} + \mathbf{J}_{3}=\mathbf{J}_{23} \ , \;\;
 \mathbf{J}_{14} + \mathbf{J}_{23}=\mathbf{J} \ . \nonumber
  } 
The relation between the states corresponding to those two
coupling schemes is
\begin{eqnarray}
|\left[(j_{1}j_{4})j_{14}(j_{2}j_{3})j_{23}\right] j, m \rangle &=& \sum_{j_{12}}\sum_{j_{34}} \langle (j_{1}j_{2})j_{12}(j_{3}j_{4})j_{34}j  | (j_{1}j_{4})j_{14}(j_{2}j_{3})j_{23}j \rangle \nonumber \\
& & \times |\left[ (j_{1}j_{4})j_{14}(j_{2}j_{3})j_{23}\right] j, m \rangle \ , \label{9-j1}
\end{eqnarray}
and the correspondingly the 9-$j$ symbol is defined by\cite{zare-book}
\begin{eqnarray}
&&\braket{ (j_{1}j_{2})j_{12}(j_{3}j_{4})j_{34}j } {(j_{1}j_{4})j_{14}(j_{2}j_{3})j_{23}j } \nonumber \\
& & = \sqrt{(2j_{12} + 1) (2j_{34} +1)(2j_{14}+1)(2j_{23} + 1)}
\left\{
\begin{array}{ccc}
j_{1}& j_{2}&j_{12} \\
j_{3}&j_{4}&j_{34} \\
j_{14}&j_{23}&j 
\end{array}
\right\} \ . \nonumber \\ \label{9-j2}
\end{eqnarray}
The last column of the 9-$j$ symbol is related to the first coupling scheme and the last row is 
associated with the second coupling scheme.

\subsection{Clebsch-Gordan series}

After discussing the coupling of angular momenta, we now return to the rotation matrix and further develop its 
properties in the context of angular momentum coupling. We consider the connection between the uncoupled 
 $\ket{j_1, m_1}\ket{j_2, m_2}$ and the coupled $\ket{j, m}$ representations under a rotational transformation. 
Applying the rotation transformation on $\ket{j_1, m_1}$, $\ket{j_1, m_1}$ and $\ket{j, m}$ individually and using
\autoref{eqn:JMTransformRotation} we obtain
\multiline{
&&\sum_{m_1^{\prime}} \sum_{m_2^{\prime} } D^{j_1}_{m_1^{\prime}, m_1}(R) D^{j_2}_{m_2^{\prime}, m_2}(R) \ket{j_1, m_1^{\prime}}\ket{j_2, m_2^{\prime} } \nonumber \\
&& = \sum_{j}\sum_{m^{\prime}} \braket{j_1, m_1; j_2, m_2} {j, m} D^{j}_{m^{\prime}, m}(R) \ket{j, m^{\prime}} \ .
}
Multiplying both sides by $\bra{j_1, m_1^{\prime}}\bra{j_2, m_2^{\prime} }$ and making use of \autoref{eqn:complexToNormal}, we obtain the so-called Clebsch-Gordan 
series:
\multiline{
D^{j_1}_{m_1^{\prime}, m_1}(R)\; D^{j_2}_{m_2^{\prime}, m_2}(R) &=& \sum_{j} \braket{j_1, m_1; j_2, m_2} {j, m}  \braket{j_1, m_1^{\prime}; j_2, m_2^{\prime} } {j, m^{\prime} }  D^{j}_{m^{\prime}, m}(R)  \nonumber \\
&=&  \sum_{j} (2 j + 1) \threejm{j_1}{m_1}{j_2}{m_2}{j}{m}  \threejm{j_1}{m_1^{\prime} }{j_2}{m_2^{\prime}}{j}{m^{\prime}} D^{j *}_{m^{\prime}, m}(R) \nonumber \\ \label{eqn:CGSeries}
}
Similarly, the inverse Clebsch-Gordan series are given by
\multiline{
D^{j *}_{m^{\prime}, m}(R)
&= & \sum_{m_1}\sum_{ m_1^{\prime}} \sum_{m_2} \sum_{m_2^{\prime} } (2 j + 1) \threejm{j_1}{m_1}{j_2}{m_2}{j}{m}  \threejm{j_1}{m_1^{\prime} }{j_2}{m_2^{\prime}}{j}{m^{\prime}}  \nonumber \\
&& \times D^{j_1}_{m_1^{\prime}, m_1}(R) \; D^{j_2}_{m_2^{\prime}, m_2}(R) \ .
}

The Clebsch-Gordan series can help us to evaluate the integral over a product of three rotation matrix elements. 
Multiplying both sides of \autoref{eqn:CGSeries} by $D^{j_3}_{m_3^{\prime}, m_3}(R)$ we have
\multiline{
D^{j_1}_{m_1^{\prime}, m_1}(R)\; D^{j_2}_{m_2^{\prime}, m_2}(R)  \; D^{j_3}_{m_3^{\prime}, m_3}(R) &=& \sum_{j} (2 j + 1) \threejm{j_1}{m_1}{j_2}{m_2}{j}{m}  \threejm{j_1}{m_1^{\prime} }{j_2}{m_2^{\prime}}{j}{m^{\prime}}  \nonumber \\
&\times& D^{j}_{m^{\prime}, m}(R)^*  \; D^{j_3}_{m_3^{\prime}, m_3}(R) \ .
}
Integrating over $d\omega = \sin\theta d\theta d\phi  d\chi$ and utilizing the normalization condition of \autoref{eqn:normalizationRotationMatrix} then gives
\multiline{
&&\int D^{j_1}_{m_1^{\prime}, m_1}(\phi, \theta, \chi) \; D^{j_2}_{m_2^{\prime}, m_2}(\phi, \theta, \chi) \; D^{j_3}_{m_3^{\prime}, m_3}(\phi, \theta, \chi) d\omega \nonumber \\
&& = 8\pi^2  \threejm{j_1}{m_1^{\prime} }{j_2}{m_2^{\prime}}{j_3}{m_3^{\prime}} \threejm{j_1}{m_1}{j_2}{m_2}{j_3}{m_3}   \ .\label{eqn:integral3RotationMatrices}
}
The 
above equation is very important because alternative ways to evaluate the integral are very laborious. We will use this
equation in the derivation of the Wigner-Eckart theorem. 
Since the rotation matrix is proportional to the spherical
harmonics under some special conditions (see \autoref{eqn:relationDandY}), we have
\multiline{
&&\int D^{l_1}_{m_1, 0}(\phi, \theta, \chi) \; D^{l_2}_{m_2, 0}(\phi, \theta, \chi) \; D^{l_3}_{m_3, 0}(\phi, \theta, \chi) d\omega \nonumber \\
&& = 8\pi^2  \threejm{l_1}{m_1}{l_2}{m_2}{l_3}{m_3}  \threejm{l_1}{0 }{l_2}{0}{l_3}{0} \nonumber \\
&& =\sqrt{ \frac{4\pi\cdot 4\pi \cdot 4\pi}{( 2 l_1 + 1) (2 l_2 + 1)(2 l_3 + 1)} } \nonumber \\
&& \hspace{0.5cm}\times 2\pi \int Y_{l_1  m_1}^* (\phi, \theta) \; Y_{l_2  m_2}^* (\phi, \theta) \; Y_{l_3  m_3}^*  (\phi, \theta) \sin\theta d\phi d\theta \ ,
}
which gives rise to
\multiline{
&&\int Y_{l_1  m_1}(\phi, \theta) \; Y_{l_2  m_2}(\phi, \theta) \; Y_{l_3  m_3}(\phi, \theta)  \sin\theta d\theta d\phi  \nonumber \\
&& = \sqrt{ \frac{( 2 l_1 + 1) (2 l_2 + 1)(2 l_3 + 1)}{4\pi} } \threejm{l_1}{m_1}{l_2}{m_2}{l_3}{m_3}  \threejm{l_1}{0 }{l_2}{0}{l_3}{0} \ . \nonumber \\ \label{eqn:threeYintegral}
}



\subsection{Spherical tensor operators}

We have seen how the angular momentum wavefunction $\ket{j, m}$ transforms under a rotation.  Specifically, based 
on  \autoref{eqn:JMTransformRotation}, a general state $\ket{\alpha}$ is changed under a rotation $R$ according to
\oneline{
\ket{\alpha} \rightarrow D(R) \ket{\alpha} \ .  \label{eqn:ketStateTransform}
}
Now we study how an operator transforms under a rotation. Let's consider 
a so-called vector operator, for example $\mathbf{J}$, which is composed of operators $J_X$, $J_Y$ and $J_Z$. We know a vector in classical physics with three components transforms like $V_j \rightarrow \sum_{j} R_{i, j} V_j$ under a rotation R. It is reasonable to expect that the expectation value of a vector operator 
$\mathbf{V}$ transforms like a classical vector, that is
\oneline{
\matelement{\alpha}{V_i}{\alpha} \rightarrow \matelement{\alpha}{D^{\dagger}(R) V_i D(R) }{\alpha} = \sum_{j} R_{i, j} \matelement{\alpha}{V_j}{\alpha} \ , \label{eqn:expectationValueTransform}
}
where \autoref{eqn:ketStateTransform} has been used. From the above equation, it follows that the transformed operator in
the original basis is given by
\oneline{
D^{\dagger}(R) V_i D(R) = \sum_{j} R_{i, j} V_j \ . \label{eqn:operatorTransform}
}

By generalizing the definition of a vector $V_j \rightarrow \sum_{j} R_{i, j} V_j$, we define a tensor as a quantity which
transforms like
\oneline{
T_{ijk \cdots } \rightarrow \sum_{i^{\prime}} \sum_{j^{\prime} } \sum_{k^{\prime} } R_{i, i^{\prime}} R_{j, j^{\prime}} R_{k, k^{\prime} }\cdots T_{i^{\prime} j^{\prime} k^{\prime} \cdots } \ . 
}
The number of indices is called the rank of a tensor. Such a tensor is called a Cartesian tensor because its components
$T_{ijk \cdots }$ are defined with respect to the Cartesian axes. 

The problem with a Cartesian tensor is that it is reducible meaning it contains parts that
transform differently under rotations. Take for example a Cartesian tensor $\mathbf{T}$ formed from the product of two vectors $\mathbf{U}$ and $\mathbf{V}$ 
\oneline{
T_{ij} \equiv U_{i} V_{j} \ .
}
It can be shown that the tensor can be decomposed into the following parts:
\oneline{
U_{i} V_{j} = \frac{\mathbf{U}\cdot\mathbf{V} }{3} \delta_{i, j} + \frac{U_i V_j - U_j V_i }{2} + \left( \frac{U_i V_j +  U_j V_i }{2} -  \frac{\mathbf{U}\cdot\mathbf{V} }{3} \delta_{i, j} \right) \ . \label{eqn:decomposeTensor}
}
The first term on the right hand side is a scalar and has 1 independent component. The second term looks like a 
cross product of two vectors  and has 3  independent components. The third term is more complicated and we can
rewrite it as
\oneline{
(1-\delta_{i, j} ) \frac{U_i V_j - U_j V_i }{2}  + \delta_{i, j}\left( U_i V_i - \frac{\mathbf{U}\cdot\mathbf{V} }{3} \right) \ , \nonumber 
}
where the first term represents the off-diagonal part of the $3\times 3$ tensor and the second term represent the 
diagonal part. Now,  it can be easily seen that this term is symmetric and its trace is zero. So it contains
5 independent components. The number of independent components associated with the three terms in \autoref{eqn:decomposeTensor} is 1, 3, and 5 respectively, which are precisely the multiplicities of angular momenta
with $l=0$, $l=1$, and $l=2$ respectively. This suggests that the tensor $\mathbf{T}$ can be decomposed into tensors
that can transform like spherical harmonics with $l=0$, 1, and 2. 

From  the above example, it seems like the spherical harmonics can be 
used as irreducible tensors to represent any reducible tensor. This motivates us to define irreducible spherical
tensors based on the spherical harmonics. Before presenting the definition of a spherical tensor, let's first investigate
how the spherical harmonics tranform under rotations. For a direction eigenket $\ket{  \mathbf{\hat{n}}  }$, a rotation 
transforms it to another direction eigenket $\ket{  \mathbf{\hat{n }^{\prime} }  }$, that is
\oneline{
\ket{  \mathbf{\hat{n}^{\prime}  }  } = D(R) \ket{  \mathbf{\hat{n}}  } \ .
}
By taking the hermitian adjoint of the above equation and then multiplying by $\ket{l m}$ on the right, we obtain
\oneline{
Y_{lm}(\mathbf{\hat{n}^{\prime} }) = \sum_{m^{\prime} } Y_{lm^{\prime} }(\mathbf{\hat{n} })\left(D^{\dagger}\right)^{l}_{m^{\prime} m}(R) \ . \label{eqn:howYTransform}
}
The spherical harmonics can be used as both functions and operators just like the coordinates $x$, $y$, $z$ can also 
be used as position operators. Treating the spherical harmonics as operators, $Y_{lm}(\mathbf{\hat{n}^{\prime} })$ on
 the left hand side of \autoref{eqn:howYTransform} are the operators after the rotation transformation, which can be written in terms of  
the original operator $D^{\dagger}(R) Y_{lm}(\mathbf{\hat{n} }) D(R)$ based on \autoref{eqn:operatorTransform}. 
So that \autoref{eqn:howYTransform} becomes
\oneline{
D^{\dagger}(R) Y_{lm}(\mathbf{\hat{n} }) D(R) = \sum_{m^{\prime} } Y_{lm^{\prime} }(\mathbf{\hat{n} })D^{l\, *}_{m m^{\prime} }(R) \ .
}
Similarly, we define an irreducible spherical tensor operator of rank $k$ with $(2 k + 1)$ components as
\oneline{
D^{\dagger}(R) T_{q}^{(k)}(\mathbf{\hat{n} }) D(R) = \sum_{q^{\prime} =-k}^{k} D^{k\, *}_{q q^{\prime} }(R) T_{q^{\prime}}^{(k)} (\mathbf{\hat{n} })\ . \label{eqn:sphericalTensorDef1}
}
Replacing the rotation $R$ by its inverse $R^{-1}$ and using \autoref{rotationInverse}, the above definition can be recast as 
\oneline{
D(R)  T_{q}^{(k)} (\mathbf{\hat{n} }) D^{\dagger}(R)  = \sum_{q^{\prime} =-k}^{k} T_{q^{\prime}}^{(k)}(\mathbf{\hat{n} }) D^{k}_{q^{\prime} q }(R) \ . \label{eqn:sphericalTensorDef2}
}
Note that $T_{q}^{(k)} $ and $T_{q^{\prime}}^{(k)}$ here are defined with respect to the same 
axis system. Because the spherical harmonics and the irreducible spherical tensors transform in the same way under 
rotations, they are proportional to each other, that is\cite{zare-book}
\oneline{
T^{(k)}_{q} (\mathbf{\hat{n} }) = \left( \frac{4\pi k!}{(2 k + 1)!!} \right)^{1/2} Y_{k q} (\mathbf{\hat{n} })\ , \label{eqn:relationTwithY}
}
where 
\multiline{
n!! = \left\{ 
  \begin{array}{l l}
    n\cdot (n-2)\cdots 5\cdot 3 \cdot 1 & \quad \text{if $n$ is odd and positive}\\
   n\cdot (n-2)\cdots 6\cdot 4 \cdot 2 & \quad \text{if $n$ is even and positive} \\
    1 & \quad \text{if $n$ is $-1$ or 0}
  \end{array} \right.
}
From \autoref{eqn:relationTwithY}, it follows that the first-rank irreducible spherical tensor is just the modified 
spherical harmonics, namely
\oneline{
T^{(1)}_{q} = C_{1 q} \ . \label{eqn:TtoC}
}


In quantum mechanics, operators are usually written in terms of Cartesian vectors, like the position vector 
$\mathbf{r}$ and the electric dipole moment operator $\mathbf{d}$, and we want to know their corresponding spherical 
tensor operators. In such cases, the definition of the spherical tensors is 
not very useful and we need to find alternative ways. It turns out that the irreducible spherical tensors can be written in 
terms of Cartesian 
 coordinates. For example the first-rank irreducible spherical tensor is given by
\begin{equation}
T^{(1)}_{1} = - \frac{1}{\sqrt{2}} (\hat{X} + i \hat{Y}) \ , 
\end{equation}
\begin{equation}
T^{(1)}_{0}  = \hat{Z} \ ,
\end{equation}
\begin{equation}
T^{(1)}_{-1} =  \frac{1}{\sqrt{2}} (\hat{X} - i \hat{Y}) \ ,  
\end{equation}
where $\hat{X}$, $\hat{Y}$, and $\hat{Z}$ represents the units lengths for the Cartesian coordinate system.
For every cartesian vector $\mathbf{V}$, there is one corresponding first-rank spherical tensor $T^{(1)}(\mathbf{V})$.  
Similar to the irreducible spherical tensor,  $T^{(1)}(\mathbf{V})$ can also be written in terms of cartesian vectors 
$\mathbf{V}$, that is
\begin{equation}
T^{(1)}_{1}(\mathbf{V}) = - \frac{1}{\sqrt{2}} (V_{X}\,\hat{X} + i V_{Y}\,\hat{Y}) \ , \label{t11}
\end{equation}
\begin{equation}
T^{(1)}_{0}(\mathbf{V})  = V_{Z}\hat{Z} \ , \label{t10}
\end{equation}
\begin{equation}
T^{(1)}_{-1}(\mathbf{V}) =  \frac{1}{\sqrt{2}} (V_{X}\hat{X} - i V_{Y}\hat{Y}) \ .  \label{t1-1}
\end{equation}


\subsection{Coupling of spherical tensors}

As we have seen, the spherical tensors behave like spherical harmonics. As a result, the spherical tensors  
couple in the same way as angular momenta. For example, two spherical tensors $\mathbf{R}^{k_1}$ and  
$\mathbf{S}^{k_2}$ can be combined to form a tensor of rank $K$,
\oneline{
T_{P}^{(K)}(\mathbf{R}^{k_1}, \mathbf{S}^{k_2} ) = \sum_{p_1,  p_2} \braket{k_1, p_1; k_2, p_2}{K, P} T_{p_1}^{(k_1)}(\mathbf{R}) T_{p_2}^{(k_2)}(\mathbf{S}) \ . \label{eqn:tensorCouple}
}
In practice, we can regard the above equation as the coupling of two angular momenta $\mathbf{j}_1$ and 
$\mathbf{j}_2$ to form $\mathbf{j}$. Comparing \autoref{eqn:defCGCoeffs} with \autoref{eqn:tensorCouple}, we can easily see that $j_1$ corresponds to $k_1$, $p_1$ to $m_1$, $j_2$ to $k_2$, $p_2$ to $m_2$,
$j$ to $K$, and $m$ to $P$. Similarly, $K$ can only takes the values from $|k_1 - k_2|$ to $k_1 + k_2$. We call 
$T_{P}^{(K)}$ the tensor product of $\mathbf{R}^{k_1}$ and  $\mathbf{S}^{k_2}$ and sometimes denote it as
\oneline{
T_{P}^{(K)}(\mathbf{R}^{k_1}, \mathbf{S}^{k_2} ) = \left[  \mathbf{R}^{(k_1)} \otimes \mathbf{S}^{(k_2)} \right]^{(K)}_{P} \ . \nonumber
}

For later reference, the two most important cases of \autoref{eqn:tensorCouple} are\cite{zare-book}
\oneline{
 \left[   \mathbf{A}^{(k)} \otimes \mathbf{B}^{(k)} \right]^{(0)}_{0}  = (2 k + 1)^{-\frac{1}{2}} \sum_{q} (-1)^{k-q} T^{(k)}_q  (\mathbf{A})  T^{(k)}_{-q} (\mathbf{B})\  , \label{eqn:tensorContraction00} 
}
\multiline{
 \left[  \mathbf{A}^{(1)} \otimes \mathbf{B}^{(1)} \right]^{(2)}_{q}  &=& \sum_{q_{A} } \braket{1, q_{A} ; 1, q_{B} } {2, q} T^{(1)}_{q_{A} } (\mathbf{A})  T^{(1)}_{q_B } (\mathbf{B}) \nonumber \\
&=& \sum_{q_{A} } (-1)^q \sqrt{5} \threejm{1}{q_A }{1}{q_B }{2}{-q} T^{(1)}_{q_{A} } (\mathbf{A})  T^{(1)}_{q_B } (\mathbf{B})
\  . \nonumber \\ \label{eqn:tensorContraction112} 
}


It is helpful to examine further the meaning of $\left[  A^{(1)} \otimes B^{(1)} \right]^{(0)}_{0}$. 
Based on \autoref{eqn:tensorContraction00}) we obtain
\begin{eqnarray}
\left[ \mathbf{A}^{(1)}\otimes \mathbf{B}^{(1)} \right]_{0}^{(0)} =\frac{1}{\sqrt{3}}\left[T^{(1)}_{1}(\mathbf{A}) T^{(1)}_{-1}(\mathbf{B}) -T^{(1)}_{0}(\mathbf{A}) T^{(1)}_{0}(\mathbf{B}) +  T^{(1)}_{-1} (\mathbf{A}) T^{(1)}_{1}(\mathbf{B})\right] \ . \nonumber \\
\end{eqnarray}
Substituting Eqs. (\ref{t11}), (\ref{t10}) and (\ref{t1-1}) into the above equation gives
\begin{equation}
\left[ \mathbf{A}^{(1)}\otimes \mathbf{B}^{(1)} \right]_{0}^{(0)}  = -\frac{1}{\sqrt{3}} (A_{X}B_{X}+ A_{Y}B_{Y} + A_{Z}B_{Z}) = -\frac{1}{\sqrt{3}} \mathbf{A}\cdot\mathbf{B} \ . \label{eqn:tensorDotProduct}
\end{equation}
Therefore, the rank-zero tensor product of two rank-one tensors is equivalent to the scalar product of the corresponding
 vectors except for a factor.  
Similarly, it can be shown that the tensor product of rank $k=1$ is related to the cross product by 
\begin{equation}
\left[ A^{(1)} \otimes B^{(1)} \right]_{q}^{(k)} = (\mathbf{A}\times \mathbf{B})\cdot \mathbf{\hat{e}}_{q} \ ,
\label{eqn:crossProduct}
\end{equation}
where
\begin{eqnarray}
\mathbf{\hat{e}}_{+1} &=& -\frac{1}{\sqrt{2}}(\mathbf{\hat{X}} + i \mathbf{\hat{Y}}) \ , \nonumber \\
\mathbf{\hat{e}}_{0} &=& \mathbf{\hat{Z}} \ , \nonumber \\
\mathbf{\hat{e}}_{-1} &=& \frac{1}{\sqrt{2}}(\mathbf{\hat{X}} - i \mathbf{\hat{Y}}) \ .
\end{eqnarray}

It is worthwhile mentioning that all the results  so far assume that the spherical
tensor operators are defined with respect to one specific fixed axis system. However,
 in practice, this assumption is usually not valid. Some tensor operators are naturally described in body-fixed coordinates, for example the molecular
dipole moment operator, and other operators are more conveniently expressed in space-fixed coordinates, for instance
the external field operator.  This leads to the question, ``How the spherical harmonics defined in the 
space-fixed axis system 
be related to those in body-fixed axis system?''. 
Suppose the molecule-fixed axes can be obtained by a rotation $R$ of the 
space-fixed axes. Equation (\ref{eqn:sphericalTensorDef2}) will still be valid.
% and we write it here:
%\oneline{
%D(R)  T_{q}^{(k)}(\mathbf{T}) D^{\dagger}(R)  = \sum_{q^{\prime} =-k}^{k} T_{q^{\prime}}^{(k)}(\mathbf{T})  D^{k}_{q^{\prime} q }(R) \ . \label{eqn:sphericalTensorDef3}
%}
Now, the tensor operators are all defined in the space-fixed axis system and the body-fixed axes can be obtained by rotation 
$R$ of the space-fixed axes. We recall that a transformation $S$ turns a wavefunction $\psi$ into  $S\psi$ and  turns 
an operator $T$ into $S T S^{-1}$ in the transformed basis set. In the case of the rotation transformation $D$, 
the new operator in the rotated basis set is given by $D T D^{-1} = D T D^{\dagger}$. Therefore, the left hand side of
\autoref{eqn:sphericalTensorDef2} can be interpreted as new tensor operator in the rotated axis system (or body-fixed axis
 system), and \autoref{eqn:sphericalTensorDef2} can be written as
\oneline{
T_{b}^{(k)}(\mathbf{T}) = \sum_{s =-k}^{k} D^{k}_{sb }(R) T_{s}^{(k)}(\mathbf{T})  \ , \label{eqn:sTob}
}
where $T_{b}^{(k)}$ is the tensor operator defined in body-fixed axis system and $T_{s}^{(k)}$ is the tensor operator
defined in space-fixed axis system. Note that $s$ labels space-fixed components and $q$ labels body-fixed 
components. Replacing $R$ with its inverse $R^{-1}$, using \autoref{rotationInverse} and exchanging $b$ and $s$, we obtain the inverse of
 \autoref{eqn:sTob}, namely
\oneline{
T_{s}^{(k)}(\mathbf{T}) = \sum_{s =-k}^{k} D^{k}_{sb }(R)^* \; T_{b}^{(k)}(\mathbf{T})  \ . \label{eqn:bTos}
}


\subsection{Matrix elements of tensor operators} 

 We now consider the evaluation of  matrix elements of tensor operators with respect to angular-momentum eigenstates. 
Given a matrix element $\matelement{\eta, j, m}{T_q^{(k)}(\mathbf{A}) } {\eta^{\prime}, j^{\prime}, m^{\prime}}$ where
$\eta$ and $\eta^{\prime}$ denotes quantum numbers other than rotational quantum numbers, we can rotate the bra, 
operator and ket using \autoref{eqn:JMTransformRotation} and \autoref{eqn:sTob}, the result must be the same. 
Therefore we obtain
\multiline{
&&\matelement{\eta, j, m}{T_q^{(k)}(\mathbf{A}) } {\eta^{\prime}, j^{\prime}, m^{\prime}} \nonumber \\
&&= \sum_{n, n^{\prime}, p} D^j_{n m}(R)^* \; D^k_{p q}(R) \; D^{j^{\prime}}_{n^{\prime} m^{\prime} }(R) \matelement{\eta, j, n}{T_p^{(k)}(\mathbf{A}) } {\eta^{\prime}, j^{\prime}, n^{\prime}}  
\ . \label{eqn:matrixElement}
}
Integrating over $d\omega = \sin\theta d\theta d\phi  d\chi$, we obtain on the left hand side $8\pi^2 \times$ ``original 
term'' as the matrix element is a scalar and independent of any angles, and we have on the right hand side the integral over three rotation matrices, that is
\oneline{
\int d\omega D^{j *}_{n m}(R) \; D^k_{p q}(R) \; D^{j^{\prime}}_{n^{\prime} m^{\prime} }(R) \ .
}
Rewriting $D^{j *}_{n m}(R)$ in terms of $D^j_{-n -m}(R)$ using \autoref{eqn:complexToNormal} and evaluating the integral 
using \autoref{eqn:integral3RotationMatrices} gives the value as
\oneline{
(-1)^{n-m} 8\pi^2  \threejm{j}{-m}{k}{q}{j^{\prime} }{m^{\prime} }  \threejm{j}{-n }{k}{p}{j^{\prime} }{n^{\prime}} \ ,
}
which then used in \autoref{eqn:matrixElement} gives
\multiline{
&&\matelement{\eta, j, m}{T_q^{(k)}(\mathbf{A}) } {\eta^{\prime}, j^{\prime}, m^{\prime}} \nonumber \\
&& = (-1)^{j-m} \threejm{j}{-m}{k}{q}{j^{\prime} }{m^{\prime} } \Bigg\{ \sum_{n, n^{\prime}, p} (-1)^{j-n} \threejm{j}{-n }{k}{p}{j^{\prime} }{n^{\prime}}  \nonumber \\
&&\hspace{0.5cm}  \times \matelement{\eta, j, n}{T_p^{(k)}(\mathbf{A}) } {\eta^{\prime}, j^{\prime}, n^{\prime}}  \Bigg \} \ .
}
The term in the braces is indepedent of the projection quantum numbers and we write it as $\matelement{\eta, j}{|T_p^{(k)}(\mathbf{A})| } {\eta^{\prime}, j^{\prime}}$. Thus, we have\cite{RotSpect}
\oneline{
\matelement{\eta, j, m}{T_q^{(k)}(\mathbf{A}) } {\eta^{\prime}, j^{\prime}, m^{\prime}} =  (-1)^{j-m} \threejm{j}{-m}{k}{q}{j^{\prime} }{m^{\prime} }  \matelement{\eta, j}{|T^{(k)}(\mathbf{A})| } {\eta^{\prime}, j^{\prime}} \ . \label{eqn:wignerEckart}
}
This is the Wigner-Eckart theorem and states that the matrix element of a tensor operator in angular-momentum
eigenstates can be separated into a part which describes all the angular dependence  and another part which is 
independent of projection quantum numbers and hence of orientation. 

To make the Wigner-Eckart theorem useful, we need to know how to evaluate the reduced matrix element 
$\matelement{\eta, j}{|T_p^{(k)}(\mathbf{A})| } {\eta^{\prime}, j^{\prime}}$. The usual approach is to calculate the 
matrix element for some special cases and then derive the reduced matrix element from the result. For example,
if we want to calculate the matrix elements associated with the tensor operator of the modified spherical harmonics 
$C_{kq}(\theta, \phi)$, $\matelement{ l_1, m_1}{C_{kq}(\theta, \phi) } { l_2, m_2}$, 
%\oneline{
%\matelement{ l_1, m_1}{C_{kq}(\theta, \phi) } { l_2, m_2} =  (-1)^{l_{1} -m_{1} } \threejm{l_1}{-m_1}{k}{q}{l_2 }{m_2 }  \matelement{ l_1}{|C^{(k)}| } {l_2} \ , \label{eqn:matrixElementSpecial}
%}
we can consider the special case when $q=0$, $m_1 =0$ and $m_2 =0$. The corresponding matrix element can be converted to an integral over a product of three spherical harmonics, that is
\oneline{
\matelement{ l_1, 0}{C_{k0}(\theta, \phi) } { l_2, 0} = \sqrt{\frac{4\pi}{2 k + 1}} \matelement{ Y_{ l_1 0} }{Y_{k0} } { Y_{ l_2 0}} \ , \nonumber \\
}
which gives rise to 
\oneline{
\left[ (2 l_1 + 1) (2 l_2 + 1) \right]^{1/2}  {\threejm{l_1}{0 }{k}{0}{l_2}{0}}^2  \ ,  \label{eqn:resultSpecialCase}
}
based on \autoref{eqn:threeYintegral}. 
Alternatively, from the Wigner-Eckart theorem we have
\oneline{
\matelement{ l_1, 0}{C_{k0}(\theta, \phi) } { l_2, 0} =  (-1)^{l_1}  \threejm{l_1}{0}{k}{0}{l_2 }{0 }  \matelement{ l_1}{|C^{(k)}| } {l_2} \ . \label{eqn:resultW-Etheorem}
}
Comparing \autoref{eqn:resultSpecialCase} with \autoref{eqn:resultW-Etheorem} gives
\oneline{
 \matelement{ l_1}{|C^{(k)}| } {l_2} = (-1)^{l_1} \left[ (2 l_1 + 1) (2 l_2 + 1) \right]^{1/2} \threejm{l_1}{0}{k}{0}{l_2 }{0 } \ .
}
From this reduced matrix element, any matrix elements can be calculated using the Wigner-Eckart theorem, so that
\autoref{eqn:wignerEckart} gives
\multiline{
\matelement{ l_1, m_1}{C_{kq}(\theta, \phi) } { l_2, m_2} &=&  (-1)^{m_{1} } \left[ (2 l_1 + 1) (2 l_2 + 1) \right]^{1/2}  \threejm{l_1}{0}{k}{0}{l_2 }{0 } \nonumber \\
&\times& \threejm{l_1}{-m_1}{k}{q}{l_2 }{m_2 }  \ . \label{eqn:matrixElementModifiedY}
}


\section{Application of angular momentum theory}
\label{sec:applicationAMT}

We have reviewed the theory of angular momentum in \autoref{sec:reviewAngularMomentum}. To  lay the foundation for later chapters, we apply the theory of angular momentum to solve some
problems closely related to the research in this thesis. 

\subsection{Diatomic molecule in external field}
\label{sec:moleculeInField}
The first problem involves calculating the dressed rotational states of molecules in an external field. For simplicity, we 
only consider $^1\Sigma$ molecules with a permanent dipole momentum and no spins. Since only rotational states are
relevant here, the total Hamiltonian can be written as
\oneline{
H = H_{\rm rot} + H_{I} \ ,
}
where $H_{\rm rot} $ describes the rotational states of the system without any external field 
and $H_{I}$ describes the interaction with the external field. For $^1\Sigma$ molecules,  
\oneline{
H_{\rm rot} = B \hat{N}^2 \ ,
}
where $\hat{N}$ is the rotational angular momentum operator and $B$ is the rotational constant obtained by averaging 
over all quantum states other than rotational states. To obtain the eigenstate of the total Hamiltonian, we can use the
bare rotational states $\ket{N, M}$ as a basis set.  We want to know the matrix representation of $H$ and calculate its 
eigenvalues and eigenvectors. The rotational part of the Hamiltonian only contributes to the diagonal part of the matrix 
and is given by
\oneline{
\matelement{N, M}{H_{\rm rot}}{N^{\prime}, M^{\prime}} = \delta_{N, N^{\prime}} \delta_{M, M^{\prime}} B N (N+1) \ . 
}
The evaluation of the contribution from the interaction part is more involved and we discuss it in the following two 
cases.

First, consider a static electric (DC) field for which
\oneline{
H_{I} =  -\mathbf{d} \cdot \mathbf{E} = d \mathcal{E}_{ \tiny \rm DC}  \;\cos\theta \ ,
}
where $\mathbf{d}$ is the permanent dipole, $\mathbf{E}$ is the external field, and $\theta$ is the 
angle between $\mathbf{d}$ and $\mathbf{E}$. To evaluate the corresponding matrix elements, we need to 
calculate $\matelement{N, M}{\cos\theta}{N^{\prime}, M^{\prime}}$. Because $\cos\theta$ can be expressed
in terms of a spherical harmonic
\oneline{
\cos\theta = \sqrt{\frac{4\pi}{3} }Y_{10}(\theta, \phi) = C_{10} \ ,
}
\autoref{eqn:matrixElementModifiedY} gives
\multiline{
&&\matelement{N, M}{H_{I}}{N^{\prime}, M^{\prime}} =  E d \matelement{N, M}{C_{10}}{N^{\prime}, M^{\prime}} \nonumber \\
 &&=(-1)^{M }  d \mathcal{E}_{ \tiny \rm DC}\sqrt{ (2N + 1) (2N^{\prime} + 1)}  \threejm{N}{0}{1}{0}{N^{\prime} }{0 }
\threejm{N}{-M}{1}{0}{N^{\prime} }{M^{\prime} } \ . \nonumber \\
}

Second, we consider a  laser (AC) field which is off-resonant with any vibrational or electronic states 
of the molecule. The strength of the AC field  is given by 
\oneline{
\mathcal{E}_{ AC} (t)= \mathcal{E}_{ AC, 0}\; \cos(\omega t) \ ,
}
where $\omega$ is the frequency of the laser field. In this case, the interaction Hamiltonian is given by \cite{friedrich-95}
\oneline{
H_{I}(t) = - d \mathcal{E}_{ AC} (t) \cos\theta_{AC} - \frac{1}{2} \mathcal{E}_{ AC}^2 (t) \left( \alpha_{\parallel} \cos^2\theta_{AC} + \alpha_{\perp}\sin^2 \theta_{AC} \right) \ , \label{eqn:Hit}
}
where $\theta_{AC}$ is the angle between the dipole moment of the molecule and the AC field, and $\alpha_\|$ and
$\alpha_{\perp}$ are the parallel and perpendicular polarizabilities of the molecule respectively.  
The interaction Hamiltonian is a function of time, and changes very fast as $\mathcal{E}_{ AC} (t)$ oscillates at optical 
frequency.
During one period of rotational motion of the molecule, the interaction Hamiltonian has oscillated many times. 
Therefore, we can average $H_{I}(t)$ over one oscillation period of the light and obtain the effective interaction
 Hamiltonian
\oneline{
H_{I} =  - \frac{1}{4} \mathcal{E}_{ AC, 0}^2 \left( \Delta \alpha \cos^2\theta_{AC} + \alpha_{\perp} \right) \ , \label{eqn:Hit}
}
where $\Delta \alpha = \alpha_{\parallel} - \alpha_{\perp}$. 
As we did in the case of the DC field, to calculate the matrix elements of the interaction Hamiltonian, we rewrite 
$\cos^2\theta_{AC}$ in terms of spherical harmonics  and express the matrix elements in terms of those of the modified spherical harmonics. As a consequence, we obtain
\multiline{
&&\matelement{N, M}{H_{I}}{N^{\prime}, M^{\prime}} \nonumber \\
&& = - \frac{1}{4} \mathcal{E}_{ AC, 0}^2 \matelement{N, M}{\left( \Delta\alpha \cos^2\theta_{AC} + \alpha_{\perp} \right)}{N^{\prime}, M^{\prime}}\nonumber \\ 
&&= - \frac{1}{4} \mathcal{E}_{ AC, 0}^2 \matelement{N, M}{\left[ \Delta\alpha \left(\frac{2}{3} C_{20} + \frac{1}{3} \right) + \alpha_{\perp} \right]}{N^{\prime}, M^{\prime}} \nonumber \\
&&= - \frac{\mathcal{E}_{ AC, 0}^2}{12}  \bigg[ (-1)^M 2\Delta\alpha\sqrt{(2 N + 1)(2 N^{\prime} + 1)} \threejm{N}{0}{2}{0}{N^{\prime} }{0 }
\threejm{N}{-M}{2}{0}{N^{\prime} }{M^{\prime} } \nonumber \\
&& \nonumber \\
&&  \hspace{2.5cm} +  \left(\Delta\alpha + 3\alpha_{\perp} \right) \delta_{N, N^{\prime}}  \delta_{M, M^{\prime}} \bigg ]
}
based on  \autoref{eqn:matrixElementModifiedY}. 

\subsection{Dipole-dipole interaction}
\label{sec::ddInteraction}

The second problem we consider calculates the matrix elements of the dipole-dipole interaction between two molecules by using the theory of angular momentum. 
Consider two molecules A and B, in rotational states $|N_{A}M_{A}\rangle$ and $|N_{B}M_{B}\rangle$, 
respectively. In free space, one molecule can rotate around the other and the state of the two molecules
can be expressed as
\begin{equation}
|N_{A}M_{A}\rangle |N_{B}M_{B}\rangle |lm\rangle  \ , \nonumber
\end{equation}
where $l$ is the orbital angular momentum  of A around B. 
However, in the solid state or optical lattices, the position of molecules are fixed to a good approximation, and they cannot rotate 
around each other. In this case, their states are given by $|N_{A}M_{A}\rangle |N_{B}M_{B}\rangle$ if there is no 
interaction between them. If we consider the interaction between the two molecules, we can always use $|N_{A}M_{A}\rangle |N_{B}M_{B}\rangle$ as basis set. By expanding the Hamiltonian in this basis set and diagonalizing 
it, we can get the new eigenstates.

As the first step, we have to know  the interaction between the two molecules. If A has a permanent dipole  $\mathbf{d}_{A}$ and B has a permanent dipole $\mathbf{d}_{B}$ and the vector connecting their centers of mass is $\mathbf{R}$, then the dipole-dipole interaction between A and B is given by
\begin{equation}
\hat{V}_{\mbox{\tiny dd}} = \left(\frac{1}{R^3}\right)\left[ \mathbf{d}_{A}\cdot\mathbf{d}_{B}-3(\mathbf{d}_{A}\cdot\mathbf{\hat{R}})(\mathbf{d}_{B}\cdot\mathbf{\hat{R}})\right]  \ , \label{dd-operator}
\end{equation}
where $\mathbf{d}_{A}$ and $\mathbf{d}_{B}$ are the dipole moment operators for molecule $A$ and molecule 
$B$. 
Here we only consider the dipole moment. There also exists quadrupole, octopole, and higher order moments, but their magnitudes are so small that we can ignore them at large distances, say, 400 nm. Our final goal is to evaluate the matrix element $\langle N_{A}M_{A}|\langle N_{B}M_{B} | \hat{V}_{\mbox{\tiny dd}}|N_{A}^{'}M_{A}^{'}\rangle |N_{B}^{'}M_{B}^{'}  \rangle$ by rewriting $\hat{V}_{\mbox{\tiny dd}}$  in terms of spherical harmonics, and then  in terms of 3-$j$ symbols which can be calculated numerically. 

The first term in the square bracket of \autoref{dd-operator}, can be expressed in terms of a tensor product using 
\autoref{eqn:tensorDotProduct}, that is
\begin{equation}
\mathbf{d}_{A}\cdot\mathbf{d}_{B} = - \sqrt{3} \left[ \mathbf{d}_{A}^{(1)} \otimes \mathbf{d}_{B}^{(1)} \right]_{0}^{(0)} \ . \label{dot-product}
\end{equation}
The second term in the square bracket of \autoref{dd-operator} is the product of two scalars $(\mathbf{d}_{A}\cdot\mathbf{\hat{R} })(\mathbf{d}_{B}\cdot\mathbf{\hat{R}})$, which we consider as a special case of the dot product of two vectors in one-dimensional space. So that according to \autoref{dot-product}, we can write
\begin{equation}
(\mathbf{d}_{A}\cdot\mathbf{\hat{R}})(\mathbf{d}_{B}\cdot \mathbf{\hat{R} })= 3\left[ \left[ \mathbf{d}_{A}^{(1)} \otimes \mathbf{\hat{R}}^{(1)} \right]^{(0)}\otimes \left[ \mathbf{d}_{B}^{(1)} \otimes \mathbf{\hat{R}}^{(1)} \right]^{(0)}  \right]_{0}^{(0)}  \ . \label{eqn:dRdR}
\end{equation}
There is a simple understanding of the above equation: an angular momentum $\mathbf{d}_{A}$ with $j_{1}=1$ 
couples  another angular momentum $\hat{R}$ with $j_{2}=1$ to give rise to a new angular momentum, and an 
angular momentum $\mathbf{d}_{B}$ with $j_{3}=1$ couples  with another angular momentum $\hat{R}$ with $j_{4}=1$ 
to give rise to another new angular momentum, and then these two new angular momenta couple with each other. 

The form of \autoref{eqn:dRdR} is not convenient for calculations. We know only $\mathbf{d}_A$ and $\mathbf{d}_B$ 
operate on the rotational states $\ket{N_A, M_A} \ket{N_B, M_B}$, but they are coupled with $\mathbf{\hat{R}}$, which makes the calculation of the matrix element associated with \autoref{eqn:dRdR} cumbersome. Therefore, it is necessary to regroup these operators so that the dipole operators are separated from the position operators. 
Let us expand $\left[ \left[ \mathbf{d}_{A}^{(1)} \otimes \mathbf{\hat{R}}^{(1)} \right]^{(0)}\otimes \left[ \mathbf{d}_{B}^{(1)} \otimes \mathbf{\hat{R}}^{(1)} \right]^{(0)}  \right]_{0}^{(0)}$ in terms of another coupling scheme, where $\mathbf{d}_{A}$ couples with $\mathbf{d}_{B}$ and $\mathbf{\hat{R}}$ couples with $\mathbf{\hat{R}}$. 
This involves the definition of 9-$j$ symbols. Based on \autoref{9-j1} and \autoref{9-j2} we have
\begin{equation}
(\mathbf{d}_{A}\cdot\mathbf{\hat{R}})(\mathbf{d}_{B}\cdot\mathbf{\hat{R} })= 3\sum_{k}(2k + 1) 
\left\{
\begin{array}{ccc}
1& 1&k \\
1&1&k \\
0&0&0 
\end{array}
\right\} 
\left[ \left[ \mathbf{d}_{A}^{(1)} \otimes \mathbf{d}_{B}^{(1)} \right]^{(k)}\otimes \left[ \mathbf{\hat{R}}^{(1)} \otimes \mathbf{\hat{R}}^{(1)} \right]^{(k)}  \right]_{0}^{(0)} \ . \label{regrouping}
\end{equation}
The 9-$j$ symbols in the above equation can be expressed in terms of 6-$j$ symbols (related to the coupling of 3 angular momenta). In the special case where the final angular momentum $j_{9}=0$, we have\cite{zare-book}
\begin{eqnarray}
\left\{
\begin{array}{ccc}
j_{1}& j_{2}&j_{3} \\
j_{4}&j_{5}&j_{6} \\
j_{7}&j_{8}&j_{9}
\end{array}
\right\} 
&=&(-1)^{j_{2} + j_{3} + j_{4} + j_{7}}[(2j_{3} +1)(2j_{7} + 1)]^{-\frac{1}{2}} \nonumber \\
&  & \times 
\left\{
\begin{array}{ccc}
j_{1}& j_{2}&j_{3} \\
j_{5}&j_{4}&j_{7} 
\end{array}
\right\} 
\delta_{j_{3}j_{6}}\delta_{j_{7}j_{8}} \ ,
\end{eqnarray}
so that
\begin{eqnarray}
\left\{
\begin{array}{ccc}
1& 1&k \\
1&1&k \\
0&0&0 
\end{array}
\right\} 
&=& (-1)^{k+2} (2k+1)^{-\frac{1}{2}}
\left\{
\begin{array}{ccc}
1& 1&k \\
1&1&0 
\end{array}
\right\} \nonumber \\
&=& (-1)^{k+2} (2k+1)^{-\frac{1}{2}} 
\left\{ 
\begin{array}{cc}
\frac{1}{3} (-1)^{-k} & 0\leq k \leq 2 \\
0 & \mbox{otherwise}
\end{array}
\right. \ . \nonumber \\
\end{eqnarray}
Because $k$ results from the coupling of $j_{1}=1$ and $j_{2}=1$, it is in the range of $|j_{1}-j_{2}|, \cdots, |j_{1}+j_{2}|$, and the above equation can be simplified as
\begin{equation}
\left\{
\begin{array}{ccc}
1& 1&k \\
1&1&k \\
0&0&0 
\end{array}
\right\} = \frac{(2k + 1)^{-\frac{1}{2}}}{3} \ . \label{9-j-value}
\end{equation}  
Substituting Eq. (\ref{9-j-value}) into Eq. (\ref{regrouping}) and using Eq. (\ref{eqn:tensorContraction00}) gives
\begin{eqnarray}
(\mathbf{d}_{A}\cdot\mathbf{\hat{R} })(\mathbf{d}_{B}\cdot  \mathbf{\hat{R} })
= \sum_{k=0,1,2}\sum_{q} (-1)^{k-q} \left[ \mathbf{d}_{A}^{(1)} \otimes \mathbf{d}_{B}^{(1)} \right]^{(k)}_{q} \left[ \mathbf{\hat{R}}^{(1)} \otimes \mathbf{\hat{R}}^{(1)} \right]^{(k)}_{-q} \ . \label{scalarXscalar}
\end{eqnarray}
Since the tensor product of rank 1 is related to the cross product (see \autoref{eqn:crossProduct}), $\left[ \mathbf{\hat{R}}^{(1)} \otimes \mathbf{\hat{R}}^{(1)} \right]^{(k=1)}_{-q}$ is associated with $\mathbf{\hat{R}}\times\mathbf{\hat{R}}=0$ and is zero. So that \autoref{scalarXscalar} simplifies to 
\begin{eqnarray}
(\mathbf{d}_{A}\cdot \mathbf{\hat{R} })(\mathbf{d}_{B}\cdot\mathbf{\hat{R} })&=&\left[ \mathbf{d}_{A}^{(1)} \otimes \mathbf{d}_{B}^{(1)} \right]^{(0)}_{0} \left[ \mathbf{\hat{R}}^{(1)} \otimes \mathbf{\hat{R}}^{(1)} \right]^{(0)}_{0} \nonumber \\
& & + \sum_{|q|=0,1,2} (-1)^{q} \left[ \mathbf{d}_{A}^{(1)} \otimes \mathbf{d}_{B}^{(1)} \right]^{(2)}_{q} \left[ \mathbf{\hat{R}}^{(1)} \otimes \mathbf{\hat{R}}^{(1)} \right]^{(2)}_{-q} \nonumber \\
&=& (-\frac{1}{\sqrt{3}} \mathbf{d}_{A}\cdot \mathbf{d}_{B})(-\frac{1}{\sqrt{3}}\mathbf{\hat{R}}\cdot\mathbf{\hat{R}} ) \nonumber \\
& & +  \sum_{q} (-1)^{q} \left[ \mathbf{d}_{A}^{(1)} \otimes \mathbf{d}_{B}^{(1)} \right]^{(2)}_{q} \left[ \mathbf{\hat{R}}^{(1)} \otimes \mathbf{\hat{R}}^{(1)} \right]^{(2)}_{-q} \nonumber \\
&=& \frac{\mathbf{d}_{A}\cdot \mathbf{d}_{B}}{3}+  \sum_{q} (-1)^{q} \left[ \mathbf{d}_{A}^{(1)} \otimes \mathbf{d}_{B}^{(1)} \right]^{(2)}_{q} \left[ \mathbf{\hat{R}}^{(1)} \otimes \mathbf{\hat{R}}^{(1)} \right]^{(2)}_{-q} \  . \nonumber \\ 
\label{term2}
\end{eqnarray}
Substituting \autoref{term2} into \autoref{dd-operator} yields
\begin{equation}
\hat{V}_{\mbox{\tiny dd}}(\mathbf{R}) = \frac{-3}{R^{3}} \sum_{q=-2}^{2} (-1)^{q} \left[ \mathbf{d}_{A}^{(1)} \otimes \mathbf{d}_{B}^{(1)} \right]^{(2)}_{q} \left[ \mathbf{\hat{R}}^{(1)} \otimes \mathbf{\hat{R}}^{(1)} \right]^{(2)}_{-q} \ .
\end{equation}
In the above equation, $\mathbf{\hat{R}}$ can be viewed as a first-order irreducible spherical tensor, therefore the
contraction of two $\mathbf{\hat{R}}$ gives rise to the second-order  irreducible spherical tensor
\oneline{
 \left[ \mathbf{\hat{R}}^{(1)} \otimes \mathbf{\hat{R}}^{(1)} \right]^{(2)}_{-q} = T^{(2)}_{-q}(\mathbf{\hat{R}}) \ .
}
Based on \autoref{eqn:relationTwithY} we obtain
\oneline{
 \left[ \mathbf{\hat{R}}^{(1)} \otimes \mathbf{\hat{R}}^{(1)} \right]^{(2)}_{-q} = \sqrt{ \frac{8 \pi}{15} } Y_{2 -q}(\theta_{R}, \phi_{R}) \ ,
}
where $\theta_{R}$ and $\phi_{R}$ describe the orientation of $\mathbf{\hat{R}}$ in the space-fixed axis system, and
then the dipole-dipole operator becomes:
\oneline{
\hat{V}_{\mbox{\tiny dd}}(\mathbf{R}) = -2\sqrt{\frac{6\pi}{5}} \frac{1}{R^{3}} \sum_{q=-2}^{2} (-1)^{q}  Y_{2 -q}(\theta_{R}, \phi_{R})\left[ \mathbf{d}_{A}^{(1)} \otimes \mathbf{d}_{B}^{(1)} \right]^{(2)}_{q}  \ . \label{eqn:dipoleOperatorInDD}
}

To calculate the matrix elements of the dipolar operator, we now only need to consider the tensor product 
$\left[ \mathbf{d}_{A}^{(1)} \otimes \mathbf{d}_{B}^{(1)} \right]^{(2)}_{q}$. Using \autoref{eqn:tensorContraction112}, we can express the tensor product in terms of the tensor components in some axis system
\multiline{
\left[ \mathbf{d}_{A}^{(1)} \otimes \mathbf{d}_{B}^{(1)} \right]^{(2)}_{q}  
= \sum_{q_{A} } (-1)^q \sqrt{5} \threejm{1}{q_A }{1}{q_B }{2}{-q} T^{(1)}_{q_{A} } (\mathbf{d}_A )  T^{(1)}_{q_B } (\mathbf{d}_B )
\  .  \label{eqn:tensorComponents}
}
The rotational states $\ket{N_A, M_A}\ket{N_B, M_B}$ are defined in the space-fixed axis system, so only tensor operators 
defined in the same axis system can operate on them directly. Therefore, for the convenience of calculation, the tensor 
components in \autoref{eqn:tensorComponents} should also be in the space-fixed axis system. 
However, the dipole moment of a molecule is
 most conveniently expressed in the body-fixed axis system. In the case of a diatomic polar molecule, we can choose
 the body-fixed $z$-axis to be in the same direction as the dipole moment.  So that the only nonzero body-fixed $b$ 
 component is given by
 \oneline{
 T^{(1)}_{b=0} (\mathbf{d} ) = d \hat{z} \ ,
 }
 where $d$ is the magnitude of the dipole moment. Now the problem is obtaining the space-fixed dipole moment
  component from the body-fixed component. From \autoref{eqn:bTos}, we have
\multiline{
T^{(1)}_{s } (\mathbf{d} ) &=& \sum_{b=-1}^{1} D^{1 *}_{sb} T^{(1)}_{b} (\mathbf{d} ) \nonumber \\
&=& D^{1 *}_{s 0} d \nonumber \\
&=& d_0 C_{1 s} \ ,  \label{eqn:dipoleSpaceFixed}
}
where $T^{(1)}_{s }$ is the tensor component in space-fixed axis system and the last equality comes from \autoref{eqn:relationDandY}. Substituting \autoref{eqn:dipoleSpaceFixed} into  \autoref{eqn:tensorComponents}
then gives 
\multiline{
\left[ \mathbf{d}_{A}^{(1)} \otimes \mathbf{d}_{B}^{(1)} \right]^{(2)}_{s}  
= \sum_{s_{A} } (-1)^s \sqrt{5}d_A d_B \threejm{1}{s_A }{1}{s_B }{2}{-s} C_{1 s_A} C_{1 s_B}
\  .  \label{eqn:dipoleOperatortoC}
}
Using \autoref{eqn:matrixElementModifiedY}, the corresponding matrix elements are given by
\multiline{
&&\bra{N_A, M_A} \matelement{N_B, M_B}{ \left[ \mathbf{d}_{A}^{(1)} \otimes \mathbf{d}_{B}^{(1)} \right]^{(2)}_{s} }{N_A^{\prime}, M_A^{\prime}}\ket{N_B^{\prime}, M_B^{\prime}} \nonumber \\
&&= (-1)^{M_A + M_B} d_A d_B \sqrt{5} \sqrt{(2 N_A + 1)(2 N_B + 1) (2 N_A^{\prime} + 1)  (2 N_B^{\prime} + 1)} \nonumber \\
&&\hspace{0.5cm} \times \threejm{N_A}{0}{1}{0}{N_A^{\prime}}{0} \threejm{N_B}{0}{1}{0}{N_B^{\prime}}{0} \nonumber \\
&&\hspace{0.5cm} \times \sum_{s_A} \threejm{N_A}{-M_A}{1}{s_A}{N_A^{\prime}}{M_A^{\prime}} \threejm{N_B}{-M_B}{1}{s_B}{N_B^{\prime}}{M_B^{\prime}} \threejm{1}{s_A}{1}{s_B}{2}{-s} \ . \nonumber \\
\label{eqn:matrixElementDD}
}


Finally, after considering every possible  value of $q$ in \autoref{eqn:dipoleOperatorInDD} and making use of \autoref{eqn:dipoleOperatorInDD}, we arrive at 
\multiline{
&&\bra{N_A , M_A} \bra{N_B , M_B} \hat{V}_{\rm dd}(\mathbf{R}) \ket{N_A^{'}, M_A^{'}} \ket{N_B^{'} , M_B^{'} } \nonumber \\
&&= - \frac{F}{2 R^3} \bigg\{ 3\sin^2\theta_{R} e^{-2 i \phi_{R} } D^{A}_{-} D^{B}_{-} - 6\sin\theta_{R}\cos\theta_{R} e^{- i\phi_{R}} \left[ D_0^A D_{-}^B + D_{-}^A D_0^B\right] \nonumber \\
&& \hspace{1.5cm} + 3\sin^2\theta_{R} e^{2 i \phi_{R} } D^{A}_{+} D^{B}_{+} + 6\sin\theta_{R}\cos\theta_{R} e^{ i\phi_{R}} \left[ D_0^A D_{+}^B + D_{+}^A D_0^B\right] \nonumber \\
&& \hspace{1.5cm} + \sqrt{6} (3\cos^2\theta_{R}-1) \left[ D_{+}^A D_{-}^B + D_{-}^A D_{+}^B + D_0^A D_0^B \right] \bigg\} \ ,
}
where
\multiline{
&& F =  (-1)^{M_A + M_B} d_A d_B \sqrt{5} \sqrt{(2 N_A + 1)(2 N_B + 1) (2 N_A^{\prime} + 1)  (2 N_B^{\prime} + 1)} \nonumber \\
&&\hspace{1cm} \times \threejm{N_A}{0}{1}{0}{N_A^{\prime}}{0} \threejm{N_B}{0}{1}{0}{N_B^{\prime}}{0} \ ,
}
and $D^{A}_{\pm, 0}D^{B}_{\pm, 0}$ are defined using
\multiline{
f_{AB}(s_A, s_B) = \threejm{N_A}{-M_A}{1}{s_A}{N_A^{\prime}}{M_A^{\prime}} \threejm{N_B}{-M_B}{1}{s_B}{N_B^{\prime}}{M_B^{\prime}} \threejm{1}{s_A}{1}{s_B}{2}{s} \ , \nonumber \\ \label{eqn:fAB}
} in the following way
\multiline{
D^{A}_{+}D^{B}_{+} &=& f_{AB}(1, 1) \ , \nonumber \\
D^{A}_{+}D^{B}_{0} &=& f_{AB}(1, 0) \ , \nonumber \\
D^{A}_{+}D^{B}_{-} &=& f_{AB}(1, -1) \ , \nonumber \\
D^{A}_{0}D^{B}_{-} &=& f_{AB}(1, -1) \ , \nonumber \\
D^{A}_{0}D^{B}_{+} &=& f_{AB}(0, 1) \ , \nonumber \\
D^{A}_{0}D^{B}_{0} &=& f_{AB}(0, 0) \ , \nonumber \\
D^{A}_{0}D^{B}_{-} &=& f_{AB}(0, -1) \ , \nonumber \\
D^{A}_{-}D^{B}_{+} &=& f_{AB}(-1, 1) \ , \nonumber \\
D^{A}_{-}D^{B}_{0} &=& f_{AB}(-1, 0) \ , \nonumber \\
D^{A}_{-}D^{B}_{-} &=& f_{AB}(-1, -1) \ . 
}
The factor $F$ describes the selection rule of rotational quantum number $N$ for the dipole-dipole interaction, which 
says that  only two consecutive rotational levels are coupled. $D^{A}_{\pm, 0}D^{B}_{\pm, 0}$ describes the change of
the projection quantum number $M$ by the dipole-dipole interaction. It can be easily seen from \autoref{eqn:fAB}
that $D^{A}_{X}D^{B}_{Y}$ is nonzero only if the projection quantum numbers change in a certain way. For example, 
$D^{A}_{+}D^{B}_{+}$ is nonzero if only $M_A - M_A^{\prime} = 1$ and $M_B - M_B^{\prime} = 1$.  

\section{Introduction to excitons in molecular crystals}
\label{sec: exciton}
In this section, we aim to give a very brief introduction to excitons. For simplicity, we consider a 1D molecular 
crystal with every lattice site occupied by one molecule. The Hamiltonian is given by
\oneline{
H = \sum_{n} H_n + \frac{1}{2} \sum_{n}\sum_{m\neq n} V_{n m} \ , \label{eqn:hamMolecularCrystal}
}
where $n$ is the molecule index, $H_{n}$ is the Hamiltonian for an isolated molecule $n$, and $V_{nm}$ describes 
the interaction between the two molecules labelled $n$ and $m$. The extension to 2D and 3D molecular arrays is straightforward:
one only needs to replace the number index $n$ by a vector index $\mathbf{n}$ in the 2D or 3D dimensional space.   

\subsection{Commutation relation}
\label{sec:commutationRelation}
 
In a molecular crystal, there are a large number of identical interacting molecules. To study the excited states of the 
system, 
it is suitable to use the second-quantization formalism which is convenient for describing many-particle systems. 


The second-quantization representation uses a basis that describes the number of particles occupying each state in a 
complete orthonormal set of single-particle states. Assuming the distance between any two molecules is large such 
that their wavefunctions hardly overlap, we can choose the single-particle states  to be the eigenfunctions $\psi^f_n$ 
of the isolated molecule Hamiltonian $H_n$.  The basis set for the second-quantization formalism is  then
given by the wavefunction that describes a crystal formed by non-interacting molecules, that is
\begin{equation}
\left. |N_{1f_{1}} N_{2f_{2}} ...N_{nf} ... \right\rangle \nonumber \ , 
\end{equation}
where $N_{nf}$ represents the occupation number of state $f$ in molecule $n$ and it is either 0 or 1. Accordingly, 
the occupation number operator is defined as
\begin{equation}
\hat{N}_{nf} |...N_{nf}...\rangle = N_{nf} |...N_{nf}...\rangle \ .
\end{equation}
For the crystal formed by interacting molecules, its eigenstates can be expressed in terms of linear combinations
of the above basis set. 


The molecule at site $n$ must be in some state, therefore the occupation operators of all states sum up to the identity
operator:
\begin{equation}
\sum_{f}\hat{N}_{nf} = 1 \ , \label{must-be-in-a-state}
\end{equation}
where $f$ denotes any (ground or excited) states  of molecule $n$. 
It is convenient to introduce two operators $b_{nf}^{\dagger}$ and $b_{nf}$ associated with molecule $n$ and state $f$ such that
\begin{eqnarray}
b_{nf}^{\dagger} |...N_{nf}...\rangle &=& (1-N_{nf})|...(N_{nf}+1)...\rangle \ , \nonumber \\
b_{nf} |...N_{nf}...\rangle &=& N_{nf}|...(N_{nf}-1)...\rangle \ , \label{single-molecule-operator}
\end{eqnarray}
and express the occupation number operator as the product of these two operators,
\begin{equation}
\hat{N}_{nf} =b_{nf}^{\dagger} b_{nf} \ .
\end{equation}
The physical meaning of the two operators is clear: $b_{nf}^{\dagger}$ creates the state $f$ in molecule $n$ and 
$b_{nf}$ destroys the state $f$ in molecule $n$.  It follows from Eq. (\ref{single-molecule-operator}) that the 
commutation relation of the $b$ operators for the same energy level of the same molecule is given by
\begin{eqnarray}
b_{nf}b_{nf}^{\dagger} &+& b_{nf}^{\dagger}b_{nf} = 1 \ , \nonumber \\
b_{nf}b_{nf}&=&b_{nf}^{\dagger}b_{nf}^{\dagger} =0 \ . \label{commutation-relation-single-molecule}
\end{eqnarray}
Because an operator for a specific molecule $n$ and state $f$ does not operate on other molecules and other states, any two operators that correspond to different molecules $n$ and $m$ or different states $f$ and $f'$ commute.

The $b$ operators defined above are not very convenient as they are associated with both the ground and excited states. Since 
we want to focus on the excited states, we define the so-called excitation creation and annihilation operators
as  
\begin{equation}
P_{ne}^{\dagger} = b_{ne}^{\dagger} b_{ng} \ , \;\;P_{ne}=b_{ng}^{\dagger} b_{ne} \ , \label{creation&annihilation}
\end{equation}
where $g$ denotes the ground state and $e$ represents any excited state. 
Based on the physical meaning of $b_{nf}^{\dagger}$ and $b_{nf}$, it is clear that $P_{ne}^{\dagger}$ creates a 
excited state from the vacuum state $\ket{0}$ and $P_{ne}$ does the reverse.
Making use of \autoref{commutation-relation-single-molecule}, we obtain from the above definitions that
\begin{eqnarray}
P_{ne}^{\dagger} P_{ne} &=& b_{ne}^{\dagger} b_{ng}b_{ng}^{\dagger} b_{ne} \nonumber \\
                                      &=& b_{ne}^{\dagger}(1-b_{ng}^{\dagger}b_{ng})b_{ne} \nonumber \\
                                      &=& \hat{N}_{ne} \ , \label{creation-annihilation}
\end{eqnarray}
\begin{eqnarray}
P_{ne} P_{ne}^{\dagger} &=& b_{ng}^{\dagger} b_{ne} b_{ne}^{\dagger} b_{ng} \nonumber \\
                                      &=& b_{ng}^{\dagger}(1-b_{ne}^{\dagger}b_{ne})b_{ng} \nonumber \\
                                      &=& \hat{N}_{ng} \ . \label{annihilation-creation}
\end{eqnarray}
Based on \autoref{must-be-in-a-state}, the subtraction and addition of \autoref{creation-annihilation} and \autoref{annihilation-creation} yield:
\begin{equation}
P_{ne} P_{ne}^{\dagger} - P_{ne}^{\dagger} P_{ne} = 1 - \sum_{f\neq g} \hat{N}_{nf}  -\hat{N}_{ne}\ , \label{exact-relation}
\end{equation}
\begin{equation}
P_{ne} P_{ne}^{\dagger} + P_{ne}^{\dagger} P_{ne} = 1  - \sum_{f\neq g} \hat{N}_{nf}  + \hat{N}_{ne} \ . \label{eqn:addition}
\end{equation}
Because $P_{nf}$ and $P_{nf}^{\dagger}$ don't operate on a different
molecule $n'$, any exciton operators corresponding to different molecules or different states
commute with each other, so that \autoref{exact-relation} can be extended as 
\begin{equation}
P_{ne}P_{n'e'}^{\dagger}-P_{n' e'}^{\dagger}P_{n e}  =  \delta_{nn'}\delta_{ee'}\left(1-\sum_{f\neq g}\hat{N}_{nf} -\hat{N}_{ne} \right)  \ . \label{exact-relation2}
\end{equation}
\Autoref{eqn:addition} and \autoref{exact-relation2} describe the exact
statistics of excitons.  Unfortunately, it is cumbersome to take into account  the exact
statistics in most cases. Usually, only a small number of molecules in a crystal is excited, so the following inequality
\begin{equation}
\left\langle N_{ne} \right\rangle \ll 1 \label{condition-for-Bose-approx}
\end{equation}
holds for any excited state $e$. This means we can ignore the operators $\hat{N}_{ne}$ in practice. However, we cannot
ignore all $\hat{N}_{ne}$ in both \autoref{exact-relation} and \autoref{eqn:addition} as it will lead to contradictory 
results, that is  
\begin{equation}
P_{ne} P_{ne}^{\dagger} - P_{ne}^{\dagger} P_{ne} = 1\ , \label{eqn:subtractionApproximate}
\end{equation}
which describes two bosons at the same site, 
and 
\begin{equation}
P_{ne} P_{ne}^{\dagger} + P_{ne}^{\dagger} P_{ne} = 1 \ , \label{eqn:additionApproximate}
\end{equation}
which describes two fermions at the same site. 
The above two equations cannot be satisfied at the same time, so we have to choose either one of them. Considering the 
fact that two excitations cannot reside at the same site as a molecule cannot be excited twice, 
it seems to more reasonable to choose \autoref{eqn:additionApproximate} to be valid. In addition, 
in the Heitler-London approximation where only the ground and first excited states are considered, 
\autoref{eqn:additionApproximate} becomes exact as $ \hat{N}_{ng}  + \hat{N}_{ne}=1$. As the Heitler-London 
approximation is usually a good approximation, it also makes sense to choose \autoref{eqn:additionApproximate}. 
Therefore, we usually rewrite the commutation relations for excitons as 
\oneline{
P_{n'e'}P_{ne}^{\dagger} - P^{\dagger}_{n'e'}  P_{ne} =0 \quad\mbox{for $n \neq n'$ or $e \neq e'$} \ ,  \label{eqn:bosonCommutation}
}
\oneline{
P_{ne} P_{ne}^{\dagger} + P_{ne}^{\dagger} P_{ne} = 1 \ . \label{eqn:fermionCommutation}
}
This is called the Pauli commutation relation as \autoref{eqn:bosonCommutation} looks like the commutation relation for 
bosons and \autoref{eqn:fermionCommutation} looks like the commutation relation for fermions. For the convience
of calculations, the commutation relation is rewritten as
\multiline{
&&\{ P_{ne} , P^{\dagger}_{me} \}= \delta_{m,n} + (1-\delta_{m,n})2 P^{\dagger}_{me} P_{ne} \ , \nonumber \\
&&[P_{ne}, P_{me}] = [P^{\dagger}_{ne}, P^{\dagger}_{me}] = 0 \ , \nonumber \\
&&[P_{ne}, P_{ne'}] = [P^{\dagger}_{ne}, P^{\dagger}_{ne'}] = 0 \ , \nonumber \\
&&P_{ne} P_{ne} = P^{\dagger}_{ne} P^{\dagger}_{ne} = 0 \ , \label{eqn:excitonCommutationRelation}
}
where $n$ and $m$, $e$ and $e'$ are assumed to be different. 

\subsection{Exciton Hamiltonian in second quantization}
\label{sec:excitonHam}

The Hamiltonian in \autoref{eqn:hamMolecularCrystal} can be rewritten in terms of the exciton creation and 
annihilation operators. Here, we first derive the Hamiltonian in the two-level approximation and then give
the Hamiltonian for the most general case.

In the first step, we express the Hamiltonian in terms of the $b$ operators. The result is\cite{zare-book}
\multiline{
H = \sum_{i}\varepsilon_{i} b_{i f}^{\dagger} b_{i f} + \frac{1}{2} \sum_{i, j\neq i} \sum_{f', f, l', l} b_{if'}^{\dagger} b_{j l'}^{\dagger} b_{i f} b_{j l} \matelement{f' l'}{V_{ij}}{f l} \ , \label{eqn:hamInb}
}
where $i$ and $j$ are site indices, $f$, $f'$, $l$ and $l'$ represent any molecular state, and $\matelement{f' l'}{V_{ij}}{f l}$ is a shorthand for $\langle f' |_{i} \langle l' |_{i}  V_{ij} | f\rangle_{i}  | l\rangle_{j}$. 

Now we consider the two-level approximation where each molecule has only two energy levels: the ground 
state $g$ and one excited state $e$ such that $f$, $f'$, $l$ and $l'$ in \autoref{eqn:hamInb} can only take $g$ or $e$. 
Making use of the definitions of exciton creation and annihilation operators(\autoref{creation&annihilation}), we can easily show that 
\autoref{eqn:hamInb} can be written as
\multiline{
&& H = \sum_{i} \left( \varepsilon_{g} \bod{ig}\bo{ig} + \varepsilon_{e} \bod{ie}\bo{ie} \right) + \sum_{i, j\neq i} \bigg[ \frac{1}{2} \bod{ig}\bo{ig}\bod{jg}\bo{jg} \matelement{gg}{V_{ij} }{gg}    \nonumber \\
&& \hspace{1cm} + \left(P^{\dagger}_{ie} + P_{ie} \right)\bod{jg}\bo{jg} \matelement{eg}{V_{ij} }{gg} + P^{\dagger}_{ie} P^{\dagger}_{je}  \matelement{ee}{V_{ij} }{gg} \nonumber \\
&& \hspace{1cm} + \bod{ie}\bo{ie}\bod{jg}\bo{jg} \matelement{eg}{V_{ij} }{eg} + P^{\dagger}_{ie} P_{je}  \matelement{eg}{V_{ij} }{ge} \nonumber \\
&& \hspace{1cm} + \bod{ie}\bo{ie} \left( P^{\dagger}_{je} + P_{je}\right) \matelement{eg}{V_{ij} }{ee} \nonumber \\
&&  \hspace{1cm} + \frac{1}{2} \bod{i e}\bo{i e} \bod{j e} \bo{j e} \matelement{ee}{V_{ij} }{ee} \bigg ]\ .
}
There are terms like $b^{\dagger}_{ig} b_{ig}$ and $b^{\dagger}_{ie} b_{ie}$ in the above equation. We want to get
rid of those terms. Noticing that 
\oneline{
b^{\dagger}_{ie} b_{ie} = P^{\dagger}_{ie} P_{ie} = \hat{N}_{ie} \ ,
}
and 
\oneline{
b^{\dagger}_{ig} b_{ig} = \hat{N}_{ig} = 1 - \hat{N}_{ie} = 1 -  P^{\dagger}_{ie} P_{ie} \ ,
}
we can rewrite the Hamiltonian as
\multiline{
H = H_0 + H_1 + H_2 + H_3 + H_4 + H_5 + H_6 \ , \label{eqn:hamSplited}
}
where 
\multiline{
H_0 = \varepsilon_{g} N_{\rm mol} + \frac{1}{2} \sum_{i, j\neq i} \matelement{gg}{V_{ij} }{gg} \ ,
}
\multiline{
H_1 = \sum_{i} \left\{ (\varepsilon_{e} - \varepsilon_{g}) + \sum_{j\neq i}\bigg[ \matelement{eg}{V_{ij} }{eg} - \matelement{gg}{V_{ij} }{gg} \bigg] \right\} P^{\dagger}_{ie} P_{ie} \ , \label{eqn:hamH1}
}
\multiline{
H_2 = \sum_{i, j\neq i} \matelement{eg}{V_{ij} }{ge} P^{\dagger}_{ie} P_{je} \ , \label{eqn:hamH2}
}
\multiline{
H_3 = \frac{1}{2} \sum_{i, j\neq i} \matelement{eg}{V_{ij} }{ge} \left( P^{\dagger}_{ie} P^{\dagger}_{je} +  P_{ie} P_{je} \right) \ , \label{eqn:hamH3}
}
\multiline{
H_4 = \frac{1}{2} \sum_{i, j\neq i} \bigg[ \matelement{ee}{V_{ij} }{ee} + \matelement{gg}{V_{ij} }{gg} - 2 \matelement{eg}{V_{ij} }{eg}\bigg]  P^{\dagger}_{ie} P_{ie} P^{\dagger}_{je}  P_{je}  \ , \label{eqn:hamH4}
}
\multiline{
H_5 =  \sum_{i, j\neq i} \matelement{eg}{V_{ij} }{gg} \left( P^{\dagger}_{ie} +  P_{ie} \right) \ , \label{eqn:hamH5}
}
\multiline{
H_6 = \sum_{i, j\neq i} \bigg[ \matelement{eg}{V_{ij} }{ee} - \matelement{eg}{V_{ij} }{gg}\bigg]  \left(P^{\dagger}_{ie} + P_{ie} \right) P^{\dagger}_{je}  P_{je}  \ . \label{eqn:hamH6}
}

For the general case with multiple energy levels, the derivation of the Hamiltonian in terms of exciton operators is
very similar but tedious. So I choose to use the Mathematica software\cite{mathematica} to write a symbolic program to handle the derivations. For 
reference, the result is given below. The Hamiltonian can be divided as 
\oneline{
H = H^{(0)} + H^{(1)} + H^{(2)} + H^{(3)} + H^{(4)}\ , \label{eqn:fullHam}
}
where the superscripts ``$(n)$'' mean that the corresponding term contains $n$ exciton operators. In the following, we 
use captical characters $L$, $L'$, $M$ and $M'$ to represent any excited state of molecule, and $G$ to denote the 
ground state. The first two term can be expressed as
\multiline{
H^{(0)} = \varepsilon_{G} N_{\rm mol} + \frac{1}{2} \sum_{i, j\neq i} \matelement{GG}{V_{ij} }{GG} \ ,
}
%$H^{(1)}$ is given by
\multiline{
H^{(1)} = \sum_{i, j \neq i} \sum_{M} \left[ \matelement{GG}{V_{ij}}{GM} P_{jM} + \matelement{GM}{V_{ij}}{GG} P^{\dagger}_{jM} \right] \ ,
}
neither of which doesn't conserve particle numbers. 
$H^{(2)}$ can be further divided into a part that conserves particle numbers and a part that doesn't, that is
\multiline{
H^{(2)} = H^{(2)}_{\rm conserving} + H^{(2)}_{\rm non-conserving} \ ,
}
where
\multiline{
&& H^{(2)}_{\rm conserving} = \sum_{i} \sum_{M} \left\{ (\varepsilon_{M} - \varepsilon_{G}) + \sum_{j\neq i}\bigg[ \matelement{MG}{V_{ij} }{MG} - \matelement{GG}{V_{ij} }{GG} \bigg] \right\} P^{\dagger}_{iM} P_{iM} \nonumber \\
&&\hspace{2cm} + \sum_{i, j\neq i}\sum_{L, M}  \matelement{GL}{V_{ij}}{MG} P_{iM}P^{\dagger}_{jL} \nonumber \\
&&\hspace{2cm} + \sum_{i, j\neq i} \sum_{L, M} (1 - \delta_{M, L}) \matelement{LG}{V_{ij}}{MG} P^{\dagger}_{iL} P_{iM} \ ,
}
\multiline{
&&H^{(2)}_{\rm non-conserving} = \frac{1}{2}\sum_{i, j\neq i} \sum_{L, M} \bigg[ \matelement{GG}{V_{ij}}{ML} P_{iM} P_{jL} +  \matelement{ML}{V_{ij}}{GG} P^{\dagger}_{iM} P^{\dagger}_{jL} \bigg ] \ . \nonumber \\
}
$H^{(3)}$ doesn't conserve particle numbers and  is given by
\multiline{
&& H^{(3)} = \sum_{i, j\neq i} \sum_{M, M', L}\Bigg\{ \bigg[ \matelement{MG}{V_{ij}}{M' L} - \delta_{M, M'} \matelement{GG}{V_{ij}}{GL} \bigg] P^{\dagger}_{iM} P_{i M'} P_{j L}  \nonumber \\
&& \hspace{1.5cm} +  \bigg[ \matelement{ML}{V_{ij}}{M' G} - \delta_{M, M'} \matelement{GL}{V_{ij}}{GG} \bigg] P^{\dagger}_{iM} P_{i M'} P_{j L} \Bigg\} \ .
}
$H^{(4)}$ conserves particle numbers and is given by
\multiline{
&& H^{(4)} =\frac{1}{2} \sum_{i, j\neq i} \sum_{M, M', L, L'} \bigg[ \delta_{M, M'} \delta_{L, L'} \matelement{GG}{V_{ij}}{GG} + \matelement{ML}{V_{ij}}{M' L'} \nonumber \\
&& \hspace{2cm} - 2\delta_{L, L'} \matelement{MG}{V_{ij}}{M' G} \bigg] P^{\dagger}_{iM} P_{i M'} P^{\dagger}_{j L} P_{j L'} \ . 
}


\subsection{Eigenstates of the exciton Hamiltonian in the Heitler-London approximation}
\label{sec:eigenstateExcitonHam}

The Heitler-London approximation was first used by Frenkel  in his study \cite{Frenkel1, Frenkel2} of
 electronic excitations in molecular crystals. Considering only two levels of a molecule, the ground state $g$ and 
excited state $e$, the approximation assumes the following:
\begin{itemize}
\item the ground-state wavefunction of the crystal is the product of the ground-state wavefunction of individual molecules that constitute the crystal
% 
\item an excited state of the crystal is a superposition of products $ \cdots \ket{g}_i \ket{e}_j\ket{g}_k\cdots$ in which
only one molecule $j$ is excited and all other molecules are in the ground state
\end{itemize}
From the above two statements, we see that the  Heitler-London approximation is expected to be good when the 
intermolecular interaction is weak and the molecules in the crystal preserve a large part of their individuality. 

We now consider the lowest excited state of a crystal, which corresponds to the basis set $\{ \; \cdots \ket{g}_i \ket{e}_j\ket{g}_k\cdots \; \}$  with only one molecule excited. In this case, the number of excitations is conserved, so
that only the parts of the Hamiltonian in \autoref{eqn:hamSplited} that conserve particle numbers need to be included, and we can rewrite   the 
Hamiltonian as
\multiline{
H = H_0 + H_1 + H_2 + H_4 \ . \label{eqn:hamConserving}
}


To understand the problem better, we analyze this Hamiltonian. 
$H_0$ is a constant and thus we can set it as zero energy and ignore it. $H_1$ can be written as 
\multiline{
H_1 = \sum_{i} \left( \Delta \varepsilon_e  +  \mathscr{D}_e \right) P^{\dagger}_{ie} P_{ie} \ ,
}
where  
\oneline{
\Delta \varepsilon_e = \varepsilon_e - \varepsilon_g
}
 is the excitation energy of an isolated molecule, and
\multiline{
\mathscr{D}_e = \sum_{j\neq i}\bigg[ \matelement{eg}{V_{ij} }{eg} - \matelement{gg}{V_{ij} }{gg} \bigg] 
}
is the difference of the two interactions: one for the excited molecule with all remaining ground-state  
molecules  in the crystal, and the other  for  the same nonexcited molecule with all remaining ground-state 
molecules. $\mathscr{D}_e$ is also called the gas-condensed matter shift and it exists when the molecules in the crystal
are interacting with each other.  $H_2$,  as given by
\oneline{
H_2 = \sum_{i, j\neq i} \matelement{eg}{V_{ij} }{ge} P^{\dagger}_{ie} P_{je} \ ,
}
destroys an excitation in molecule $j$ and then creates an excitation in molecule $i$. Therefore, it describes the 
propagation of an excitation in the crystal and we call it the hopping term. The strength of the hopping interaction 
is given by the matrix elements $\matelement{eg}{V_{ij} }{ge}$. Usually, the interaction between two nearest 
molecules is strongest and it decays fast with respect to the separation of two molecules, so sometimes it is justified
to consider only the interaction between two nearest neighbors. We call this the nearest-neighbor approximation.  
$H_4$ is the so-called the dynamic interaction, which describes the interaction between two excitations. It can be written as
\multiline{
H_4 = \frac{1}{2} \sum_{i, j\neq i} \Phi_{ij} P^{\dagger}_{ie} P_{ie} P^{\dagger}_{je}  P_{je} \ ,
}
and its strength
is determined by the matrix elements
\[
\Phi_{ij}  = \matelement{ee}{V_{ij} }{ee} + \matelement{gg}{V_{ij} }{gg} - 2 \matelement{eg}{V_{ij} }{eg} \ .
\]
In molecular crystals, the dipole-dipole interaction is an important kind of interaction $V_{ij}$ between molecules. For 
molecules with a center of inversion, their eigenstates (for example $\ket{e}$ and $\ket{g}$) have well-defined
parity, and the above matrix elements are zero. Under these circumstance, the dynamic interaction is zero and $H_4$ can 
be ignored. In the case studied in this thesis, when we are concerned with the one-excitation subspace, we can safely ignore the 
dynamic interaction because it requires two excitations in the crystal. From another perspective, we can ignore the 
dynamic interaction because its matrix elements in the one-excitation subspace vanish. For illustration purposes, we give the derivation here. Based on \autoref{eqn:excitonCommutationRelation}, it can be shown that the
matrix element of $H_4$ is given by
\multiline{
&&\matelement{e_n, g_m } { \frac{1}{2} \sum_{i, j\neq i} \Phi_{ij} P^{\dagger}_{ie} P_{ie} P^{\dagger}_{je}  P_{je} } {g_n, e_m} \nonumber \\
&&= \frac{1}{2} \sum_{i, j\neq i} \Phi_{ij} \matelement{0 } { P_{ne} P^{\dagger}_{ie} P_{ie} P^{\dagger}_{je}  P_{je} P^{\dagger}_{me} } {0} \nonumber \\
&&= \frac{1}{2} \sum_{i, j\neq i} \Phi_{ij} \matelement{0 } {\left[ \delta_{i, n} + (1-2\delta_{i, n}) P^{\dagger}_{ie} P_{ne} \right] P^{\dagger}_{je} P_{ie}\left[ \delta_{j, m} + (1-2\delta_{j, m}) P^{\dagger}_{je} P_{me} \right]} {0 } \ . \nonumber \\
}
Without proceeding further, we can easily see that
the above equation gives  0 as it consists of terms in which either $P^{\dagger}$ operates on $\bra{0}$ or $P$
operates on $\ket{0}$. 


The above analysis shows that \autoref{eqn:fullHam} can be simplifed to
\oneline{
H =  \sum_{n} \left( \Delta \varepsilon_e  +  \mathscr{D}_e \right) P^{\dagger}_{ne} P_{ne} + \sum_{n, m\neq n} \matelement{eg}{V_{n m} }{ge} P^{\dagger}_{ne} P_{me} \ , \label{eqn:hamHeitler-London}
}
if one is only interested in the lowest-energy excited state of the crystal. 
\Autoref{eqn:hamHeitler-London} gives the exciton Hamiltonian in the Heitler-London approximation. The corresponding 
eigenstates are called Frenkel excitons, which are many-body excited states of the whole crystal.
 
The Hamiltonian in the Heitler-London approximation can be diagonalized analytically. Before showing this, we first introduce the concepts of site representation and quasimomentum representation. 
Previously,  the excitation creation and annihilation operators $P_{nf}$ and $P_{nf}^{\dagger}$ were defined with respect 
to the lattice site $n$. 
These operators form the site representation. Due to the periodicity of the crystal, we can use the quasimomentum representation, to represent excitations by taking the Fourier transforms of $P_{nf}$ and $P_{nf}^{\dagger}$, that is
\begin{eqnarray}
P_{f}(k)  &=& \frac{1}{\sqrt{N}} \sum_{n}  e^{-ikn}  P_{nf} \ , \nonumber \\
P_{f}^{\dagger}(k)  &=& \frac{1}{\sqrt{N}} \sum_{n}  e^{ikn} P_{nf}^{\dagger} \ , \label{eqn:KInTermsOfN}
\end{eqnarray}
where $N$ is the number of molecules in the crystal. Accordingly, the inverse Fourier transforms are
\begin{eqnarray}
P_{nf}   &=& \frac{1}{\sqrt{N}} \sum_{k}  e^{ikn} P_{f}(k)  \ , \nonumber \\
P_{nf}^{\dagger}   &=& \frac{1}{\sqrt{N}} \sum_{k}  e^{-ikn} P_{f}^{\dagger}(k) \ . \label{eqn:NInTermsOfK}
\end{eqnarray}
The conversion between the site representation and the quasimomentum representation usually involves the 
 equalities
\multiline{
\frac{1}{N} \sum_{k} e ^{ i k (n - m)} &=& \delta_{n, m} \ , \nonumber \\
\frac{1}{N} \sum_{n} e ^{ i n (k - k' )} &=& \delta_{k, k'} \ . \label{eqn:orthonormalityCondition}
}
These two equalities are very important and we will use them frequently in later derivations. 


%It is useful to explain the wavevector $k$ in the quasimomentum representation. 

Substituting \autoref{eqn:NInTermsOfK} into \autoref{eqn:hamHeitler-London} gives
\multiline{
&& H  = \sum_{n} \left( \Delta \varepsilon_e  +  \mathscr{D}_e \right) \frac{1}{N} \sum_{k} e^{-ikn} P_{f}^{\dagger}(k)  \sum_{k'} e^{ik'n} P_{f}(k') \nonumber \\
&&\hspace{1cm} + \sum_{n}\sum_{ m\neq n} \matelement{eg}{V_{n m} }{ge} \frac{1}{N} \sum_{k} e^{-ik m} P_{f}^{\dagger}(k)  \sum_{k'} e^{ik'n} P_{f}(k') \nonumber \\
&&\hspace{0.5cm}= \sum_{k, k'} \left( \frac{1}{N} \sum_{n} e^{i n (k' - k )}\right) P_{f}^{\dagger}(k) P_{f}(k') \nonumber \\
&&\hspace{1cm} + \sum_{n} \sum_{ l = n - m } \matelement{eg}{V_{n, n-l} }{ge} e^{i k l} \left( \frac{1}{N}  e^{i n (k' - k )}\right) P_{f}^{\dagger}(k) P_{f}(k') \ . \nonumber \\
}
Because of the periodicity of the crystal, the interaction $V_{n, m}$ only depends on the separation of the two molecules
at site $n$ and $m$. Thus  $\matelement{eg}{V_{n, n-l} }{ge} $ in the above equation is independent of the index $n$, so 
we rewrite it as $\matelement{eg}{V(l) }{ge}$. After that, making use of \autoref{eqn:orthonormalityCondition}, we can easily show
that the diagonal form of the Hamiltonian is
\oneline{
H = \sum_{k} E_f(k) P_{f}^{\dagger}(k) P_{f}(k) \ ,
}
where $E_f(k) $ is the eigenenergy of the system and is given by
\oneline{
E_f(k) = \Delta \varepsilon_e  +  \mathscr{D}_e + \sum_{l} \matelement{eg}{V(l) }{ge} e^{i k l} \ , \label{eqn:EkDispersion}
}
and thus the eigenstates of the Hamiltonian are
\oneline{
\ket{k} = P_{f}^{\dagger} (k) \ket{0} = \frac{1}{\sqrt{N} } \sum_{n} e^{i k n} \ket{n}  \ , 
}
where $\ket{0}$ is the vacuum state and $\ket{n}$ represents the single-excitation state in which molecule $n$ is in the 
excited state and all other molecules are in the ground state. The state $\ket{k}$ is a Frenkel exciton with 
quasimomentum (or wavevector) $k$. 


%Since the unitary transformation does not change the commutation rule of operators, we expect that the following equation
%\begin{equation}
%P_{f',k'}P_{f,k}^{\dagger} - P_{f,k}^{\dagger}P_{f',k'}=\delta_{kk'}\delta_{ff'} \ . \label{approx-relation-in-wavevector}
%\end{equation}
%also hold. This can be verified by substituting Eq. (\ref{unitary-transform}) into Eq. (\ref{approx-relation}). 







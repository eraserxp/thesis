
\chapter{Background material}
\label{ch:background}
This chapter  briefly outlines the background knowledge that is directly related to the research in later chapters.


\section{Review of angular momentum theory}
\label{sec:angularMomentum}

In this section, we briefly review the theory of angular momentum. Instead of giving a comprehensive 
picture as various books\cite{edmonds-book, rose-book, brink-book, zare-book, silver-book, kleinman-book, Varshalovich-book, sakurai-book, RotSpect} have already done, we take a pragmatic approach that is more concerned about intuitive 
understanding, discussing only the concepts and ideas that are relevant to research in this thesis. 

\subsection{Symmetry group}  
\label{sec:groupTheory}

Symmetry is very important in that the laws of physics are easier to understand if the underlying symmetry are 
appreciated. The theory of angular momentum is essentially the study of symmetry under rotations in quantum mechanics. 
As group theory is an important tool that is used to determine symmetry, we introduce the concepts of group
 in this subsection.

A set of operations $\{ A, B, C, \cdots \}$ is called a group if it satisfies the following properties:
\begin{itemize}
\item there is an identity operation $I$ in the set such that $A I = A$
\item  each operation $A$ has an inverse operation $A^{-1}$ in the same set such that $A A^{-1} = I$
\item  the multiplication of two operations is also a operation in the same set
\item  operations are associative such that $(AB)C = A(BC)$
\end{itemize}
%To simplify the expression of all the symmetry operations in a group, the concept of class is introduced. A class of a 
%group is defined as all the elements in the group that are conjugated to each other. By ``conjugated'', we means that 
% any two elements $A$ and $B$ in a class satisfy
%\oneline{
%X^{-1} A X = B
%}
%for any element $X$ in the group. A group can be divided into classes and then we can work with classes rather than 
%all the symmetry operations in a group. 

\subsection{Rotation operators and spherical harmonics}

As the theory of angular
 momentum concerns about the symmetry under rotations, we start by defining the rotation operator in quantum 
 mechanics. Given a rotation $R$, we associate a rotation operator 
 $D(R)$ that transform a state from the original system to the rotated system:
 \oneline{
 \ket{\psi}_{R} = D(R) \ket{\psi} \ .
 }
To construct the rotation operator, we denote the rotation operator $D(\mathbf{\hat{n}}, d\phi)$ for infinitesimal 
rotation through angle $d\phi$ about an axis $\mathbf{\hat{n}}$. Because $D(\mathbf{\hat{n}}, d\phi)$ must be linear
in $d\phi$ for small rotations and becomes identity operator when $d\phi = 0$, it is useful to define the angular
 momentum operator $J_{\mathbf{\hat{n}}}$ about the axis $\mathbf{\hat{n}}$ as 
 \oneline{
 J_{\mathbf{\hat{n}}} \equiv i \hbar\lim_{\phi \rightarrow 0} \frac{D(\mathbf{\hat{n}}, \phi) - 1}{\phi} \ . \label{eqn:Jdef} \ ,
 }
such that
\oneline{
D(\mathbf{\hat{n}}, d\phi) = 1 - i\left(\frac{ J_{\mathbf{\hat{n}}}  }{\hbar} \right) d\phi \ . \label{eqn:rotationOperator}
}
From the above equation, we can regard $J_{\mathbf{\hat{n}}}$ as the generator of rotation in the sense that $ J_{\mathbf{\hat{n}}}$ produces the increment of a general function $f$ due to a rotation $R$ 
through $d\phi$ about the axis $\mathbf{\hat{n}}$, that is,
\oneline{
-i d\phi \left(\frac{ J_{\mathbf{\hat{n}}}  }{\hbar} \right) f = D(\mathbf{\hat{n}}, d\phi) f - f \ .
}
By compounding successively infinitesimal rotations about the same axis, we can obtain the finite rotation operator
\multiline{
D( \mathbf{\hat{n}}, \alpha) &=& \lim_{m\rightarrow \infty} \left[ 1- i\left(\frac{ \alpha}{m}\right) \left( \frac{ J_{\mathbf{\hat{n}}} }{\hbar}\right) \right]^m \nonumber \\
&=& \exp\left( \frac{ - i \alpha  J_{\mathbf{\hat{n}} } }{\hbar} \right) \nonumber \\
&=& 1 - \frac{i\alpha J_{\mathbf{\hat{n}} } }{\hbar} - \frac{1}{2!} \frac{ \alpha^2 J_{\mathbf{\hat{n}} }^2}{\hbar^2} + \cdots  \label{eqn:rotationInTermsOfJ}
} 
It is obvious that the set of all rotations satisfy the four conditions of a group (see 
\autoref{sec:groupTheory}). Correspondingly, 
 the infinite set of rotation operators also satisfy the same conditions and thus form a group which is called the full 
rotation group. 


Rotations have the following important property:  rotations about the same axis commute whereas rotations about 
different axes do not.  This property leads to the commutation relationship of the angular momentum operator $J$. 
To illustrate this point, let's consider rotations $R_x$, $R_y$, and $R_z$ about the $x$, $y$ and $z$ axis respectively. In 
coordinate space, these rotations can be represented by $3\times 3$ matrices. For instance, the rotation of angle $\phi$ about the $z$ axis can be expressed as
\oneline{
R_{z} (\phi) = \mat{\cos\phi & -\sin\phi &  0 \\ \sin\phi& \cos\phi& 0 \\ 0& 0& 1}
} 
In the case of an infinitesimal angle $\varepsilon$, the above equation can be written as
\oneline{
R_{z} (\varepsilon) = \mat{1-\frac{\varepsilon^2}{2} & -\varepsilon &  0 \\ \varepsilon& 1-\frac{\varepsilon^2}{2}& 0 \\ 0& 0& 1}
} 
By comparing the effect of a $y$-axis rotation followed by an $x$-axis rotation with that of an $x$-axis rotation 
followed by a $y$-axis rotation, we can show that 
\oneline{
R_x (\varepsilon) R_y (\varepsilon) - R_y (\varepsilon) R_x (\varepsilon) = R_z (\varepsilon^2) - I \ . \label{eqn:rotationCommutation}
}
Assuming the corresponding rotation operators satisfy a similar equation as \autoref{eqn:rotationCommutation} and
making use of \autoref{eqn:rotationOperator}, we obtain
\oneline{
\left[ J_x, J_y\right] = i\hbar J_{z} \ .
}
Repeating the above analysis for other axes, we obtain the commutation relations of angular momentum:
\oneline{
\left[ J_i, J_j\right] = i\hbar \varepsilon_{ijk} J_{k} \ , \label{eqn:angularMomentumCommutation}
}
where $i$, $j$, $k$ can be any one of $x$, $y$ and $z$. 

Many important properties follow from the angular-momentum commutation relation represented by \autoref{eqn:angularMomentumCommutation}. For example, by making use of \autoref{eqn:angularMomentumCommutation}, we can show that the operator $\mathbf{J}^2$ defined by
\oneline{
\mathbf{J}^2 \equiv J_x^2 + J_y^2 + J_z^2
}
commutes with any one of $J_x$, $J_y$ and $J_z$, namely,
\oneline{
\left[ \mathbf{J}^2, J_k \right] = 0 , \;\;\; (k = x, y, z) \ .
}
Since $\mathbf{J}^2$ and $J_z$ commutes, the eigenstates of $\mathbf{J}^2$ can be chosen to be the eigenstates of
$J_z$ simultaneously. We denote these state by $\ket{j, m}$ such that
\multiline{
\mathbf{J}^2\ket{j, m} &=& j (j+1)\hbar^2 \ket{j, m} \nonumber \\
J_z \ket{j, m} &=& m\hbar \ket{j, m} \ .
}
To determine the allowed values for $j$ and $m$, it is convenient to work with the non-Hermitian operators
\oneline{
J_{\pm} = J_x \pm i J_y \ ,
}
where $J_{+}$ is called the raising operator and $J_{-}$ is called the lowering operator. $J_{\pm}$ satisfy the following
commutation relations:
\multiline{
\left[ J_{+}, J_{-}\right] &=& 2 \hbar J_z \nonumber \\
\left[ J_{z}, J_{\pm}\right] &=& \pm \hbar J_{\pm} \nonumber \\
\left[ \mathbf{J}^2, J_{\pm}\right] &=& 0 \ ,
}
all of which can easily be derived from \autoref{eqn:angularMomentumCommutation}. 

As a special case of the angular momentum operator, the orbital angular momentum operator 
$\mathbf{L} = \mathbf{r} \times \mathbf{p}$ satisfies the same commutation relations
\oneline{
\left[ L_i, L_j\right] = i\hbar \varepsilon_{ijk} L_{k} \ .
}
Using the fact that momentum is the generator of translation:
\multiline{
T(d\mathbf{x}) \ket{\mathbf{r}} = \left[ 1 - i \left(\frac{\mathbf{p}}{\hbar}\right) \cdot d\mathbf{x}\right]\ket{\mathbf{r}} = \ket{\mathbf{r} + dx \,\mathbf{\hat{x}}} \ ,
}
 we have
\multiline{
&&\left[ 1 - i \left(\frac{d\phi}{\hbar}\right) L_z\right] \ket{x, y, z} = \left[ 1 - i \left(\frac{d\phi}{\hbar}\right) \left( x p_y - y p_x \right)\right] \ket{x, y, z} \nonumber \\
&& = \left[ 1 - i \left(\frac{p_y}{\hbar}\right) x d\phi + i \left(\frac{p_x}{\hbar}\right) y d\phi \right] \ket{x, y, z} \nonumber \\
&& = \ket{x - y d\phi, y + x d\phi, z} \ . \label{eqn:rotationZ}
}
The above equation shows exactly the effect of an infinitesimal rotation about $z$ axis as one would expect. For an 
arbitrary state $\ket{\psi}$, an infinitesimal rotation about $z$ axis change the wavefunction $\left\langle x, y, z \right.\ket{\psi}$ to 
\oneline{
\bra{x, y, z} \left[ 1 - i \left(\frac{d\phi}{\hbar}\right) L_z\right] \ket{\psi} = \left\langle x + y d\phi, y - x d\phi, z\right. \ket{\psi} \ .
}
By changing to the spherical coordinates $(r, \theta, \phi)$, the above equation becomes
\multiline{
&&\bra{r, \theta, \phi} \left[ 1 - i \left(\frac{d\phi}{\hbar}\right) L_z\right] \ket{\psi} = \bra{r, \theta, \phi-d\phi} \psi \rangle \nonumber \\
&& =\bra{r, \theta, \phi} \psi \rangle -d\phi \frac{\partial}{\partial \phi} \bra{r, \theta, \phi} \psi \rangle \ .
}
Therefore, we obtain
\oneline{
L_{z} = -i\hbar \frac{\partial}{\partial \phi} \ .
}
In the way, we can also derive the expressions for other orbital angular momentum operators:
\multiline{
L_x &=& - i\hbar \left( -\sin\phi \frac{\partial}{\partial \theta} - \cot\theta \cos\phi \frac{\partial}{\partial \phi}\right) \ , \nonumber \\
L_y &=& - i\hbar \left( \cos\phi \frac{\partial}{\partial \theta} - \cot\theta \sin\phi \frac{\partial}{\partial \phi}\right) \ .
}


The eigenfunctions of $\mathbf{L}^2$ and $L_z$ are known as the spherical harmonics, which are given by
\oneline{
\braket{\mathbf{\hat{n}}}{l, m} = Y_{l m}(\theta, \phi)
}
where $\theta$ and $\phi$ specify the orientation of $\mathbf{\hat{n}}$. The dependence of spherical harmonics on 
angles can be separated:
\oneline{
Y_{l m}(\theta, \phi) = \Theta_{lm}(\theta) \Phi_{m}(\phi) \ , \label{eqn:sphericalHarmonicsExpression}
}
where
\oneline{
\Phi_{m}(\phi) = \sqrt{\frac{1}{2\pi}}\exp(i m \phi) \ ,
}
and 
\oneline{
\Theta_{lm}(\theta) = (-1)^{m} \left[ \frac{2 l + 1}{2} \frac{(l-m)!}{(l+m)!} \right]^{1/2} P_l^{m}(\cos\theta)
}
for $m\ge 0$ and $\Theta_{lm}(\theta) =(-1)^{m} \Theta_{l-m}(\theta) $ for $m<0$, and $P_l^{m}(\cos\theta)$ are the 
associated Legendre polynomials. Sometimes, it is more convenient
to work with modified spherical harmonics $C_{lm}$ defined by
\oneline{
C_{lm} = \sqrt{\frac{4\pi}{2 l + 1}} Y_{lm}(\theta, \phi) \ .
}

\subsection{Rotation matirces}

Since an arbitrary rotation can be decomposed into rotations about three coordinate axes ($x$, $y$, and $z$ axis), 
it is usually convenient to express the orientations of a body in terms of rotations about some fixed axes. For this to 
work, we have to choose an initial orientation as a reference on which the rotations are operating. 
For a vector described by the two spherical polar angles $\theta$ and $\phi$, we use a reference orientation parallel 
to the $Z$ axis. It is obvious to see that a rotation of this reference orientation through an angle $\theta$ about 
the space-fixed $Y$ axis and a rotation through an angle $\phi$ about the space-fixed $Z$ axis can reproduce the 
orientation of the vector. For a general body, three Euler angles $\phi$, $\theta$ and $\chi$ are needed to
 describe an arbitrary rotation.
We attach a second coordinate system 
$(x, y, z)$ to the body and refer it as the body-fixed axis system. This axis system is called 
body-fixed as by construction it moves and rotates along with the body as a whole. 
In contrast, the original coordinate system $(X, Y, Z)$ is fixed in space, so it is called the space-fixed axis system. 
To describe the orientation of a body, we imagine that the body is initially at a position with its body-fixed axis system
 coincident with the space-fixed axis system, and then we carry out the following three rotations:
\begin{enumerate}
  \item rotate through $\chi$ about the space-fixed $Z$ axis, 
  \item  rotate through $\theta$ about the space-fixed $Y$ axis,
  \item  rotate through $\phi$ about the space-fixed $Z$ axis. 
\end{enumerate}
These three rotations can produce arbitrary orientation of the body. For an intuitive understanding of the Euler 
angles, we link the orientation of the body directly with $\phi$, $\theta$ and $\chi$. It can be shown that $\theta$ and
$\phi$ can be used to define the orientation of the body-fixed $z$ axis in the space-fixed axis system,  in the same way as
$\theta$ and $\phi$ are used to define the orientation of a vector. The angle $\chi$ measures a rotation about the 
body-fixed $z$ axis. It is an azimuthal angle about the $z$ axis just as $\phi$ is an azimuthal angle about the $Z$ axis. 

Based on \autoref{eqn:rotationInTermsOfJ}, the rotation operator that corresponds to the three Euler angles can be 
written as
\oneline{
D(\phi, \theta, \chi) = \exp\left(\frac{-i \phi J_{Z}}{\hbar}\right) \exp\left(\frac{-i \theta J_{Y}}{\hbar}\right) \exp\left(\frac{-i \chi J_{Z}}{\hbar}\right) \label{eqn:rotationEulerAngles}
}
Because of the closure property of the full rotation group, the multiplication of three rotation operators in 
\autoref{eqn:rotationEulerAngles} is equivalent to another rotation operator
\oneline{
D(R) =  \exp\left(\frac{-i \alpha \mathbf{J}\cdot{\mathbf{\hat{n}}}}{\hbar}\right) = D(\phi, \theta, \chi) \ ,
}
which corresponds to a rotation $R$  through an angle $\alpha$ about an axis $\mathbf{\hat{n}}$. 

Now we study the 
matrix elements of the rotation operator $D(R)$. As a consequence of that fact that $\mathbf{J}^2$ commutes with 
any component of $\mathbf{J}$,  the rotation operator $D(R)$, a function of $J_{\mathbf{\hat{n}}}$, also commute
with $\mathbf{J}^2$. Thus the eigenfunctions $\ket{j, m}$ of $\mathbf{J}^2$ are also eigenfunctions of the rotation
operator and rotations don't change the total angular momentum. 
As a result, to see the effect of a rotation on a state with a definite angular momentum, we only need to calculate the
 matrix elements of the rotation operator, 
\oneline{
 D^{j}_{m^{\prime} m} = \bra{j, m^{\prime}}  \exp\left(\frac{-i \alpha \mathbf{J}\cdot{\mathbf{\hat{n}}}}{\hbar}\right) \ket{j, m} \ ,
}
between two states with the same $j$ value. 
The $(2 j + 1) \times (2 j + 1)$ matrix formed by $D^{j}_{m^{\prime} m}$ in the above equation is called the rotation
matrix.  Making use of \autoref{eqn:rotationEulerAngles}, we can easily show that
\multiline{
D^j_{m^{\prime}, m} (\phi, \theta, \chi) &=& \exp(-i\phi m^{\prime} - i\chi m) \matelement{j, m^{\prime} }{\exp(-i\theta J_Y /\hbar)}{j, m} \nonumber \\
&\equiv& \exp(-i\phi m^{\prime} - i\chi m) d^j_{m^{\prime}, m}(\theta) \ , \label{eqn:rotationMatrixEulerAngle}
}
where $d^j_{m^{\prime}, m}(\theta)$ is the element of a reduced rotation matrix. As the above equation shows, the
reduced rotation matrix can be calculated from the matrix representation of $J_Y$. We give its 
expression here:
\multiline{
d^j_{m, n}(\theta) =&& \sum_{t} (-1)^t \frac{\left[ (j+m)! (j-m)! (j+n)! (j-n)! \right]^{1/2} }{(j + m -t)! (j-n-t)! t! (t+n-m)!} \nonumber \\
&& \times \left( \cos\frac{\theta}{2} \right)^{2 j + m - n - 2 t} \left( \sin\frac{\theta}{2} \right)^{2 t - m + n } \ , \label{eqn:reducedRotationMatrix}
}
where the sum over $t$ is for all integers for which the factorial arguments are nonnegative. 

To understand the physical meaning of the rotation matrix, we start from a state $\ket{j, m}$ and rotate it. Even though
the rotation $R$ doesn't change the $j$ value, we will generally expect the $m$ value to be different after the rotation. 
Consequently, one would be interested in knowing the probability of the system being found in a state $\ket{j, m^{\prime}}$. By inserting an identity relation, the final state can be written as
\multiline{
D(R) \ket{j, m} &=& \sum_{j^{\prime}} \sum_{m^{\prime}} \ket{j^{\prime},m^{\prime}} \bra{j^{\prime},m^{\prime}} D(R) \ket{j, m} \nonumber \\
&=& \sum_{m^{\prime}} \ket{j, m^{\prime}} \bra{j, m^{\prime}} D(R) \ket{j, m} \nonumber \\
&=&  \sum_{m^{\prime}}  \ket{j, m^{\prime}} D^{j}_{m^{\prime}, m}(R) \ . \label{eqn:JMTransformRotation}
}
From the above equation, it is now obvious that the matrix element $D^{j}_{m^{\prime}, m}(R)$ is simply the amplitude for the rotated state to
be found in $\ket{j, m^{\prime}}$ when the original state is $\ket{j, m}$. 

\subsection{Properties of rotation matrices}

Let's consider some important properties of the rotation matrices. From the last subsection, we know  the basis ket 
$\ket{j, m}$ are orthonormal and remain so on rotation, this means that the matrices that represents rotations are
unitary, that is, $D^{-1} = D^{\dagger}$. More specifically, we have 
\oneline{
D^{j}_{m, n}(R^{-1}) = \left( D^{\dagger} \right)^{j}_{m, n}(R) = \left[ D_{n, m}^{j} (R)\right]^{*} \ ,
}
where $R^{-1}$ denotes the reverse of the rotation $R$. 

Another important property that we cover here is that the 
rotation matrix elements  $D^{j}_{m, n}$ reduces to spherical harmonics: 
\oneline{
D^{j}_{m, 0}(\phi, \theta, \chi) = C_{jm}(\theta, \phi)^{*} = \sqrt{\frac{2 j + 1}{4 \pi } } Y_{jm}(\theta, \phi)^{*} \ , \label{eqn:relationDandY}
}
when $j$ is an integer and $m$ or $n$ is zero. Given a general state $\ket{ \mathbf{\hat{n} } }$ with the unit vector 
$\mathbf{\hat{n}}$ in the orientation of $(\theta, \phi)$, we know from previous discussion that  it can be constructed 
from the state $\ket{\mathbf{\hat{z}}}$ by a rotation about $y$ axis by angle $\theta$ and a rotation about $z$ axis by
angle $\phi$, thus we obtain
\multiline{
\ket{ \mathbf{\hat{n} } } &=& D(\phi, \theta, \chi) \ket{\mathbf{\hat{z}}} \nonumber \\
&=& \sum_{l^{\prime}}\sum_{m^{\prime}} D(\phi, \theta, \chi) \ket{l^{\prime}, m^{\prime} } \braket{l^{\prime}, m^{\prime} } {\mathbf{\hat{z}}}
}
where $\chi$ is undetermined. Multiplying both sides of the above equation by $\bra{j, m}$, we get 
\oneline{
\braket{l, m } {\mathbf{\hat{n}}} = \sum_{m^{\prime}} D^{j}_{m, m^{\prime} } (\phi, \theta, \chi)  \braket{l, m^{\prime} } {\mathbf{\hat{z}}} \ . \label{eqn:DtransformSphericalHarmonics}
}
Based on the definition of the spherical harmonics, $\braket{l, m^{\prime} } {\mathbf{\hat{z}}}$ is just 
$Y_{lm}^{*}(\theta, \phi)$ with $\theta = 0$ and $\phi$ undetermined. At $\theta = 0$, the associated Legendre
 Polynomials $P_{lm}(cos\theta)$ vanish except $m=0$. Then from the expression of spherical harmonics,
 \autoref{eqn:sphericalHarmonicsExpression} , we know that $Y_{lm}$ vanish for $m\neq 0$. Then 
\multiline{
\braket{l, m^{\prime} } {\mathbf{\hat{z}}}  &=& Y_{lm}^{*}(\theta=0, \phi) \delta_{m, 0} \nonumber \\
&=& \sqrt{ \frac{2 l + 1}{4 \pi } }  \delta_{m, 0} \ ,
} 
and a substitution of this equation into \autoref{eqn:DtransformSphericalHarmonics} gives rise to \autoref{eqn:relationDandY}. 

Finally, we consider the integral of rotation matrices. From \autoref{eqn:rotationMatrixEulerAngle} and 
\autoref{eqn:reducedRotationMatrix}, it can shown that the integral of a rotation matrix over $d\omega = \sin\theta d\phi d\theta d\chi$ is a
product of delta functions:
\oneline{
\int D^j_{m^{\prime}, m}(\phi, \theta, \chi) d\omega = \delta_{j, 0}  \delta_{m^{\prime}, 0}  \delta_{m, 0} \ .
}
Similarly, the integral of two rotation matrices is given by
\oneline{
\int D^{j_1}_{m_1^{\prime}, m_1}(\phi, \theta, \chi)^* \; D^{j_2}_{m_2^{\prime}, m_2}(\phi, \theta, \chi)d\omega = \frac{8\pi^2}{2 j_1 + 1}\delta_{j_1, j_2}  \delta_{m_1^{\prime}, m_2^{\prime}}  \delta_{m_1, m_2} \ . \label{eqn:normalizationRotationMatrix} \ ,
}
which describes the normalization condition for rotation matrices. 

\subsection{Coupling of angular momenta}

Suppose a system can be divided into two parts with different anglular momentum $\mathbf{J}_1$ and $\mathbf{J}_2$ 
respectively. When the two parts of the system interact through some physical mechanism, $\mathbf{J}_1$ and 
$\mathbf{J}_2$ become coupled and we define the addition of the two angular momenta as
\oneline{
\mathbf{J} = \mathbf{J}_1 +  \mathbf{J}_2 \ . 
 }  
It is easy to verify that $\mathbf{J}$ also satisfies the commutation rule of angular momentum
 (\autoref{eqn:angularMomentumCommutation}) and thus the sum of two angular momenta  is also an angular momentum.

In quantum mechanics, the state of a system is described by the simultaneous eigenfunctions of a complete set of 
commuting operators. For the current system of angular momenta, there are two complete sets of angular momentum
operators.  One set is $\mathbf{J}_1^2$, $\mathbf{J}_2^2$, $J_{1Z}$, and $J_{2Z}$, and therefore we can use their
simultaneous eigenkets $\ket{j_1, m_1; j_2, m_2}\equiv \ket{j_1, m_1}\ket{j_2, m_2}$ to describe the system.
This representation is called the uncouple representation. The other complete set of commuting angular 
 momentum operators is $\mathbf{J}_1^2$, $\mathbf{J}_2^2$, $\mathbf{J}^2$, and $J_Z$. Their simultaneous 
 eigenkets $\ket{(j_1 j_2)j, m}$ which we sometimes write as $\ket{j, m}$ are used to describe the system. This
  representation is called the coupled representation as the quantum numbers $j$ and $m$ of the coupled angular 
  momentum are used. Note that the bracket ``()'' in $\ket{(j_1 j_2)j, m}$ indicates the coupling of the two 
  angular momenta $\mathbf{J}_1$ and $\mathbf{J}_2$ and the result of the coupling is represented by the number $j$ immediately after the bracket.

The two representations are describing the same set of states of the system, so they can be related by the following unitary
transformation in the sense of connecting two bases:
\oneline{
\ket{(j_1 j_2)j, m} = \sum_{m_1, m_2} \ket{j_1, m_1; j_2, m_2} \braket{j_1, m_1; j_2, m_2} {(j_1 j_2)j, m} \ , \label{eqn:defCGCoeffs}
}
where $\braket{j_1, m_1; j_2, m_2} {(j_1 j_2)j, m}$ are called the Clebsch-Gordan coefficients. Due to symmetry 
consideration, it is often convenient to expressed the coefficients by the 3-$j$ symbols. The relation between Clebsch-Gordan 
coefficient and 3-$j$ symbol is given by the following two equations:
\begin{equation}
\left( 
\begin{array}{ ccc }
j_{1} & j_{2} & j_{3} \\
m_{1} & m_{2} & m_{3}
\end{array}
\right) \equiv (-1)^{j_{1} - j_{2} - m_{3}} (2j_{3} +1)^{-\frac{1}{2}} \langle j_{1}m_{1}, \; j_{2} m_{2} | j_{3}\; -m_{3}\rangle \ ,
\end{equation}
\begin{equation}
 \langle j_{1}m_{1}, \; j_{2} m_{2} | j_{3}\; -m_{3}\rangle \equiv (-1)^{j_{1} - j_{2} + m_{3}} (2j_{3} +1)^{\frac{1}{2}}
\left( 
\begin{array}{ ccc }
j_{1} & j_{2} & j_{3} \\
m_{1} & m_{2} & -m_{3}
\end{array}
\right)  \ . \label{CG-3j}
\end{equation}
The 3-$j$ symbols are more symmetric than Clebsch-Gordan coefficients. But we do not discuss the symmetry properties here as they are not important for understanding the thesis. The interested readers should refer to Zare's book\cite{zare-book}.  The 3-$j$ symbols contain some selection rules of angular momentum, that is, 
%%\begin{itemize}
%%\item 
\begin{equation}
\left(
\begin{array}{ccc}
j_{1}&j_{2}&j_{3} \\
m_{1}&m_{2}&m_{3}
\end{array}
\right)
=0 \;\;\; \mbox{unless $m_{1} + m_{2} + m_{3} = 0$ and $|j_1 - j_2| \leq j_3 \leq j_1 + j_2$, } \nonumber 
\end{equation}
which physically means that only certain angular momenta are coupled. Many calculations in molecular spectroscopy boil
down to the evaluation of 3-$j$ symbols. For details on the evaluation, readers should refer to Zare's book\cite{zare-book}, which mentions some efficient algorithms for calculating 3-$j$ 
symbols and gives algebraic expressions for the commonly encountered 3-$j$ symbols in Table 2.5. 
 

%%
%%\item An even permutation of the columns of 3-$j$ symbol does not change its value and an odd permutation multiplier the initial value by $(-1)^{j_{1} + j_{2} + j_{3}}$. 
%%
%%\item Moreover,
%\begin{equation}
%\left\{
%\begin{array}{ccc}
%j_{1}&j_{2}&j_{3} \\
%m_{1}&m_{2}&m_{3}
%\end{array}
%\right\}
%=
%(-1)^{j_{1} + j_{2} + j_{3}}
%\left\{
%\begin{array}{ccc}
%j_{1}&j_{2}&j_{3} \\
%-m_{1}&-m_{2}&-m_{3}
%\end{array}
%\right\}
%\end{equation}
%This implies
%\begin{equation}
%\left\{
%\begin{array}{ccc}
%j_{1}&j_{2}&j_{3} \\
%0&0&0
%\end{array}
%\right\}
%=0 \;\; \;\;\mbox{unless } j_{1} + j_{2} + j_{3} = \mbox{even.}
%\end{equation}
%%
%\end{itemize}

Different from the coupling of two angular momenta, the coupling of three angular momenta has more than one possible
 coupling schemes, and all these coupling schemes are related by unitary transformations. We might couple $\mathbf{j}_{1}$, $\mathbf{j}_{2}$ and $\mathbf{j}_{3}$ in such a way that
  $\mathbf{j}_{1} + \mathbf{j}_{2}=\mathbf{j}_{12}$, $\mathbf{j}_{12} + \mathbf{j}_{3}= \mathbf{j}$. The eigenfunctions in this coupling scheme is give by $\ket{\left[(j_1 j_2) j_{12} j_3\right] j, m}$, where
  the square bracket ``[]'', just like the bracket ``()'',  indicates the coupling of two angular momenta $j_{12}$ and $j_3$ 
  to produce  $j$. We can also couple $\mathbf{j}_{2}$ and $\mathbf{j}_{3}$ to produce a new angular momentum 
  $\mathbf{j}_{23}$, and then couple $\mathbf{j}_{23}$ with  $\mathbf{j}_{1}$ to produce the total angular
   momentum  $\mathbf{j}$. The corresponding eigenfunction is given by $\ket{\left[j_1(j_2 j_3) j_{23}\right] j, m}$. 
The transformation between the two different coupling schemes is
\oneline{
 \ket{ (j_1 j_{23}) j, m } = \sum_{ j_{12} } \braket{ (j_{12} j_3) j, m }{ (j_1 j_{23}) j^{\prime}, m^{\prime} } \ket{(j_{12} j_3) j^{\prime}, m^{\prime}} \delta_{j, j^{\prime}} \delta_{m, m^{\prime}} \ ,
}
where $\ket{ (j_1 j_{23}) j, m }$ is a short-hand way to write  $\ket{\left[j_1(j_2 j_3) j_{23}\right] j, m}$.  In the above 
equation, since $\braket{ (j_{12} j_3) j, m \allowbreak}{ (j_1 j_{23}) j^{\prime}, m^{\prime} }$ is a scalar product, it 
doesn't depend on the orientation of the coordinate system and is independent of the projection quantum numbers. Thus we can drop $m$ and $m^{\prime}$ and write it as 
$\braket{ (j_{12} j_3) j }{ (j_1 j_{23}) j^{\prime} }$. These recoupling coefficients can be replaced with the so-called
6-$j$ symbols defined as
\multiline{
\left\{
\begin{array}{ccc}
j_{1}&j_{2}&j_{12} \\
j_{3}&j&j_{23}
\end{array}
\right\}
= (-1)^{j_1 + j_2 + j_3 + j} \left[ (2 j_{12} + 1 ) ( 2 j_{23} + 1 )\right]^{-\frac{1}{2} } \braket{ (j_{12} j_3) j }{ (j_1 j_{23}) j^{\prime} } \nonumber \\
}


  
  
Similarly, the coupling of four angular momenta also has more than one possible coupling schemes. For example, 
  $|\left[(j_{1}j_{2})j_{12}(j_{3}j_{4})j_{34}\right] j, m \rangle$ is associated with this coupling scheme:
 \oneline{
 \mathbf{j}_{1} + \mathbf{j}_{2}=\mathbf{j}_{12}\ , \;\;
 \mathbf{j}_{3} + \mathbf{j}_{4}=\mathbf{j}_{34} \ , \;\;
 \mathbf{j}_{12} + \mathbf{j}_{34}=\mathbf{j} \ , \nonumber
  } 
and  $|\left[(j_{1}j_{4})j_{14}(j_{2}j_{3})j_{23}\right] j, m \rangle$ is associated with another coupling scheme:
 \oneline{
 \mathbf{j}_{1} + \mathbf{j}_{4}=\mathbf{j}_{14}\ , \;\;
 \mathbf{j}_{2} + \mathbf{j}_{3}=\mathbf{j}_{23} \ , \;\;
 \mathbf{j}_{14} + \mathbf{j}_{23}=\mathbf{j} \ . \nonumber
  } 
The relation between the states corresponding to different
coupling schemes is
\begin{eqnarray}
|\left[(j_{1}j_{4})j_{14}(j_{2}j_{3})j_{23}\right] j, m \rangle &=& \sum_{j_{12}}\sum_{j_{34}} \langle (j_{1}j_{2})j_{12}(j_{3}j_{4})j_{34}j  | (j_{1}j_{4})j_{14}(j_{2}j_{3})j_{23}j \rangle \nonumber \\
& & \times |\left[ (j_{1}j_{4})j_{14}(j_{2}j_{3})j_{23}\right] j, m \rangle \ , \label{9-j1}
\end{eqnarray}
and correspondingly the 9-$j$ symbol is defined by
\begin{eqnarray}
&&\braket{ (j_{1}j_{2})j_{12}(j_{3}j_{4})j_{34}j } {(j_{1}j_{4})j_{14}(j_{2}j_{3})j_{23}j } \nonumber \\
& & = \sqrt{(2j_{12} + 1) (2j_{34} +1)(2j_{14}+1)(2j_{23} + 1)}
\left\{
\begin{array}{ccc}
j_{1}& j_{2}&j_{12} \\
j_{3}&j_{4}&j_{34} \\
j_{14}&j_{23}&j 
\end{array}
\right\} \ . \label{9-j2}
\end{eqnarray}
The last column of the 9-$j$ symbol is related to the first coupling scheme and the last row is 
associated with the second coupling scheme.

\subsection{Clebsch-Gordan series}

After discussing the coupling of angular momenta, we now return to rotation matrix and further develop its 
properties in the context of angular momentum coupling. We consider the connection between the uncoupled 
 $\ket{j_1, m_1}\ket{j_2, m_2}$ and the coupled $\ket{j, m}$ representations under a rotational transformation. 
Applying the rotation transformation on $\ket{j_1, m_1}$, $\ket{j_1, m_1}$ and $\ket{j, m}$ individually and using
\autoref{eqn:JMTransformRotation}, we obtain
\multiline{
&&\sum_{m_1^{\prime}} \sum_{m_2^{\prime} } D^{j_1}_{m_1^{\prime}, m_1}(R) D^{j_2}_{m_2^{\prime}, m_2}(R) \ket{j_1, m_1^{\prime}}\ket{j_2, m_2^{\prime} } \nonumber \\
&& = \sum_{j}\sum_{m^{\prime}} \braket{j_1, m_1; j_2, m_2} {j, m} D^{j}_{m^{\prime}, m}(R) \ket{j, m^{\prime}} \ .
}
Multiplying both sides by $\bra{j_1, m_1^{\prime}}\bra{j_2, m_2^{\prime} }$, we obtain the so-called Clebsch-Gordan 
series:
\multiline{
D^{j_1}_{m_1^{\prime}, m_1}(R)\; D^{j_2}_{m_2^{\prime}, m_2}(R) &=& \sum_{j} \braket{j_1, m_1; j_2, m_2} {j, m}  \braket{j_1, m_1^{\prime}; j_2, m_2^{\prime} } {j, m^{\prime} }  D^{j}_{m^{\prime}, m}(R)  \nonumber \\
&=&  \sum_{j} (2 j + 1) \threejm{j_1}{m_1}{j_2}{m_2}{j}{m}  \threejm{j_1}{m_1^{\prime} }{j_2}{m_2^{\prime}}{j}{m^{\prime}} D^{j}_{m^{\prime}, m}(R)^* \nonumber \\ \label{eqn:CGSeries}
}
Similarly, the inverse Clebsch-Gordan series are given by
\multiline{
D^{j}_{m^{\prime}, m}(R)^* 
&= & \sum_{m_1}\sum_{ m_1^{\prime}} \sum_{m_2} \sum_{m_2^{\prime} } (2 j + 1) \threejm{j_1}{m_1}{j_2}{m_2}{j}{m}  \threejm{j_1}{m_1^{\prime} }{j_2}{m_2^{\prime}}{j}{m^{\prime}}  \nonumber \\
&& \times D^{j_1}_{m_1^{\prime}, m_1}(R) \; D^{j_2}_{m_2^{\prime}, m_2}(R) \ .
}

The Clebsch-Gordan series can help us to evaluate the integral over a product of three rotation matrix elements. 
Multiplying both sides of \autoref{eqn:CGSeries} by $D^{j_3}_{m_3^{\prime}, m_3}(R)$, we have
\multiline{
D^{j_1}_{m_1^{\prime}, m_1}(R)\; D^{j_2}_{m_2^{\prime}, m_2}(R)  \; D^{j_3}_{m_3^{\prime}, m_3}(R) &=& \sum_{j_3} (2 j_3 + 1) \threejm{j_1}{m_1}{j_2}{m_2}{j_3}{m_3}  \threejm{j_1}{m_1^{\prime} }{j_2}{m_2^{\prime}}{j_3}{m_3^{\prime}}  \nonumber \\
&\times& D^{j_3}_{m_3^{\prime}, m_3}(R)^*  \; D^{j_3}_{m_3^{\prime}, m_3}(R) \ .
}
Then integrating over $d\omega = \sin\theta d\phi d\theta d\chi$, we obtain
\multiline{
&&\int D^{j_1}_{m_1^{\prime}, m_1}(\phi, \theta, \chi) \; D^{j_2}_{m_2^{\prime}, m_2}(\phi, \theta, \chi) \; D^{j_3}_{m_3^{\prime}, m_3}(\phi, \theta, \chi) d\omega \nonumber \\
&& = 8\pi^2  \threejm{j_1}{m_1}{j_2}{m_2}{j_3}{m_3}  \threejm{j_1}{m_1^{\prime} }{j_2}{m_2^{\prime}}{j_3}{m_3^{\prime}} 
}
by utilizing the normalization condition of rotation matrix (see \autoref{eqn:normalizationRotationMatrix}). The 
above equation is very important because alternative ways to evaluate the integral is very laborious. We will use this
equation in the derivation of the Wigner-Eckart theorem. 

\subsection{Spherical tensor operators}

We have seen how the angular momentum wavefunction $\ket{j, m}$ transforms under a rotation.  Specifically, based 
on  \autoref{eqn:JMTransformRotation}, a general state ket $\ket{\alpha}$ is changed under a rotation $R$ according to
\oneline{
\ket{\alpha} \rightarrow D(R) \ket{\alpha} \ .  \label{eqn:ketStateTransform}
}
Now we study how an operator transforms under a rotation. Let's consider 
a so-called vector operator, for example $\mathbf{J}$, which is composed of operators $J_X$, $J_Y$ and $J_Z$
in three directions. We know a vector in classical physics with three components transforms like $V_j \rightarrow \sum_{j} R_{i, j} V_j$ under a rotation R. It is reasonable to expect that the expectation value of a vector operator 
$\mathbf{V}$ transforms like a classical vector, that is,
\oneline{
\matelement{\alpha}{V_i}{\alpha} \rightarrow \matelement{\alpha}{D^{\dagger}(R) V_i D(R) }{\alpha} = \sum_{j} R_{i, j} \matelement{\alpha}{V_j}{\alpha} \ , \label{eqn:expectationValueTransform}
}
where \autoref{eqn:ketStateTransform} is used. From the above equation, it follows that the transformed operator in
the original basis is given by
\oneline{
D^{\dagger}(R) V_i D(R) = \sum_{j} R_{i, j} V_j \ . \label{eqn:operatorTransform}
}

By generalizing the definition of a vector $V_j \rightarrow \sum_{j} R_{i, j} V_j$, we define a tensor as a quantity which
transforms like
\oneline{
T_{ijk \cdots } \rightarrow \sum_{i^{\prime}} \sum_{j^{\prime} } \sum_{k^{\prime} } R_{i, i^{\prime}} R_{j, j^{\prime}} R_{k, k^{\prime} }\cdots T_{i^{\prime} j^{\prime} k^{\prime} \cdots } \ . 
}
The number of indices is called the rank of a tensor. Such a tensor is called a Cartesian tensor because its components
$T_{ijk \cdots }$ is defined with respect to the Cartesian axes. 

The problem with a Cartesian tensor is that it is reducible, which means that it contains parts that
transform differently under rotations. Take for example a Cartesian tensor $\mathbf{T}$ formed from two vectors $\mathbf{U}$ and $\mathbf{V}$ as follows:
\oneline{
T_{ij} \equiv U_{i} V_{j} \ .
}
It can be shown that the tensor can be decomposed into the following parts:
\oneline{
U_{i} V_{j} = \frac{\mathbf{U}\cdot\mathbf{V} }{3} \delta_{i, j} + \frac{U_i V_j - U_j V_i }{2} + \left( \frac{U_i V_j +  U_j V_i }{2} -  \frac{\mathbf{U}\cdot\mathbf{V} }{3} \delta_{i, j} \right) \ . \label{eqn:decomposeTensor}
}
The first term on the right hand side is a scalar and has 1 independent component. The second term looks like a 
cross product of two vectors  and has 3  independent components. The third term is more complicated and we can
rewrite it as
\oneline{
(1-\delta_{i, j} ) \frac{U_i V_j - U_j V_i }{2}  + \delta_{i, j}\left( U_i V_i - \frac{\mathbf{U}\cdot\mathbf{V} }{3} \right) \ , \nonumber 
}
where the first term represents the off-diagonal part of the $3\times 3$ tensor and the second term represent the 
diagonal part. Now,  it can be easily seen that the tensor is symmetric and its trace is zero. Thus, the tensor contains
5 independent components. The number of independent components associated with the three terms in \autoref{eqn:decomposeTensor} is 1, 3, and 5 respectively, which are precisely the multiplicities of angular momenta
with $l=0$, $l=1$, and $l=2$ respectively. This suggests that the tensor $\mathbf{T}$ can be decomposed into tensors
that can transform like spherical harmonics with $l=0$, 1, and 2. 

From  the above example of decomposition of the tensor $\mathbf{T}$, it seems like the spherical harmonics can be 
used as the irreducible tensor to represent any reducible tensor. This motivates us to define the irreducible spherical
tensors based on the spherical harmonics. Before presenting the definition of a spherical tensor, let's first investigate
how the spherical harmonics tranform under rotations. For a direction eigenket $\ket{  \mathbf{\hat{n}}  }$, a rotation 
transform it another direction eigenket $\ket{  \mathbf{\hat{n }^{\prime} }  }$:
\oneline{
\ket{  \mathbf{\hat{n}^{\prime}  }  } = D(R) \ket{  \mathbf{\hat{n}}  } \ .
}
By taking the conjugate of the above equation and then multiplying $\ket{l m}$ on the right, we obtain
\oneline{
Y_{lm}(\mathbf{\hat{n}^{\prime} }) = \sum_{m^{\prime} } Y_{lm^{\prime} }(\mathbf{\hat{n} })\left(D^{\dagger}\right)^{l}_{m^{\prime} m}(R) \ . \label{eqn:howYTransform}
}
The spherical harmonics can be used as both function and operators just like the coordinates $x$, $y$, $z$ can also 
be used as position operators. Treating the spherical harmonics as operators, $Y_{lm}(\mathbf{\hat{n}^{\prime} })$ on
 the left hand side of \autoref{eqn:howYTransform} are the operator after the rotation transformation, which can be written in terms of  
the original operator $D^{\dagger}(R) Y_{lm}(\mathbf{\hat{n} }) D(R)$ based on \autoref{eqn:operatorTransform}. 
Then \autoref{eqn:howYTransform} becomes
\oneline{
D^{\dagger}(R) Y_{lm}(\mathbf{\hat{n} }) D(R) = \sum_{m^{\prime} } Y_{lm^{\prime} }(\mathbf{\hat{n} })D^{l\, *}_{m m^{\prime} }(R)
}
Similar to the above equation, we define a spherical tensor operator of rank $k$ with $(2 k + 1)$ components as
\oneline{
D^{\dagger}(R) T_{q}^{(k)}(\mathbf{T}) D(R) = \sum_{q^{\prime} =-k}^{k} D^{k\, *}_{q q^{\prime} }(R) T_{q^{\prime}}^{(k)}(\mathbf{T}) \ .
}
Replacing the rotation $R$ by its inverse $R^{-1}$, the above definition can be recasted into the following form:
\oneline{
D(R)  T_{q}^{(k)}(\mathbf{T}) D^{\dagger}(R)  = \sum_{q^{\prime} =-k}^{k} T_{q^{\prime}}^{(k)}(\mathbf{T})  D^{k}_{q^{\prime} q }(R) \ .
}

\subsection{Coupling of spherical tensors}

As we have seen, the spherical tensors behave like spherical harmonics. As a result, the spherical tensors  
couple in the same way as angular momenta. For example, two spherical tensor $\mathbf{R}^{k_1}$ and  
$\mathbf{S}^{k_2}$ can be combined to form a tensor of rank $K$:
\oneline{
T_{P}^{(K)}(\mathbf{R}^{k_1}, \mathbf{S}^{k_2} ) = \sum_{p_1,  p_2} \braket{k_1, p_1; k_2, p_2}{K, P} T_{p_1}^{(k_1)}(\mathbf{R}) T_{p_2}^{(k_2)}(\mathbf{S}) \ . \label{eqn:tensorCouple}
}
In practice, we can regard the above equation as the coupling of two angular momenta $\mathbf{j}_1$ and 
$\mathbf{j}_2$ to form $\mathbf{j}$. Comparing \autoref{eqn:defCGCoeffs} with \autoref{eqn:tensorCouple}, we can easily see that $j_1$ corresponds to $k_1$, $p_1$ to $m_1$, $j_2$ to $k_2$, $p_2$ to $m_2$,
$j$ to $K$, and $m$ to $P$. Similarly, $K$ can only takes the values from $|k_1 - k_2|$ to $k_1 + k_2$. We call 
$T_{P}^{(K)}$ as the tensor produce of $\mathbf{R}^{k_1}$ and  $\mathbf{S}^{k_2}$ and sometimes denote it as
\oneline{
T_{P}^{(K)}(\mathbf{R}^{k_1}, \mathbf{S}^{k_2} ) = \left[  T^{(k_1)}(\mathbf{R}) \otimes T^{(k_2)}(\mathbf{S}) \right]^{(K)}_{P} \ . \nonumber
}

For later reference, we write the two important cases of \autoref{eqn:tensorCouple} here:
\oneline{
 \left[  A^{(k)} \otimes B^{(k)} \right]^{(0)}_{0}  = (2 k + 1)^{-\frac{1}{2}} \sum_{q} (-1)^{k-q} A^{(k)}_q  B^{(k)}_{-q} \  , \label{eqn:tensorContraction00} 
}
\multiline{
 \left[  A^{(1)} \otimes B^{(1)} \right]^{(0)}_{0}  &=& \sum_{q_{A} } \braket{1, q_{A} ; 1, q_{B} } {2, q} A^{(1)}_{q_{A} }  B^{(1)}_{q_B } \nonumber \\
&=& \sum_{q_{A} } (-1)^q \sqrt{5} \threejm{1}{q_A }{1}{q_B }{2}{-q} A^{(1)}_{q_{A} }  B^{(1)}_{q_B }
\  . \label{eqn:tensorContraction112} 
}

\subsection{Matrix elements of tensor operators} 



\section{Diatomic molecule in external field}
\label{sec:moleculeInField}

\subsection{DC field}
\label{sec:dcField}

\subsection{AC field}
\label{sec:acField}



\section{Dipole-dipole interaction between two molecules in optical lattices}
\label{sec::ddInteraction}

The energy difference between the excited sate and the ground state, $\Delta_{eg}$ can be easily calculated by diagonalizing  the total Hamiltonian including the external potential. We know how to calculate the potential due to dipole-dipole interaction 
\begin{equation}
\hat{V}_{\mbox{\tiny dd}} = \left(\frac{1}{R^3}\right)\left[ \mathbf{d}_{A}\cdot\mathbf{d}_{B}-3(\mathbf{d}_{A}\cdot\hat{R})\cdot(\mathbf{d}_{B}\cdot\hat{R})\right] \ , \nonumber
\end{equation}
($\vec{R}$ is the vector connecting the centers of mass of two molecules, and $\hat{R}$ is an unit vector in the direction of $\vec{R}$), 


In this appendix, we will calculate the dipole-dipole interaction between two molecules by using the theory of angular momentum. The treatment of the theory of angular momentum is simplified so that only the necessary part that are directly used in the derivation of dipole-dipole interaction is covered. The structure of this appendix is  unconventional. We starts with a problem and then develop the techniques to solve it. In a sense, the appendix is a series of problem-solving blocks. However, in order not to interrupt the logic of problem solving, we will not separate them into different subsections. 

We have two molecules A and B, which are in states $|N_{A}M_{A}\rangle$ and $|N_{B}M_{B}\rangle$ respectively. In free space, the molecules can rotate along the axis connecting their center of mass, then their state are expressed by
\begin{equation}
|N_{A}M_{A}\rangle |N_{B}M_{B}\rangle |lm\rangle  \ , \nonumber
\end{equation}
where $l$ is the total angular momentum of A and B together. 

However, in solid state or optical lattices, the position of molecules are fixed to a good approximation, they cannot rotate along that axis. In this case, their states are given by $|N_{A}M_{A}\rangle |N_{B}M_{B}\rangle$ if there is no interaction between them. If we consider the interaction between the two molecules, we can always use $|N_{A}M_{A}\rangle |N_{B}M_{B}\rangle$ as basis set. By expanding the Hamiltonian in this basis set and diagonalizing it, we can get the new eigenstates.

As the first step, we have to know what the interaction between the two molecules is. If A has a permanent dipole of $\mathbf{d}_{A}$ and B has a permanent dipole of $\mathbf{d}_{B}$ and the vector connecting their centers of mass is $\mathbf{R}$, then the dipole-dipole interaction between A and B is given by
\begin{equation}
\hat{V}_{\mbox{\tiny dd}} = \left(\frac{1}{R^3}\right)\left[ \mathbf{d}_{A}\cdot\mathbf{d}_{B}-3(\mathbf{d}_{A}\cdot\hat{R})(\mathbf{d}_{B}\cdot\hat{R})\right]  \ . \label{dd-operator}
\end{equation}
Here we only consider the dipole moment, there also exists quadrupole, octopole, ..., etc., but their magnitudes are so small that we can ignore them at large distances, say, 400 nm. Our final goal is to evaluate the matrix element $\langle N_{A}M_{A}|\langle N_{B}M_{B} | \hat{V}_{\mbox{\tiny dd}}|N_{A}^{'}M_{A}^{'}\rangle |N_{B}^{'}M_{B}^{'}      \rangle$. The general scheme to calculate the matrix elements is based on the fact that the dipole-dipole operator (Eq. (\ref{dd-operator}) ) can be expressed in terms of harmonics operators. Then its matrix elements can be written in terms of 3-j symbols, which can be calculated numerically. 

We first consider the simpler term in Eq. (\ref{dd-operator}), $\mathbf{d}_{A}\cdot\mathbf{d}_{B}$, which can be expressed in terms of a tensor product:
\begin{equation}
\mathbf{d}_{A}\cdot\mathbf{d}_{B} = - \sqrt{3} \left[ \mathbf{d}_{A}^{(1)} \otimes \mathbf{d}_{B}^{(1)} \right]_{0}^{(0)} \ . \label{dot-product}
\end{equation}
But why should we do that? The answer will become clear later when you see the tensor product is related to spherical harmonics. Now, the task is to prove Eq. (\ref{dot-product}). The proof is enclosed between two straight lines. 

-----------------------------------------------------------------------------------------------

Based on the property of tensor product, we have
\begin{equation}
\left[ A^{(1)} \otimes B^{(1)} \right]_{q}^{(k)} =\sum_{m} \langle 1m, 1\;q-m | kq \rangle A(1,m)B(1, q-m) \ , \label{tensor-product}
\end{equation}
where the superscript $(k)$ is the rank of the tensor. A tensor $T$ of rank one is a vector operator. It consists of three operators 
\begin{equation}
T(1,1) = - \frac{1}{\sqrt{2}} (x + i y) \ , \label{t11}
\end{equation}
\begin{equation}
T(1,0) = z \ , \label{t10}
\end{equation}
and 
\begin{equation}
T(1,-1) =  \frac{1}{\sqrt{2}} (x - i y) \ .  \label{t1-1}
\end{equation}

The intuitive understanding of Eq. (\ref{tensor-product}) is that two angular momentums $j_{1} =1$ and $j_{2}=1$ couple to give rise to another angular momentum $j_{3}=k$ with the projection along some direction to be $q$. 

In order to know the relation between $\mathbf{d}_{A}\cdot\mathbf{d}_{B}$ and $\left[ \mathbf{d}_{A}^{(1)} \otimes \mathbf{d}_{B}^{(1)} \right]_{0}^{(0)}$, we need to calculate $\langle 1m, 1\;q-m | kq \rangle$ in Eq. (\ref{tensor-product}).

First, we consider the case where $k=1$ in Eq. (\ref{tensor-product}), that is $\left[ \mathbf{d}_{A}^{(1)} \otimes \mathbf{d}_{B}^{(1)} \right]_{0}^{(0)}$. Based on Eq. (\ref{tensor-product}), we obtain
\begin{eqnarray}
\left[ \mathbf{d}_{A}^{(1)} \otimes \mathbf{d}_{B}^{(1)} \right]_{0}^{(0)} & = & \sum_{m=-1,0,1} \langle 1 m, 1 -m | 00\rangle \mathbf{d}_{A}(1, m) \mathbf{d}_{B}(1, -m) \nonumber \\
&=& \langle 1 1, 1 -1 | 00\rangle \mathbf{d}_{A}(1, 1) \mathbf{d}_{B}(1, -1) + \langle 1 0, 1 0 | 00\rangle \mathbf{d}_{A}(1, 0) \mathbf{d}_{B}(1, 0) \nonumber \\
&  & + \langle 1 -1, 1 1 | 00\rangle \mathbf{d}_{A}(1, -1) \mathbf{d}_{B}(1, 1) \nonumber \\
&=&\frac{1}{\sqrt{3}} \mathbf{d}_{A}(1, 1) \mathbf{d}_{B}(1, -1) - \frac{1}{\sqrt{3}}\mathbf{d}_{A}(1, 0) \mathbf{d}_{B}(1, 0) +  \frac{1}{\sqrt{3}} \mathbf{d}_{A}(1, -1) \mathbf{d}_{B}(1, 1) \nonumber \\
\end{eqnarray}
(Mathematica can be used to evaluate the Clebsch-Gordan coefficients.)
Substituting Eq. (\ref{t11}), (\ref{t10}) and (\ref{t1-1}) into the above equation, we obtain
\begin{equation}
\left[ \mathbf{d}_{A}^{(1)} \otimes \mathbf{d}_{B}^{(1)} \right]_{0}^{(0)}  = -\frac{1}{\sqrt{3}} (\mathbf{d}_{A,x}\mathbf{d}_{B,x} + \mathbf{d}_{A,y}\mathbf{d}_{B,y} + \mathbf{d}_{A,z}\mathbf{d}_{B,z}) = -\frac{1}{\sqrt{3}} \mathbf{d}_{A}\cdot\mathbf{d}_{B} \ .
\end{equation}
Therefore, the tensor product of rank $k=0$ is related to the scalar product. Similarly, it can be shown that the tensor product of rank $k=1$ is related to the cross product. 
\begin{equation}
\left[ A^{(1)} \otimes B^{(1)} \right]_{q}^{(k)} = (\mathbf{A}\times \mathbf{B})\cdot \mathbf{\hat{e}}_{q} \ ,
\end{equation}
where
\begin{eqnarray}
\mathbf{\hat{e}}_{+1} &=& -\frac{1}{\sqrt{2}}(\mathbf{\hat{X}} + i \mathbf{\hat{Y}}) \nonumber \\
\mathbf{\hat{e}}_{0} &=& \mathbf{\hat{Z}} \nonumber \\
\mathbf{\hat{e}}_{-1} &=& \frac{1}{\sqrt{2}}(\mathbf{\hat{X}} - i \mathbf{\hat{Y}})
\end{eqnarray}


-----------------------------------------------------------------------------------------------

As for the second term in the square bracket in Eq. (\ref{dd-operator}), we can do the same thing. Instead of considering the product of two scalars $(\mathbf{d}_{A}\cdot\hat{R})(\mathbf{d}_{B}\cdot\hat{R})$, we consider it as a special case of the scalar product of two vectors (thinking of a vector with two components being zero ). Then according to Eq. (\ref{dot-product}), we have
\begin{equation}
(\mathbf{d}_{A}\cdot\hat{R})(\mathbf{d}_{B}\cdot\hat{R})= 3\left[ \left[ \mathbf{d}_{A}^{(1)} \otimes \mathbf{\hat{R}}^{(1)} \right]^{(0)}\otimes \left[ \mathbf{d}_{B}^{(1)} \otimes \mathbf{\hat{R}}^{(1)} \right]^{(0)}  \right]_{0}^{(0)}  \ .
\end{equation}
There is a simple understanding of the above equation: an angular momentum $\mathbf{d}_{A}$ with $j_{1}=1$ couples  another angular momentum $\hat{R}$ with $j_{2}=1$ to give rise to a new angular momentum, and an angular momentum $\mathbf{d}_{B}$ with $j_{3}=1$ couples  another angular momentum $\hat{R}$ with $j_{4}=1$ to give rise to another new angular momentum, and then these two new angular momentum couples with each other. It is necessary to regroup these operators such that dipole operators are separated from the position operators. The regrouping of operators corresponds to different coupling schemes (orders) of the four angular momentums. In the following, we will talk about the coupling of four angular momentums.

-----------------------------------------------------------------------------------------------



-----------------------------------------------------------------------------------------------

Let's expand $\left[ \left[ \mathbf{d}_{A}^{(1)} \otimes \mathbf{\hat{R}}^{(1)} \right]^{(0)}\otimes \left[ \mathbf{d}_{B}^{(1)} \otimes \mathbf{\hat{R}}^{(1)} \right]^{(0)}  \right]_{0}^{(0)}$ in terms of another coupling scheme, namely $\mathbf{d}_{A}$ couples $\mathbf{d}_{B}$ and $\mathbf{\hat{R}}$ couples $\mathbf{\hat{R}}$. Based on Eq. (\ref{9-j1}) and (\ref{9-j2}), we have
\begin{equation}
(\mathbf{d}_{A}\cdot\hat{R})(\mathbf{d}_{B}\cdot\hat{R})= 3\sum_{k}(2k + 1) 
\left\{
\begin{array}{ccc}
1& 1&k \\
1&1&k \\
0&0&0 
\end{array}
\right\} 
\left[ \left[ \mathbf{d}_{A}^{(1)} \otimes \mathbf{d}_{B}^{(1)} \right]^{(k)}\otimes \left[ \mathbf{\hat{R}}^{(1)} \otimes \mathbf{\hat{R}}^{(1)} \right]^{(k)}  \right]_{0}^{(0)} \ . \label{regrouping}
\end{equation}
Obviously, in the above $9-j$ symbol $1+1=k$ and $1+1=k$ is the new coupling scheme, and $0+0=0$ is the original coupling scheme. The $9-j$ symbols can be expressed in terms of $6-j$ symbols (related to the coupling of 3 angular momenta). In the special case where the final angular momentum $j_{9}=0$, we have
\begin{eqnarray}
\left\{
\begin{array}{ccc}
j_{1}& j_{2}&j_{3} \\
j_{4}&j_{5}&j_{6} \\
j_{7}&j_{8}&j_{9}
\end{array}
\right\} 
&=&(-1)^{j_{2} + j_{3} + j_{4} + j_{7}}[(2j_{3} +1)(2j_{7} + 1)]^{-\frac{1}{2}} \nonumber \\
&  & \times 
\left\{
\begin{array}{ccc}
j_{1}& j_{2}&j_{3} \\
j_{5}&j_{4}&j_{7} 
\end{array}
\right\} 
\delta_{j_{3}j_{6}}\delta_{j_{7}j_{8}} \ . 
\end{eqnarray}
Then
\begin{eqnarray}
\left\{
\begin{array}{ccc}
1& 1&k \\
1&1&k \\
0&0&0 
\end{array}
\right\} 
&=& (-1)^{k+2} (2k+1)^{-\frac{1}{2}}
\left\{
\begin{array}{ccc}
1& 1&k \\
1&1&0 
\end{array}
\right\} \nonumber \\
&=& (-1)^{k+2} (2k+1)^{-\frac{1}{2}} 
\left\{ 
\begin{array}{cc}
\frac{1}{3} (-1)^{-k} & 0\leq k \leq 2 \\
0 & \mbox{otherwise}
\end{array}
\right.
\end{eqnarray}
Because $k$ results from the coupling of $j_{1}=1$ and $j_{2}=1$, it is in the range of $|j_{1}-j_{2}|, \cdots, |j_{1}+j_{2}|$, and the above equation can be simplified as
\begin{equation}
\left\{
\begin{array}{ccc}
1& 1&k \\
1&1&k \\
0&0&0 
\end{array}
\right\} = \frac{(2k + 1)^{-\frac{1}{2}}}{3} \ . \label{9-j-value}
\end{equation}  
Substituting Eq. (\ref{9-j-value}) into Eq. (\ref{regrouping}) and using Eq. (\ref{tensor-product}), we obtain
\begin{eqnarray}
(\mathbf{d}_{A}\cdot\hat{R})(\mathbf{d}_{B}\cdot\hat{R})&=&\sum_{k=0,1,2}\sum_{q}(2k+1)^{\frac{1}{2}}\langle k q, k -q | 00 \rangle\left[ \mathbf{d}_{A}^{(1)} \otimes \mathbf{d}_{B}^{(1)} \right]^{(k)}_{q} \left[ \mathbf{\hat{R}}^{(1)} \otimes \mathbf{\hat{R}}^{(1)} \right]^{(k)}_{-q}    \nonumber \\
&=& \sum_{k=0,1,2}\sum_{q} (-1)^{k-q} \left[ \mathbf{d}_{A}^{(1)} \otimes \mathbf{d}_{B}^{(1)} \right]^{(k)}_{q} \left[ \mathbf{\hat{R}}^{(1)} \otimes \mathbf{\hat{R}}^{(1)} \right]^{(k)}_{-q} \ . \label{scalarXscalar}
\end{eqnarray}
Since the tensor product of rank 1 is related to the cross product, $\left[ \mathbf{\hat{R}}^{(1)} \otimes \mathbf{\hat{R}}^{(1)} \right]^{(k=1)}_{-q}$ is associated with $\mathbf{\hat{R}}\times\mathbf{\hat{R}}$ and it is zero. Then Eq. (\ref{scalarXscalar}) can be simplified further as
\begin{eqnarray}
(\mathbf{d}_{A}\cdot\hat{R})(\mathbf{d}_{B}\cdot\hat{R})&=&\left[ \mathbf{d}_{A}^{(1)} \otimes \mathbf{d}_{B}^{(1)} \right]^{(0)}_{0} \left[ \mathbf{\hat{R}}^{(1)} \otimes \mathbf{\hat{R}}^{(1)} \right]^{(0)}_{0} \nonumber \\
& & + \sum_{|q|=0,1,2} (-1)^{q} \left[ \mathbf{d}_{A}^{(1)} \otimes \mathbf{d}_{B}^{(1)} \right]^{(2)}_{q} \left[ \mathbf{\hat{R}}^{(1)} \otimes \mathbf{\hat{R}}^{(1)} \right]^{(2)}_{-q} \nonumber \\
&=& (-\frac{1}{\sqrt{3}} \mathbf{d}_{A}\cdot \mathbf{d}_{B})(-\frac{1}{\sqrt{3}}\mathbf{\hat{R}}\cdot\mathbf{\hat{R}} ) \nonumber \\
& & +  \sum_{q} (-1)^{q} \left[ \mathbf{d}_{A}^{(1)} \otimes \mathbf{d}_{B}^{(1)} \right]^{(2)}_{q} \left[ \mathbf{\hat{R}}^{(1)} \otimes \mathbf{\hat{R}}^{(1)} \right]^{(2)}_{-q} \nonumber \\
&=& \frac{\mathbf{d}_{A}\cdot \mathbf{d}_{B}}{3}+  \sum_{q} (-1)^{q} \left[ \mathbf{d}_{A}^{(1)} \otimes \mathbf{d}_{B}^{(1)} \right]^{(2)}_{q} \left[ \mathbf{\hat{R}}^{(1)} \otimes \mathbf{\hat{R}}^{(1)} \right]^{(2)}_{-q} \nonumber \\ 
\label{term2}
\end{eqnarray}
Substitution of Eq. (\ref{term2}) into the expression for the dipole-dipole interaction (Eq. (\ref{dd-operator})) yields
\begin{equation}
\hat{V}_{\mbox{\tiny dd}}(\mathbf{R}) = \frac{-3}{R^{3}} \sum_{q} (-1)^{q} \left[ \mathbf{d}_{A}^{(1)} \otimes \mathbf{d}_{B}^{(1)} \right]^{(2)}_{q} \left[ \mathbf{\hat{R}}^{(1)} \otimes \mathbf{\hat{R}}^{(1)} \right]^{(2)}_{-q} \ .
\end{equation}

-----------------------------------------------------------------------------------------------

Before we go on, it is helpful to get familiar with rotational matrices and spherical harmonics.


-----------------------------------------------------------------------------------------------



\section{Introduction to exciton}
\label{sec: exciton}
We describe the state of a crystal system by specifying the occupation state of each molecules that constitute the crystal. For example, we denote the crystal wavefunction as
\begin{equation}
\left. |N_{1f_{1}} N_{2f_{2}} ...N_{nf} ... \right\rangle \nonumber \ ,
\end{equation}
where $N_{nf}$ represents the occupation number of state $f$ in molecule $n$ and it is either 0 or 1. Accordingly, the occupation number operator is defined as
\begin{equation}
\hat{N}_{nf} |...N_{nf}...\rangle = N_{nf} |...N_{nf}...\rangle \ .
\end{equation}
For one molecule $n$, it must be in some state $g$, therefore
\begin{equation}
\sum_{g}\hat{N}_{ng} = 1 \ . \label{must-be-in-a-state}
\end{equation}
It is convenient to introduce two operators $b_{nf}^{\dagger}$ and $b_{nf}$ associated with molecule $n$ and state $f$,
\begin{eqnarray}
b_{nf}^{\dagger} |...N_{nf}...\rangle &=& (1-N_{nf})|...(N_{nf}+1)...\rangle \ , \nonumber \\
b_{nf} |...N_{nf}...\rangle &=& N_{nf}|...(N_{nf}-1)...\rangle \ , \label{single-molecule-operator}
\end{eqnarray}
and express the occupation number operator as the product of these two operators,
\begin{equation}
\hat{N}_{nf} =b_{nf}^{\dagger} b_{nf} \ .
\end{equation}
The physical meaning of the two operators is clear: $b_{nf}$ creates a state $f$ in molecule $n$ and $b_{nf}$ destroy a state $f$ in molecule $n$.  It follows from Eq. (\ref{single-molecule-operator}) that
\begin{eqnarray}
b_{nf}b_{nf}^{\dagger} &+& b_{nf}^{\dagger}b_{nf} = 1 \ , \nonumber \\
b_{nf}b_{nf}&=&b_{nf}^{\dagger}b_{nf}^{\dagger} =0 \ . \label{commutation-relation-single-molecule}
\end{eqnarray}
Because an operator for a specific molecule $n$ and state $f$ does not operate on other molecules and other states, any two operators that correspond to different molecules $n$ and $m$ or different states $f$ and $f'$ commute all the time. 

Based on the physical meaning of $b_{nf}^{\dagger}$ and $b_{nf}$, it can be easily seen that the exciton creation and annihilation operators can be written as
\begin{equation}
P_{nf}^{\dagger} = b_{nf}^{\dagger} b_{n0} \ , \;\;P_{nf}=b_{n0}^{\dagger} b_{nf} \ . \label{creation&annihilation}
\end{equation}
Substituting Eq. (\ref{commutation-relation-single-molecule}), we have
\begin{eqnarray}
P_{nf}^{\dagger} P_{nf} &=& b_{nf}^{\dagger} b_{n0}b_{n0}^{\dagger} b_{nf} \nonumber \\
                                      &=& b_{nf}^{\dagger}(1-b_{n0}^{\dagger}b_{n0})b_{nf} \nonumber \\
                                      &=& \hat{N}_{nf} \ , \label{creation-annihilation}
\end{eqnarray}
\begin{eqnarray}
P_{nf} P_{nf}^{\dagger} &=& b_{n0}^{\dagger} b_{nf} b_{nf}^{\dagger} b_{n0} \nonumber \\
                                      &=& b_{n0}^{\dagger}(1-b_{nf}^{\dagger}b_{nf})b_{n0} \nonumber \\
                                      &=& \hat{N}_{n0} \ . \label{annihilation-creation}
\end{eqnarray}
Subtracting Eq. (\ref{creation-annihilation}) from Eq. (\ref{annihilation-creation}) and making use of Eq. (\ref{must-be-in-a-state}), we obtain
\begin{equation}
P_{nf} P_{nf}^{\dagger} - P_{nf}^{\dagger} P_{nf} = 1-\hat{N}_{nf} - \sum_{g\neq 0} \hat{N}_{ng}  \ . \label{exact-relation}
\end{equation}
Because $P_{nf}$ and $P_{nf}^{\dagger}$ don't operate on a different
molecule $m$, any exciton operators corresponding to different molecules
commute with each other. Then the exact statistics is given by 
\begin{equation}
P_{nf}P_{n'f}^{\dagger}-P_{nf}^{\dagger}P_{n'f}  =  \delta_{nn'}\left(1-\hat{N}_{nf}-\sum_{g\neq0}\hat{N}_{ng}\right)  \label{exact-relation2}
\end{equation}
and 
\begin{equation}
P_{nf}P_{nf'}^{\dagger}-P_{nf}^{\dagger}P_{nf'}=\delta_{ff'}\ . 
\end{equation}
Unfortunately, it is cumbersome to take into account of the exact
statistics of excitons in most cases. So it is often desirable to approximate excitons as Boses when only a small number of molecules in the crystal are excited. In other words, if the following inequality
\begin{equation}
\left\langle N_{nf} \right\rangle \ll 1 \label{condition-for-Bose-approx}
\end{equation}
holds, the exciton creation and annihilation operators satisfy the commutation rules for Bose operators
\begin{equation}
P_{n'f'}P_{nf}^{\dagger} - P_{nf}^{\dagger}P_{n'f'}=\delta_{nn'}\delta_{ff'} \ . \label{approx-relation}
\end{equation}
In the following derivation, we are assuming the exciton operators are Bose operators and satisfy Eq. (\ref{approx-relation}). 

The way to use operator $P_{nf}$ and $P_{nf}^{\dagger}$ to describe excitons is called site representation. Due to the periodicity of crystal, there is another way to represent excitons by making use of the Fourier transforms of $P_{nf}$ and $P_{nf}^{\dagger}$: 
\begin{eqnarray}
P_{nf} &=& \frac{1}{\sqrt{N}} \sum_{k} P_{f}(k) e^{ikn} \nonumber \\
P_{nf}^{\dagger} &=& \frac{1}{\sqrt{N}} \sum_{k} P_{f}^{\dagger}(k) e^{-ikn} \ , \label{unitary-transform}
\end{eqnarray}
and this is called wavevector representation. Since the unitary transformation does not change the commutation rule of operators, we expect that the following equation
\begin{equation}
P_{f',k'}P_{f,k}^{\dagger} - P_{f,k}^{\dagger}P_{f',k'}=\delta_{kk'}\delta_{ff'} \ . \label{approx-relation-in-wavevector}
\end{equation}
also hold. This can be verified by substituting Eq. (\ref{unitary-transform}) into Eq. (\ref{approx-relation}). 






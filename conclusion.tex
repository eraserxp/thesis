\chapter{Conclusions}
\label{ch:conclusion}

Trapping ultracold atoms and molecules in optical lattices has opened an exciting frontier of condensed 
matter research\cite{Baranov2012}. In the limit of strong trapping field that completely suppresses particle tunneling
in optical lattices, an artifical crystal of atoms and molecules, called the Mott insulator state, appears. Compared with their natural counterparts, these artifical crystals have some 
unique properties -- in particular, the ability to address single lattice site\cite{atom-mott1, atom-mott1} and the possibility of controlling interparticle interactions with external electric or magnetic fields\cite{quemener2012, Baranov2012},  which make them a very good platform to
study collective phenomena\cite{our-njp-review, quemener2012, Baranov2012}. The thesis, by utilizing  the controlliability of the artifical crystals of ultracold particles, aims to 
control the quantum dynamics
of quasiparticles. 
In the following, we summarize the research results of the thesis and point out possible future research directions.  

\section{Summary of the thesis}

In \autoref{ch:biexciton}, we studied a particular kind of quasiparticles -- rotational Frenkel excitons in a 
periodic lattice potential. These quasiparticles are the collective excitonic modes of polar molecules 
trapped on an optical lattice in the Mott insulator state, and are induced by the dipole-dipole interaction which couples the 
rotational states of different molecules. As superpositions of elementary excitations localized on different
 lattice sites, the rotational excitons  bear a strong resemblance to the electronic excitons in organic solids\cite{agranovich}.
Because of the controllability of the ultracold system of polar molecules, the properties of the rotational excitons can also 
be manipulated. In \autoref{ch:biexciton}, we were interested in the binding of the excitons. 
We showed that the application of an moderate electric field,  through mixing rotational states of different parity, can give rise 
to an non-linear dynamic interaction $D$ between the rotational excitons.  This dynamic interaction doesn't occur in molecular crystals 
with inversion symmetry and has so far eluded the experimental observation with only one exception\cite{frenkelxx, 
frenkelxx2}. For the aritifical crystal of polar molecules, we found that the dynamic interaction is always attractive and its 
strength can be tunned by the external DC field. This leads to controllable formation of biexciton states and tunable binding
energy as demonstrated numerically for a 1D array of LiCs molecules on an optical lattice. We also obtained the 
two-excitation spectra of the rotational excitons and derived analytical expressions for the wavefunction of biexciton states 
using the nearest-neighbor approximation. In an effort to extend the theoretical model of exciton binding, we calculated 
three-excitation spectra of rotational excitons and observed the three-body bound states of the excitons.
To make our theoretical study of biexciton states more relevant to experimental investigation, we proposed an nonoptical way to create the rotational biexciton states, avoiding the difficulty involved in 
direct excitation of biexcitons. This method makes use of the resonance between the high-energy (N=2) excitonic states 
and the biexciton states of low-energy (N=1) excitons and can produce biexcitons with high efficiency. 



In \autoref{ch:energy-transfer}, we studied a more general type of quasiparticles -- collective excitations in an array of 
coupled monomers and considered the generic problem of excitation energy transfer in the array of monomers. In our research, the excitations can be of any type and the array can be ordered or disordered.  We 
showed that the energy transfer through an array of coupled quantum monomers can be controlled by applying a 
transient external potential which modifies the phase of the quantum states of the individual monomers. The success of the method relies 
on the two very different time scales in the quantum evolution of the many-body states: one time scale is related to the 
local excitation of a single monomer, and the other depends on the interaction between two monomers that is responsible
for the propagation of excitation energy. If the former time scale is much faster than the latter, it is possible to find a suitable
local perturbation to a single monomer that is adiabatic with respect to the former time scale but is sudden
with respect to the latter time scale.  In an ordered crystal, if such perturbations are applied to give different phases to 
different monomers, the quasimoment of the collective excitation is modified and its propagation behavior is influenced
as well. Our research showed that different phase transformations can accelerate or decelerate quantum energy transfer and 
spatially focus delocalized excitations onto different parts of those ordered arrays. On the other hand, for a completely disordered array,  random scattering at numerous lattice sites disturbs the above phase transformations. In this case,  inspired by the ``transfer matrix'' methods for focusing of a 
collimated light beam in opaque medium\cite{opaque-1, Gigan-TMeasure-PRL10, Mosk-NPhot10, Cizmar-NPhot10, 
Silberberg-11, Chatel-Focusing-11, Lagendijk-Focusing-11, zhenia-11, cui-11, kim-11},
we developed another scheme of phase transformation that 
can achieve effective focusing of a delocalized excitation in the presence of strong disorders. To make connection with the current study of  
ultracold molecules,  we also considered possible experimental implementations of the proposed technique in an array of 
ultracold atoms or molecules trapped on an optical lattice and demonstrated the feasibility of the phase transformations. 
  


In \autoref{ch:greenfunc}, I developed a numerically efficient method to calculate the two-particle Green's functions in a 
lattice with arbitrary disorders. The method can be viewed as an extension of Berciu's method \cite{Berciu2010, Berciu2011, 
Berciu2012} to an aperiodic lattice. By grouping Green's functions into different set of vectors, this method rewrites the 
equation of motion of Green's function to a recursion relation that links three consecutive vector. Then  based on physical 
reasoning, certain boundary conditions can be assumed. This initialize the recursive calculations and all the Green's 
functions can be calculated.  The recursive method is presented in the form of a generic algorithm which can be easily 
adapted to long-range interactions and high-dimensional systems. As an application of the method, I proposed 
to use the Green’s function to study the tunneling of biexciton states through
 impurities.



\section{Future research directions}
%

The research on exciton binding, as presented in \autoref{ch:biexciton}, proposed a conceptually simple scheme to control 
exciton--exciton interactions and offered a new tool for various fundamental directions of research. In our research, we 
assumed the optical lattices are deep and ignored the translational motion of molecules within the trap. Future work might 
consider the scenario of Frenkel excitons coupled with lattice phonons in shallow optical lattices to form the so-called polarons. It would be very interesting to explore whether the attractive dynamic interaction 
between excitons leads to the formation of bipolarons. Another interesting direction of research is to extend the study of 
two-body and three-body bound states to $n$-body bound states and to investigate the mechanism of exciton clustering. 

The research on exciton propagation, as presented in \autoref{ch:energy-transfer}, proposed a very general way to control
the energy transfer. Future work might apply the schemes to various types of quasiparticles. For 
example, the biexciton state found in \autoref{ch:biexciton} is a very promising candidate. As the biexciton state is a 
bound state due to the dynamic interaction, it would be very interesting to see how the exciton-exciton interaction will 
influence the control of energy transfer. 

The method to calculate Green's functions, as presented in \autoref{ch:greenfunc}, are very generic. Future research might
use it to study all sorts of many-body physics in the presence of strong disorders. For example, it is possible to
use the method to calculate the local density of state in a  high-dimensional disordered lattice, which might yields useful information about the 
existence of bound states of quasiparticles in the presence of strong disorders. 


 
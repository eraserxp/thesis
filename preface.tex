%% The following is a directive for TeXShop to indicate the main file
%%!TEX root = diss.tex

\chapter{Preface}

Part of the material in \autoref{ch:biexciton} was published in the article: P. Xiang, M. Litinskaya and R. V. Krems, \textit{ Tunable exciton 
interactions in optical lattices with polar molecules}, Physical Review A {\bf 85}, 061401(R) (2012). The project was identified and 
designed by Roman Krems,  Marina Litinskaya and  the author. The author performed all the numerical calculations and
derived all the analytical expression, except the expression for the biexciton wavefunction, which is the work of Marina 
Litinskaya. Marina Litinskaya also contributed to the numerical calculations in an indirect way by providing analytical 
results that were used to check the results of numerical computation. 

Part of the material in \autoref{ch:energy-transfer} was published in the paper: P. Xiang, M. Litinskaya, E. A. Shapiro, and R. V. Krems, 
 \textit{ Non-adiabatic control of quantum energy transfer in ordered and disordered arrays}, 
 New Journal of Physics {\bf 15}, 063015 (2013).  
The project was designed by E. A. Shapiro, M. Litinskaya, R. V. Krems and the author. The idea of phase kicking was initially
proposed by E. A. Shapiro and the idea of controlling the group velocity of exciton wavepackets by external fields came 
from Marina Litinskaya. The author performed all the numerical calculations and analytical derivations, except the 
derivations in \autoref{sec:singleSiteFocusing} and \autoref{sec:planewaveFocusing}, which are the work of Marina Litinskaya, and the estimates in \autoref{sec:excitationAtoms}, which was done by  Evgeny Shapiro. 

\autoref{ch:greenfunc} is unpublished work by the author. The project was designed by the author under the guidance of
Roman Krems. We are preparing a research publication based on these results. 